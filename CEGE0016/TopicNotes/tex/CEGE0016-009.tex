\chapter{Project Management}
\section{Environmental liability}
\subsection{What is environmental liability?}
The Environmental Damage (Prevention and Remediation) (England) Regulations 2015 impose obligations on operators of economic activities, requiring them to:
\begin{itemize}
    \item Prevent
    \item Limit
    \item Remediate
\end{itemize}
environmental damage caused by their operations. Construction activities have the potential for significant water, soil and air pollution. The risk of this pollution is influenced by:
\begin{itemize}
    \item Contamination conditions
    \item Exposure pathways
\end{itemize}
Construction activities may also result in significant damage to:
\begin{itemize}
    \item Natural habitats
    \item Protected species
    \item Other biodiversity
\end{itemize}
\begin{table}[H]
    \centering
    \begin{tabular}{@{}ll@{}}
        \toprule
        \textbf{Contaminants}                                        & \textbf{Exposure pathways}                        \\
        \midrule
        Metals                                                       & Storm water / sediment run-off                    \\
        Inorganic compounds (e.g., sulphuric acid)                   & Noise and vibration from heavy machinery          \\
        Oils and tars                                                & Groundwater disturbance from borehole drilling    \\
        Pesticides                                                   & Disturbance of unknown pre-existing contamination \\
        Flammable / combustible substances (e.g., petrol and diesel) & Interference with the site's natural resources    \\
        Fibres (e.g., synthetic mineral fibres)                      &                                                   \\
        \bottomrule
    \end{tabular}
    \caption{Table of exemplar contaminants and exposure pathways.}
\end{table}
\subsection{Environmental impact assessment}
Some construction projects require an environmental impact assessment e.g.:
\begin{itemize}
    \item Nuclear power stations
    \item Industrial plants
    \item Large wastewater treatment plants
    \item Motorways
\end{itemize}
Environmental impact assessments are governed by the Town and Country Planning (Environmental Impact Assessment) Regulations 2017. The aim of environmental impact assessment is to protect the environment. They assist local planning authorities in making planning decision on projects that are likely to have significant environmental effects.

There are four stages of an environmental impact assessment:
\begin{enumerate}
    \item Screening and scoping
          \begin{itemize}
              \item Is an environmental impact assessment required and to what scale?
          \end{itemize}
    \item Preparing an environmental statement
          \begin{itemize}
              \item What data will be collected to determine environmental risk?
              \item What measures will be followed to minimise environmental impacts?
              \item Must include:
                    \begin{enumerate}
                        \item A description of the proposed development
                        \item A description of the likely significant environmental effects of the proposed development
                        \item A description of measures to mitigate adverse environmental effects
                        \item A description of the reasonable alternatives, and why the chosen option was selected, accounting for environmental effects
                    \end{enumerate}
          \end{itemize}
    \item Making a planning application
          \begin{itemize}
              \item The environmental statement is publicised electronically and by public notice
          \end{itemize}
    \item Decision making
          \begin{itemize}
              \item Local planning authority / Secretary of State will make the final decision on the permit issuance
          \end{itemize}
\end{enumerate}
\section{Conditions of contract}
\subsection{What is a contract?}
\begin{quoting}
    A contract is a \textbf{promise} or \textbf{set of promises} which the law will \textbf{enforce}.
\end{quoting}
It is a legal document. Contracts form a part of everyday life, whenever you buy or sell something, a contract is made. Contract documents also serve as the foundation of an engineering project.
\subsection{Types of contract}
There are two main types of contract:
\begin{itemize}
    \item Unilateral
          \begin{itemize}
              \item A promise is met with a requested action
              \item Only one party to the contract has a legally binding liability
          \end{itemize}
    \item Bilateral
          \begin{itemize}
              \item An action is met with another action
              \item Both actions are obligations and will be legally enforceable
          \end{itemize}
\end{itemize}
There are two types of bilateral contract:
\begin{itemize}
    \item Formal
          \begin{itemize}
              \item Made under seal
              \item The contractor's liability under the contract lasts for 12 years
          \end{itemize}
    \item Informal
          \begin{itemize}
              \item Signed without a seal
              \item The contractor's liability under the contract lasts for 6 years
          \end{itemize}
\end{itemize}
\subsection{Why are contracts needed?}
Contracts define and cover all activities during a construction project. They also apportion risk between the purchaser and the contractor. Examples of risk covered by contracts are:
\begin{itemize}
    \item Unanticipated cash flow problems
    \item Unanticipated delays
    \item Change in the cost of raw materials
    \item Poor project management
    \item Force majeure
\end{itemize}
\subsection{Force majeure}
A force majeure clause in a contract is intended to protect the contracting parties if part of the contract cannot be performed because of some exceptional event. It usually allows a contractor more time to perform the contract (but does not allow additional costs for rising raw material prices).
\subsection{Conditions of contract}
Terms and conditions refer to the contractual rights and obligations of a party to any contract. The construction industry has developed a range of Standard sets of terms and conditions of contract, over a length of time. It is now common practice to specify one of these Standard documents.
\subsubsection{Example conditions of contract}
There are many examples of Standard terms and conditions of contract. The following Engineering Institutions publish their own examples:
\begin{itemize}
    \item The Institute of Chemical Engineers (IChemE)
    \item The Institute of Civil Engineers (ICE)
    \item The Institution of Mechanical Engineers (IMechE)
    \item The International Federation of Consulting Engineers (FIDIC)
\end{itemize}
\section{Summary}
Environmental liabilities
\begin{itemize}
    \item Definition of environmental liabilities
    \item Description of environmental impact assessments
\end{itemize}
Conditions of contract
\begin{itemize}
    \item Definition of a contract
    \item Distinction between different contract types
    \item Explanations of why contracts exist
    \item Explanation and examples of conditions of contract
\end{itemize}