\documentclass[class=report, crop=false, 12pt,a4paper]{standalone}
\usepackage{enumitem}
\usepackage{multicol}
\usepackage{graphicx}
\usepackage{float}
\usepackage{amsmath}
\usepackage{amssymb}
\usepackage{mathtools}
\usepackage{siunitx}
\usepackage{commath}
\usepackage{array}
\usepackage{booktabs}
\usepackage{xcolor}
\usepackage{natbib}
\usepackage[a4paper,width=150mm,top=25mm,bottom=25mm]{geometry}
\setlength{\parindent}{0pt}
\begin{document}
\chapter{Cost-Benefit Analysis}
\section{What is cost-benefit analysis?}
\begin{quote}
	``Assessing costs and benefits across all affected groups or places matters because even a proposal with a relatively low public sector cost such as new regulation, may have significant effects on specific groups in society, places or business''
\end{quote}
\begin{itemize}
	\item A cost-benefit analysis is the process used to measure the benefits of a decision or taking action relative to the associated costs
	\item If benefits $>$ costs, the decision or action is a good one to take
	\item If costs $>$ benefits, the proposed action or decision should be reconsidered
	\item Cost benefit analysis can also be used to compare altenate decisions or actions
\end{itemize}
Both costs and benefits are required to be expressed in monetary terms, accounting for the time value of money. Costs may be categorised as:
\begin{itemize}
	\item Direct - e.g. labour costs, manufacturing costs, material costs
	\item Indirect - e.g. utilities, rent
	\item Intangible - e.g. reduced productivity because of a new process
	\item Opportunity - lost benefits (opportunities) when pursuing one stratey over another 
\end{itemize}
Benefits may be categorised as:
\begin{itemize}
	\item Direct - e.g. increased revenue and sales
	\item Indirect - e.g. increased consumer interest
	\item Intangible - e.g. improved employee morale
	\item Competitive - e.g. being an industry leader
\end{itemize}
\section{The Benefit-Cost Ratio method}
The Benefit-Cost Ratio (BCR) is defined as the ratio of the equivalent value of benefits to the equivalent value of costs. The equivalent-value measure can be:
\begin{itemize}
	\item Annual value (AV)
	\item Present value (PV)
	\item Future value (FV)
\end{itemize}
The BCR method has been the accepted procedure for making decisions and comparing projects in the public sector for many decades.
\begin{quote}
	if BCR $\geq$ 1, the project is acceptable
\end{quote}
Several different formulations of the BCR method have been developed. We examine two formulations of the BCR method that are commonly used by government agencies:
\begin{itemize}
	\item Conventional BCR method
	\item Modified BCR method
\end{itemize}
Both formulations will lead to identical project acceptability decisions (i.e. BCR $\geq$ 1 or BCR $<$ 1). 
\subsection{Conventional BCR method}
Present value (PV) formulation:
\begin{gather}
	BCR_{PV} = \dfrac{PV_{benefits}}{PV_{costs}}\\
	BCR_{PV} = \dfrac{PV_{benefits}}{I- PV_{MV}+PV_{O\&M}}
\end{gather}
Annual value (AV) formulation:
\begin{gather}
	BCR_{AV} = \dfrac{AV_{benefits}}{AV_{costs}}\\
	BCR_{AV} = \dfrac{AV_{benefits}}{CR + AV_{O\&M}}
\end{gather}
Both ratios lead to identical numerical results. Where:
\begin{itemize}
	\item $I$ is initial investment in the proposed project
	\item $MV$ is market value at the end of useful life
	\item $O\&M$ is operating and maintenance costs
	\item $CR$ is capital-recovery amount (i.e. equivalent cost of $I$, includance an allowance for market or salvage value)
\end{itemize}
\subsection{Modified BCR method}
Present value (PV) formulation:
\begin{gather}
	BCR_{PV} = \dfrac{PV_{benefits}}{PV_{costs}}\\
	BCR_{PV} = \dfrac{PV_{benefits} - PV_{O\&M}}{I - PV_{MV}}
\end{gather}
Annual value (AV) formulation:
\begin{gather}
	BCR_{AV} = \dfrac{AV_{benefits}}{AV_{costs}}\\
	BCR_{AV} = \dfrac{AV_{benefits}-AV_{O\&M}}{CR}
\end{gather}
Both ratios lead to identical numerical results.
\subsection{Convential and modified BCRs}
Remember the formula for PV:
\begin{gather}
	PV = \dfrac{FV}{(1+i)^n}
\end{gather}
Present value (PV) of a future value (FV) in $n$ years, for an interest rate $i$. 
Remember the formula for AV:
\begin{gather}
	AV = PV\dfrac{i(1+i)^n}{(1+i)^n - 1}
\end{gather}
Value of a series of uniform (annual) receipts (AV) that occur at the end of periods (years) 1 to $n$, given their present value (PV) and an interest rate $i$.
\subsection{What value of $i$ to use?}
There are three main considerations when it comnes to what interest rate to use for engineering economy studies of public-sector projects:
\begin{itemize}
	\item the interest rate on borrowed capital
	\item The opportunity cost of capital to the government agency
	\item The opportunity cost of capital to the taxpayers
\end{itemize}
\subsection{Why do conventional and modified BCRs lead to the same decision?}
Convential BCR formulation:
\begin{gather}
	BCR_V = \frac{V_{benefits}}{I - V_{MV}+V_{O\& M}} = \frac{B}{C}
\end{gather}
Where subscript $V$ denotes either PV or AV. Modified BCR formulation:
\begin{gather}
	BCR_V = \frac{V_{benefits} \textcolor{red}{-V_{O\&M}}}{I - V_{MV} + V_{O\&M} \textcolor{red}{-V_{O\&M}}} = \frac{B\textcolor{red}{-X}}{C\textcolor{red}{-X}}
\end{gather}
Both the numerator and denominator differ by the same constant.
\begin{gather}
	\textcolor{red}{\frac{B}{C>1}}\rightarrow B > C \rightarrow B - X > C - X \rightarrow \textcolor{red}{\frac{B-X}{C-X}>1}
\end{gather}
leading to the same decision.
\subsection{Example 1}
The Greater London Authority is considering extending the runways of Stansted Airport so that larger commercial airplanes can use the facility. The land necessary for the runway extension is currently a farmland that can be purchased for £350,000. Construction costs for the runway extension are projected to be £600,000, and the additional annual maintenance costs for the extension are estimated to be £22,500. If the runways are extended, a small terminal will be constructed at a cost of £250,000. The annual operating and maintenance costs for the terminal are estimated at £75,000. Finally, the projected increase in flights will require the addition of two air traffic controllers at an annual cost of £100,000. Annual benefits of the runway extension have been estimated as follows:
\begin{table}
	\centering
	\begin{tabular}{@{}ll@{}}
		\toprule
		\textbf{Description} & \textbf{Annual benefit}\\
		\midrule
		Leasing fee receipts from airlines & £325,000\\
		Passenger airport tax receipts & £65,000\\
		Convenience benefit for residents near Stansted & £50,000\\
		Additional tourism money for London & £50,000\\
		\midrule
		\textbf{Total} & \textbf{£490,000}\\
		\bottomrule
	\end{tabular}
	\caption{Example 1.}
\end{table}
Apply the BCR method with a study of 20 years and a MARR of 10\% per year to determine whether the runways at Stansted airport should be extended.

Information provided:
\begin{gather}
	i = 0.1\\
	n = 20\, \textrm{years}\\
	I = £350000 + £600000 + £250000 = £1200000\\
	AV_{benefits} = £490000\\
	PV_{MV} = AV_{MV} = £0\\
	AV_{O\&M} = £22500 + £75000 + £100000 = £197500
\end{gather}
First, we need to determine PVs and AVs using:
\begin{gather}
	PV = AV\frac{\left(1+i\right)^n - 1}{i\left(1 + i\right)^n}\\
	AV = PV\frac{i\left(1+i\right)^n}{\left(1+i\right)^n - 1}
\end{gather}
\begin{gather}
	PV_{benefits} = £4171646\\
	PV_{O\&M} = £1681429\\
	AV_I = CR = £140951
\end{gather}
Conventional BCRs:
\begin{gather}
	BCR_{PV} = \frac{PV_{benefits}}{I - PV_{MV}+PV_{O\&M}} = 1.448\\
	BCR_{AV} = \frac{AV_{benefits}}{CR + AV_{O\&M}} = 1.448
\end{gather}
Modified BCRs:
\begin{gather}
	BCR_{PV} = \frac{PV_{benefits} - PV_{O\&M}}{I - PV_{O\&M}} = 2.075\\
	BCR_{AV} = \frac{AV_{benefits}- AV_{O\&M}}{CR} = 2.075
\end{gather}
BCR $\geq$ 1 in all cases, so runway should be extended.
\subsection{Issues of concern using BCRs}
\begin{itemize}
	\item The treatment of disbenefits
	\begin{itemize}
		\item Negative consequences to the public resulting from the implementation of a public sector project
	\end{itemize}
	\item The treatment of certain cash flows as additional benefits or reduced costs
\end{itemize}
\subsection{Treatment of disbenefits}
Disbenefits can be incorporated in BCR calculations by:
\begin{itemize}
	\item Reducing benefits accordingly (traditional approach) or
	\item Increasing costs accordingly
\end{itemize}
How do these approaches affect the BCR? How do these approaches affect the final decision? 
\end{document}