\section*{Module information}
\subsection*{Team}
\begin{itemize}
    \item Professor Richard Bucknall
    \item Mr Chris Greenough
    \item Mr Konrad Yearwood - Helpdesk email: k.yearwood@ucl.ac.uk
\end{itemize}
\subsection*{Course Aim}
The aim of this course is to provide students with detailed knowledge and understanding of the design, performance and analysis of electrical power systems.

Students will increase their knowledge and understanding through face-to-face / synchronous lectures, asynchronous (including tutorials) tasks and a computer simulation workshop and demonstrate their learning through summative coursework and an examination.
\subsection*{Student learning outcomes}
\begin{itemize}
    \item Appreciate the components that make up electrical power systems and understand the similarities and differences between large, medium and small scale power systems.
    \item Develop skills needed to be able to design electrical power systems including analytical and computer based methods.
    \item Understand the behaviour of steady-state, transient and faulted networks and appreciate how such behaviour influences design.
    \item Understand the benefits of electrical propulsion for different vehicle types be able to undertake designs.
    \item Appreciate future developments and applications in electrical power and electrical propulsion systems.
\end{itemize}
\subsection*{Assessment}
\begin{itemize}
    \item Coursework - summative assessment exercise based around computer simulations
    \item Examination - two hour examination in January
\end{itemize}
\subsection*{Textbooks}
Kirtley, James. \textit{Electric Power Principles: Sources, Conversion, Distribution and Use.} Wiley. 2020. ISBN: 9781119585305.t
\subsection*{Softwares}
\begin{itemize}
    \item PSCAD
\end{itemize}