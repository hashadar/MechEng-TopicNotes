\documentclass[conference]{IEEEtran}
\IEEEoverridecommandlockouts
% The preceding line is only needed to identify funding in the first footnote. If that is unneeded, please comment it out.
\usepackage{cite}
\usepackage{amsmath,amssymb,amsfonts}
\usepackage{algorithmic}
\usepackage{url}
\makeatletter
\g@addto@macro{\UrlBreaks}{\UrlOrds}
\makeatother
\usepackage{graphicx}
\usepackage{siunitx}
\usepackage{textcomp}
\usepackage{xcolor}
\def\BibTeX{{\rm B\kern-.05em{\sc i\kern-.025em b}\kern-.08em
    T\kern-.1667em\lower.7ex\hbox{E}\kern-.125emX}}
\begin{document}

\title{MECH0010 Case Study Report}

\author{\IEEEauthorblockN{Candidate number: NCWT3}}

\maketitle

\begin{abstract}
This document is a piece of coursework for the Control and Instrumentation module - UCL Mechanical Engineering Department. It seeks to describe the characteristics of a real-world control system, which uses transducers and feedback control. I have chosen a CPU cooler as my control system, commonly found in laptops and PCs.
\end{abstract}

\begin{IEEEkeywords}
control, instrumentation, feedback system, closed loop control
\end{IEEEkeywords}

\section{Product overview}
The product chosen is a CPU cooler called the Intel Thermal Solution, designed for Intel processors. The part number of the CPU cooler is E97379-003 and there are three manufacturers: Delta, Foxconn and Nidec \cite{b1}

The purpose of the Intel Thermal Solution is to provide cooling to the processor. During operation, the processor outputs heat energy, which must be drawn away from the chip to prevent overheating and damage to the processor and surrounding components. This is achieved through the use of a heat-sink and fan assembly. The heat sink is attached to the processor via mounting hardware on the motherboard. A film of thermal paste is also added in between the heat sink and the processor to aid thermal conduction. On top of the heat sink is an impeller, which passes air through the fins of the heat sink, drawing heat energy away. This keeps the CPU cool and prevents overheating. Cooling the processor allows for more performance to be extracted from the processor before overheating occurs. 

Feedback control is used here to control the speed of the fan. The fan does not need to run at the maximum possible speed at all times because the processor's thermal output is not constant, varying with processor load. As the fan also makes audible noise and consumes power whilst in use, it is desirable to be able to have the fan run at lower speeds when possible. The feedback loop in this system looks at the temperature of the processor and sets the fan speed accordingly. When the processor is in high-performance mode, the fan speed can be increased and when the processor is idling, it can be reduced. 
\section{Feedback system}
The goal of the feedback system is to provide the system with the information on the current temperature of the system. It is important to measure the actual temperature of the CPU as simply calculating the expected temperature based on processor load is not sufficient. Factors such as air temperature, airflow, humidity, dust and other environmental factors can all cause the temperature of the CPU to fluctuate. With this information, the closed loop system can make adjustments to the fan speed based. For example, on particularly hot days, the cooler may need to run at higher speeds to achieve the same amount of heat exchange. 
\section{Transducers in feedback system}
The transducer used in our system is part of the processor to which the Thermal Solution is attached. Intel processors measure the temperature of each core by measuring the forward voltage drop of a silicon diode on the chip. This is usually a bipolar junction transistor and the forward voltage drop is temperature dependent. This dependence is described as follows \cite{b2}:
\begin{multline}
    V_{BE} = V_{G0} \left(1-\frac{T}{T_0}\right) + V_{BE0}\left(\frac{T}{T_0}\right) + \left(\frac{nKT}{q}\right)\ln\left(\frac{T_0}{T}\right) \\+ \left(\frac{KT}{q}\right)\ln\left(\frac{IC}{IC_0}\right) \label{azo1}
\end{multline}
where:
\begin{itemize}
    \item $n$ is a device-dependent constant
    \item $q$ is the charge on an electron
    \item $K$ is Boltzmann's constant
    \item $V_{BE0}$ is the bandgap voltage at temperature $T_0$ and current $IC_0$
    \item $V_{G0}$ is the bandgap voltage at absolute zero
    \item $T$ is the temperature in \si{\kelvin}
\end{itemize}
The voltage is measured at two different currents and inputted into \eqref{azo1}, giving us a simplified equation:
\begin{equation}
    \Delta V_{BE} = \left(\frac{KT}{q}\right)\ln\left(\frac{IC_1}{IC_2}\right)
    \label{bandgap}
\end{equation}
Looking at a similar silicon temperature sensor to the one used in a CPU chip (NXP PCT2075), we can look at the data sheet to find how accurate the sensor typically is. The power supply range is from \SI{2.7}{\volt} to \SI{5.5}{\volt}. This gives us temperatures ranging from between \SI{-55}{\celsius} to \SI{+125}{\celsius}. Within this temperature range, the accuracy of the temperature reading is $\pm$\SI{1}{\celsius} between \SI{-25}{\celsius} to \SI{+100}{\celsius} and $\pm$\SI{2}{\celsius} between \SI{-55}{\celsius} to \SI{+125}{\celsius} \cite{b3}. 

The analogue output is converted with an 11-bit ADC that offers a temperature resolution of \SI{0.125}{\celsius}. The measurement is made every \SI{100}{\milli\second} and takes \SI{28}{\milli\second} to convert into bit data. The data is stored in the temperature register and can be read at any time. The register contains two 8-bit data bytes: one Most Significant Byte and one Least Significant Byte. The first significant bit is used as a positive (0) - negative (1) indicator and the other 10 significant bits are used to store the temperature data, the rest of the bits are not relevant to the temperature reading. For example, 011 1111 0001 converted from binary to decimal is 1009. This then needs to be multiplied with \SI{0.125}{\celsius} to get our temperature value, in this case being \SI{126.125}{\celsius}.

A difference between the NXP chip and the transducer on a CPU chip are the activation states of the chips. The NXP chip has a temperature at which it activates ($T_{ots}$ and stays activated and then a reset temperature ($T_{hys}$ hysteresis factor). Compared to a CPU chip, there is only an upper temperature limit, above which the CPU will decrease its clock speed and even reduce its $VCC$ to avoid thermal damage. On an Intel i7-4790k chip, this temperature is \SI{100}{\celsius} (TCC activation temperature) \cite{b4}. The temperature is always monitored, when power is being received to the processor.
\section{Alternative transducers}
An alternative transducer that can be used (and is used for testing purposes on Intel chips) is a thermocouple. This is attached to the integrated heat spreader by machining a groove to the geometric centre of the heat spreader. This thermocouple gives a different set of temperature readings (called $T_{case}$) than the silicon temperature sensors as they are in different positions. As the silicon sensors are embedded in the actual die of the CPU, they are closer to the source of heat than the thermocouple, which is only attached to the heat spreader. This results in the $T_{case}$ temperatures being lower than the core temperatures (which are called junction temperatures $T_{j}$).

A thermocouple operates on the principle of joining two different metals at both ends (junctions). One junction is for measurement and the other is for reference. As the temperature rises or falls at the measuring junction, a voltage is generated. This is directly related to the temperature and can be converted by the reference with an appropriate table of values \cite{b5}. However, since this method of measurement requires altering the chip, it is unsuitable for use in the design environment of the product.

Other transducers which could be used include a thermistor, which operates on the principle of resistance changing with temperature. They are durable, small and affordable. However, the nature of their output is non-linear and they are best suited to measuring a single point of temperature within a limited temperature range \cite{b6}. Their non-linear output would require there to be additional signal processing to extract the temperature data. 

A silicon bandgap temperature sensor is the most suitable transducer in this system. As explored previously we do not need to add any additional components to the chip, resulting in an extremely compact package - important for a processor. They also provide a suitable degree of accuracy over a wide range of temperatures and have a linear output, allowing for easier signal processing. 
\section{Expected nature of signal output from transducer}
The signal output from our silicon temperature sensor is expected to be a linear signal, as defined by \eqref{bandgap}, as long as we keep the ratios of the currents the same. However, as with any system, a totally linear output is not received. 
\section{Costs}
\section{Accuracy of feedback system}
\section{Limitations and improvements}

\section{Introduction}
This document is a model and instructions for \LaTeX.
Please observe the conference page limits. 

\section{Ease of Use}

\subsection{Maintaining the Integrity of the Specifications}

The IEEEtran class file is used to format your paper and style the text. All margins, 
column widths, line spaces, and text fonts are prescribed; please do not 
alter them. You may note peculiarities. For example, the head margin
measures proportionately more than is customary. This measurement 
and others are deliberate, using specifications that anticipate your paper 
as one part of the entire proceedings, and not as an independent document. 
Please do not revise any of the current designations.

\section{Prepare Your Paper Before Styling}
Before you begin to format your paper, first write and save the content as a 
separate text file. Complete all content and organizational editing before 
formatting. Please note sections \ref{AA}--\ref{SCM} below for more information on 
proofreading, spelling and grammar.

Keep your text and graphic files separate until after the text has been 
formatted and styled. Do not number text heads---{\LaTeX} will do that 
for you.

\subsection{Abbreviations and Acronyms}\label{AA}
Define abbreviations and acronyms the first time they are used in the text, 
even after they have been defined in the abstract. Abbreviations such as 
IEEE, SI, MKS, CGS, ac, dc, and rms do not have to be defined. Do not use 
abbreviations in the title or heads unless they are unavoidable.

\subsection{Units}
\begin{itemize}
\item Use either SI (MKS) or CGS as primary units. (SI units are encouraged.) English units may be used as secondary units (in parentheses). An exception would be the use of English units as identifiers in trade, such as ``3.5-inch disk drive''.
\item Avoid combining SI and CGS units, such as current in amperes and magnetic field in oersteds. This often leads to confusion because equations do not balance dimensionally. If you must use mixed units, clearly state the units for each quantity that you use in an equation.
\item Do not mix complete spellings and abbreviations of units: ``Wb/m\textsuperscript{2}'' or ``webers per square meter'', not ``webers/m\textsuperscript{2}''. Spell out units when they appear in text: ``. . . a few henries'', not ``. . . a few H''.
\item Use a zero before decimal points: ``0.25'', not ``.25''. Use ``cm\textsuperscript{3}'', not ``cc''.)
\end{itemize}

\subsection{Equations}
Number equations consecutively. To make your 
equations more compact, you may use the solidus (~/~), the exp function, or 
appropriate exponents. Italicize Roman symbols for quantities and variables, 
but not Greek symbols. Use a long dash rather than a hyphen for a minus 
sign. Punctuate equations with commas or periods when they are part of a 
sentence, as in:
\begin{equation}
a+b=\gamma\label{eq}
\end{equation}

Be sure that the 
symbols in your equation have been defined before or immediately following 
the equation. Use ``\eqref{eq}'', not ``Eq.~\eqref{eq}'' or ``equation \eqref{eq}'', except at 
the beginning of a sentence: ``Equation \eqref{eq} is . . .''

\subsection{\LaTeX-Specific Advice}

Please use ``soft'' (e.g., \verb|\eqref{Eq}|) cross references instead
of ``hard'' references (e.g., \verb|(1)|). That will make it possible
to combine sections, add equations, or change the order of figures or
citations without having to go through the file line by line.

Please don't use the \verb|{eqnarray}| equation environment. Use
\verb|{align}| or \verb|{IEEEeqnarray}| instead. The \verb|{eqnarray}|
environment leaves unsightly spaces around relation symbols.

Please note that the \verb|{subequations}| environment in {\LaTeX}
will increment the main equation counter even when there are no
equation numbers displayed. If you forget that, you might write an
article in which the equation numbers skip from (17) to (20), causing
the copy editors to wonder if you've discovered a new method of
counting.

{\BibTeX} does not work by magic. It doesn't get the bibliographic
data from thin air but from .bib files. If you use {\BibTeX} to produce a
bibliography you must send the .bib files. 

{\LaTeX} can't read your mind. If you assign the same label to a
subsubsection and a table, you might find that Table I has been cross
referenced as Table IV-B3. 

{\LaTeX} does not have precognitive abilities. If you put a
\verb|\label| command before the command that updates the counter it's
supposed to be using, the label will pick up the last counter to be
cross referenced instead. In particular, a \verb|\label| command
should not go before the caption of a figure or a table.

Do not use \verb|\nonumber| inside the \verb|{array}| environment. It
will not stop equation numbers inside \verb|{array}| (there won't be
any anyway) and it might stop a wanted equation number in the
surrounding equation.

\subsection{Some Common Mistakes}\label{SCM}
\begin{itemize}
\item The word ``data'' is plural, not singular.
\item The subscript for the permeability of vacuum $\mu_{0}$, and other common scientific constants, is zero with subscript formatting, not a lowercase letter ``o''.
\item In American English, commas, semicolons, periods, question and exclamation marks are located within quotation marks only when a complete thought or name is cited, such as a title or full quotation. When quotation marks are used, instead of a bold or italic typeface, to highlight a word or phrase, punctuation should appear outside of the quotation marks. A parenthetical phrase or statement at the end of a sentence is punctuated outside of the closing parenthesis (like this). (A parenthetical sentence is punctuated within the parentheses.)
\item A graph within a graph is an ``inset'', not an ``insert''. The word alternatively is preferred to the word ``alternately'' (unless you really mean something that alternates).
\item Do not use the word ``essentially'' to mean ``approximately'' or ``effectively''.
\item In your paper title, if the words ``that uses'' can accurately replace the word ``using'', capitalize the ``u''; if not, keep using lower-cased.
\item Be aware of the different meanings of the homophones ``affect'' and ``effect'', ``complement'' and ``compliment'', ``discreet'' and ``discrete'', ``principal'' and ``principle''.
\item Do not confuse ``imply'' and ``infer''.
\item The prefix ``non'' is not a word; it should be joined to the word it modifies, usually without a hyphen.
\item There is no period after the ``et'' in the Latin abbreviation ``et al.''.
\item The abbreviation ``i.e.'' means ``that is'', and the abbreviation ``e.g.'' means ``for example''.
\end{itemize}
An excellent style manual for science writers is \cite{b7}.

\subsection{Authors and Affiliations}
\textbf{The class file is designed for, but not limited to, six authors.} A 
minimum of one author is required for all conference articles. Author names 
should be listed starting from left to right and then moving down to the 
next line. This is the author sequence that will be used in future citations 
and by indexing services. Names should not be listed in columns nor group by 
affiliation. Please keep your affiliations as succinct as possible (for 
example, do not differentiate among departments of the same organization).

\subsection{Identify the Headings}
Headings, or heads, are organizational devices that guide the reader through 
your paper. There are two types: component heads and text heads.

Component heads identify the different components of your paper and are not 
topically subordinate to each other. Examples include Acknowledgments and 
References and, for these, the correct style to use is ``Heading 5''. Use 
``figure caption'' for your Figure captions, and ``table head'' for your 
table title. Run-in heads, such as ``Abstract'', will require you to apply a 
style (in this case, italic) in addition to the style provided by the drop 
down menu to differentiate the head from the text.

Text heads organize the topics on a relational, hierarchical basis. For 
example, the paper title is the primary text head because all subsequent 
material relates and elaborates on this one topic. If there are two or more 
sub-topics, the next level head (uppercase Roman numerals) should be used 
and, conversely, if there are not at least two sub-topics, then no subheads 
should be introduced.

\subsection{Figures and Tables}
\paragraph{Positioning Figures and Tables} Place figures and tables at the top and 
bottom of columns. Avoid placing them in the middle of columns. Large 
figures and tables may span across both columns. Figure captions should be 
below the figures; table heads should appear above the tables. Insert 
figures and tables after they are cited in the text. Use the abbreviation 
``Fig.~\ref{fig}'', even at the beginning of a sentence.

\begin{table}[htbp]
\caption{Table Type Styles}
\begin{center}
\begin{tabular}{|c|c|c|c|}
\hline
\textbf{Table}&\multicolumn{3}{|c|}{\textbf{Table Column Head}} \\
\cline{2-4} 
\textbf{Head} & \textbf{\textit{Table column subhead}}& \textbf{\textit{Subhead}}& \textbf{\textit{Subhead}} \\
\hline
copy& More table copy$^{\mathrm{a}}$& &  \\
\hline
\multicolumn{4}{l}{$^{\mathrm{a}}$Sample of a Table footnote.}
\end{tabular}
\label{tab1}
\end{center}
\end{table}

\begin{figure}[htbp]
\centerline{\includegraphics{fig1.png}}
\caption{Example of a figure caption.}
\label{fig}
\end{figure}

Figure Labels: Use 8 point Times New Roman for Figure labels. Use words 
rather than symbols or abbreviations when writing Figure axis labels to 
avoid confusing the reader. As an example, write the quantity 
``Magnetization'', or ``Magnetization, M'', not just ``M''. If including 
units in the label, present them within parentheses. Do not label axes only 
with units. In the example, write ``Magnetization (A/m)'' or ``Magnetization 
\{A[m(1)]\}'', not just ``A/m''. Do not label axes with a ratio of 
quantities and units. For example, write ``Temperature (K)'', not 
``Temperature/K''.

\section*{Acknowledgment}

The preferred spelling of the word ``acknowledgment'' in America is without 
an ``e'' after the ``g''. Avoid the stilted expression ``one of us (R. B. 
G.) thanks $\ldots$''. Instead, try ``R. B. G. thanks$\ldots$''. Put sponsor 
acknowledgments in the unnumbered footnote on the first page.

\section*{References}

Please number citations consecutively within brackets \cite{b1}. The 
sentence punctuation follows the bracket \cite{b2}. Refer simply to the reference 
number, as in \cite{b3}---do not use ``Ref. \cite{b3}'' or ``reference \cite{b3}'' except at 
the beginning of a sentence: ``Reference \cite{b3} was the first $\ldots$''

Number footnotes separately in superscripts. Place the actual footnote at 
the bottom of the column in which it was cited. Do not put footnotes in the 
abstract or reference list. Use letters for table footnotes.

Unless there are six authors or more give all authors' names; do not use 
``et al.''. Papers that have not been published, even if they have been 
submitted for publication, should be cited as ``unpublished'' \cite{b4}. Papers 
that have been accepted for publication should be cited as ``in press'' \cite{b5}. 
Capitalize only the first word in a paper title, except for proper nouns and 
element symbols.

For papers published in translation journals, please give the English 
citation first, followed by the original foreign-language citation \cite{b6}.

\begin{thebibliography}{00}
\bibitem{b1} Intel Corporation, ``Specifications and datasheets of Intel\textsuperscript{\tiny\textregistered} Thermal Solution,'' URL: \url{https://www.intel.com/content/www/us/en/support/articles/000055841/processors.html} Accessed: 04-03-2021 18:05:21.
\bibitem{b2} AZoSensors, ``An introduction to silicon bandgap temperature sensors,'' URL: \url{https://www.azosensors.com/article.aspx?ArticleID=369#:~:text=A%20silicon%20bandgap%20temperature%20sensor,integral%20stability%20of%20crystalline%20silicon.} Accessed: 06-03-2021 16:41:56.
\bibitem{b3} NXP Semiconductors, ``PCT2075 product data sheet,'' URL: \url{https://www.nxp.com/docs/en/data-sheet/PCT2075.pdf} Accessed: 06-03-2021 17:19:32.
\bibitem{b4} Intel Corporation, ``Desktop 4th Generation Intel\textsuperscript{\tiny\textregistered} Core\textsuperscript{\tiny TM} Processor Family, Desktop Intel\textsuperscript{\tiny\textregistered} Pentium\textsuperscript{\tiny\textregistered} Processor Family, and Desktop Intel\textsuperscript{\tiny\textregistered} Celeron\textsuperscript{\tiny\textregistered} Processor Family Datasheet – Volume 1 of 2, March 2015,'' URL: \url{http://www.intel.com/content/dam/www/public/us/en/documents/datasheets/4th-gen-core-family-desktop-vol-1-datasheet.pdf} Accessed: 06-03-2021 20:45:11.
\bibitem{b5} Process Parameters Ltd, ``What is a thermocouple? How do they work?'' URL: \url{https://www.processparameters.co.uk/thermocouples-sensor/what-is-a-thermocouple/} Accessed: 06-03-2021 21:43:33
\bibitem{b6} Pyrosales, ``What is a thermistor?'' URL: \url{https://www.pyrosales.com.au/blog/thermocouple-information/what-is-a-thermistor} Accessed: 06-03-2021 22:37:12
\bibitem{b7} M. Young, The Technical Writer's Handbook. Mill Valley, CA: University Science, 1989.
\end{thebibliography}
\end{document}
