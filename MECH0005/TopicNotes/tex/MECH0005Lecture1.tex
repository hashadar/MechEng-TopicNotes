\chapter{Introduction to Thermodynamics}
\section{Second Law of Thermodynamics}
First, let us remind ourselves of the first law of thermodynamics and some of its limitations.
\begin{center}
  Energy can neither be created or destroyed during a process; it can only change forms.
\end{center}
A certain energy balance will hold when a system undergoes change or a thermodynamic process.
\begin{itemize}[noitemsep]
  \item But does not give information on whether the change of state or the process is at all feasible or not.
  \item It cannot indicate whether a metallic bar of uniform temperature can spontaneously become warmer at one end and cooler at the other.
  \item However, if that process did occur, all that law can state is that the energy gained at one end would exactly equal the energy lost at the other.
\end{itemize}
\subsection{Introduction to the second law}
The second law of thermodynamics provides the criterion as to the \emph{probability} of various processes.
\begin{center}
  Spontaneous processes in nature occur only in one direction.
\end{center}
Heat flows from a body at high temperature to a body at low temperature; water always flows downwards etc. The spontaneity of the process is due to finite driving potential, sometimes called 'force' or 'cause', e.g. a temperature or concentration gradient or an electric potential. What happens as a result of this finite driving potential is called the 'flux' or the 'current' or the 'effect' (heat transfer, mass transfer, flow of electric current). This directional law puts a limitation on energy transformation other than that imposed by the first law. The second law also asserts that energy has \emph{quality} as well as \emph{quantity.} The first law is concerned with the quantity of energy and the transformations of energy from one form to another, with no regard to its quality.
\subsection{Qualitative difference between heat and work}
There is also a qualitative difference between heat and work. Energy supplied as work can be completely converted to heat, e.g. paddle wheel work on a liquid in an adiabatic vessel. However, the complete conversion of heat into work is not possible, thus making heat and work not completely interchangeable forms of energy. Also, we considered the problem of a simple steam power plant and by using the Steady Flow Energy Equation (i.e. first law) and the properties of steam, were able to calculate the work done and heat transfers for individual components. However, we are not yet able to understand ways of improving steam engine efficiency.

The second law is based on experimental observation and was the result of the question, 'how efficient can one make a steam engine?' From now we will start by considering engines and define them with the precision that thermodynamics requires. The sort of steam engines we shall discuss are heat devices in boxes with no fluid entering or leaving but with just heat and work crossing the boundaries.
\subsection{Thermodynamic cycles and thermal reservoirs}
Thermodynamic cycles consist of a system, a cold reservoir and a hot reservoir. Reservoirs are regions outside a system that are so large that their intensive properties remain constant. Thermal reservoirs are bodies that exchange an infinite amount of heat with the system. The temperature of a thermal reservoir never changes. For example: Earth's atmosphere, large bodies of water, vapour condensing at a constant pressure. A \emph{heat sink} absorbs heat energy. A \emph{heat source} transfers energy to the system.
\subsection{Heat engine}
A heat engine (or Cyclic Heat Power Plant - CHPP) is a continuously operating thermodynamic system at the boundary of which there are heat and work transfers.
Notes:
\begin{itemize}[noitemsep]
  \item 'Continuously operating' means that the state of the system exhibits only periodic (cyclic) changes.
  \item The heat engine is a thermodynamic system and so no matter crosses the boundary e.g. simple steam power plant and closed-cycle gas turbine plant.
\end{itemize}
As we know that heat transfer to work does not occur. However devices like heat engines have been created, which are special devices which are used to produce work from heat. All heat engines differ but can be characterised by the following:
\begin{itemize}[noitemsep]
  \item They receive heat from a high temperature source (solar energy, oil furnace, nuclear reactor, etc.)
  \item They convert part of this heat to work (usually in the form of a rotating shaft.)
  \item They reject the remaining heat to a low temperature sink (the atmosphere, body of water, etc.)
  \item They operate on a cycle.
\end{itemize}
\begin{figure}[H]
  \centering
  \includegraphics[width = \textwidth]{./img/SimpleSteamPP}
  \caption{A simple steam power plant}
\end{figure}
\begin{figure}[H]
  \centering
  \includegraphics[width = \textwidth]{./img/ClosedCycleGasPP}
  \caption{A closed cycle gas power plant}
\end{figure}
\begin{figure}[H]
  \centering
  \includegraphics[width = 0.7\textwidth]{./img/DieselEngine}
  \caption{A diesel engine}
  \label{dieselengine}
\end{figure}
Diesel engines (Figure \ref{dieselengine}) (and internal combustion engines generally) are not heat engines (CHPP) because matter crosses its boundaries. A jet engine is also not a CHPP because matter, air, fuel and exhaust cross the boundary of the system.
\subsection{Steam power plant}
The work producing device that best suits this definition of a heat engine is the steam power plant, which is an external combustion engine.
\begin{enumerate}[noitemsep]
  \item Combustion takes place outside.
  \item Transferred to steam as heat.
  \item Passed through various devices that transfer its energy to work.
\end{enumerate}
\begin{figure}[H]
  \centering
  \includegraphics[width = \textwidth]{./img/SteamPowerPlantDiagram}
  \caption{Steam power plant operation}
  \label{steampowerplant}
\end{figure}
Figure \ref{steampowerplant} shows a simplified diagram of a steam plant. From this we can see that \(W_{\textrm{net out}} = W_{\textrm{out}} - W_{\textrm{in}}\) (\si{\kilo\joule}). Considering that the change in internal energy is zero for cycles, we can derive, \(W_{\textrm{net out}} = Q_{\textrm{in}} - Q_{\textrm{out}}\) (\si{\kilo\joule}).
\subsection{Thermal efficiency of direct engines / steam power plants}
For heat engines, the desired output is the net work output and the required input is the amount of heat supplied to the working fluid. Thus the thermal efficiency is expressed as:
\begin{align}
  \textrm{Thermal efficiency} & = \frac{\textrm{Net work output}}{\textrm{Total heat input}} \\
  \eta                        & = \frac{W_{\textrm{net out}}}{Q_{\textrm{in}}}               \\
  \eta                        & = 1 - \frac{Q_{\textrm{out}}}{Q_{\textrm{in}}}
\end{align}
Since cyclic devices at practical interest operate between a high temperature \(T_H\) and a low temperature medium \(T_L\), we define these two quantities:
\begin{itemize}
  \item {\(Q_H =\)} magnitude of heat transfer between the cyclic device and the high temperature medium \(T_H\).
  \item {\(Q_L =\)} magnitude of heat transfer between the cyclic device and the low temperature medium \(T_L\).
\end{itemize}
Thus, the previous equations can be written as follows:
\begin{align}
  W_{\textrm{net out}} & = Q_H - Q_L           \\
  \eta                 & = 1 - \frac{Q_L}{Q_H}
\end{align}
Thermal efficiencies of work producing devices are relatively low. Ordinary spark ignition automobile engines have a thermal efficiency of about 25\%. Gas steam power plants reach only 60\%.
\subsection{Can we save \(Q_{out}\)?}
In a steam power plant, the condenser is the device where large quantities of waste heat is rejected to rivers, lakes or the atmosphere. One may ask, can we not save all this waste energy? The answer is a firm \emph{no}. Without a heat rejection process in the condenser, the cycle cannot be completed. Cyclic devices such as steam power plants cannot run continuously, unless the cycle is completed.
\subsection{The Kelvin-Planck statement}
The Kelvin-Planck statement of the second law of thermodynamics is expressed as follows:
\begin{center}
  It is impossible for any device that operates on a cycle to receive heat from a single reservoir and produce a net amount of work.
\end{center}
A heat engine must exchange heat with a low temperature sink as well as a high temperature source to keep operating. This implies that it is impossible to build a heat engine that has a thermal efficiency of 100\%. Complete conversion of heat into work is \emph{not possible.} This is contrary to the fact that 100\% of work can be transferred to heat. This is the directional implication of the second law. This is not due to frictional/non-adiabatic effects. It is a \emph{necessity.}

Proof on page 128 of Cengel (insert edition.)
\subsection{Refrigerators and heat pumps}
From experience, heat always from from high temperature to low temperature. These reverse process, however, cannot occur by itself. This reverse process requires special devices called refrigerators. Refrigerators are cyclic devices and the working fluid is called a refrigerant. A frequently used refrigeration cycle is the 'vapour-compression refrigeration cycle.' This involves a compressor, a condenser, an expansion valve and an evaporator.
\begin{figure}[H]
  \centering
  \includegraphics[width = \textwidth]{./img/RefrigerationCycle}
  \caption{Refrigeration cycle}
\end{figure}
\subsection{Coefficient of performance}
The efficiency of a refrigerator is called the coefficient of performance (COP), denoted by \(\textrm{COP}_R\). The objective of a refrigerator is to: remove heat from the refrigerated space by being provided with a work input \(W_{\textrm{net in}}\). Thus:
\begin{equation}
  \textrm{COP}_R = \frac{Q_L}{W_{\textrm{net in}}} = \frac{\dot{Q}_L}{\dot{W}_{\textrm{net in}}}
\end{equation}
Also, since \(W_{\textrm{net in}} = Q_{\textrm{H}} - Q_L\)
\begin{align}
  \textrm{COP}_R = \frac{Q_L}{Q_H-Q_L} \\
  \textrm{COP}_R = \frac{1}{\frac{Q_H}{Q_L} - 1}
\end{align}
COP can also be greater than unity. This means the heat removed can be greater than the work input. Whereas thermal efficiency can never be greater than 1.
\subsection{Heat pumps}
ANother device that transfers heat from a low temperature medium to a high temperature one is a heat pump.
Here are some differences between refrigerators and heat pumps.
\begin{itemize}[noitemsep]
  \item Refrigerators:
        \begin{itemize}
          \item Maintain the refrigerated space at a low temperature by removing heat from it.
          \item This extracted heat is then discharged to a high temperature medium out of necessity.
        \end{itemize}
  \item Heat pumps:
        \begin{itemize}
          \item Maintain a heated spaced at a high temperature.
          \item This is accomplished by absorbing heat from a low temperature source e.g. well, water, cold air and then supplying this heat to the high temperature medium such as a house.
        \end{itemize}
\end{itemize}
A heat pump runs on the same cycle as a refrigerator. The measure of performance of a heat pump is also expressed in terms of coefficient of performance \(\textrm{COP}_{HP}\).
\begin{align}
  \textrm{COP}_{HP} & = \frac{Q_H}{W_{\textrm{net in}}} = \frac{Q_H}{Q_H - Q_L} = \frac{1}{1 - \frac{Q_L}{Q_H}} \\
  \textrm{COP}_{HP} & = \textrm{COP}_R + 1
\end{align}
This relation shows us that $\textrm{COP}_{HP} > \textrm{COP}_R$ at all times as $\textrm{COP}_R$ is always positive.
\subsection{Air conditioners}
Air conditioners are basically refrigerators where refrigerated space is a room instead of a food compartment. The same air conditioning unit can be used as a heat pump in winter by installing it backwards. In this mode, the unit absorbs heat from the cold outside and delivers it to the room.
\subsection{The Clausius statement}
The Kelvin-Planck statement was for direct heat engines. The Clausius statement is for reversed heat engines. The statement is as follows:
\begin{center}
  It is impossible to construct a device that operates in a cycle and produces no effect other than the transfer of heat from a low temperature body to a higher temperature body.
\end{center}
This means that it is impossible to construct a refrigerator that operates without an input of work. \(W_{\textrm{net in}} \neq 0 \therefore \textrm{COP}_R \neq 0\).
\subsection{Reversible and irreversible processes}
As a result of the second law we know that the complete conversion of heat to work is impossible. Thus, an efficiency of 100\% cannot be achieved for a heat engine. Then what is the maximum efficiency of a heat engine/CHPP? Before we answer this question, we need to define an idealised process first, which is called a reversible process.
Reversible process:
\begin{center}
  A process that can be reversed without having any trace to the surroundings. Both the system and the surroundings are returned to their initial states ath the end of the reverse process. Thus, for the combined forward and reverse processes the \emph{net heat} and \emph{net work} transfer is \emph{zero}.
\end{center}
Reversible processes do not occur in nature naturally. They are merely an idealisation of real processes. They can be approximated by actual devices but can never be achieved. We consider theme even though they are impossible, as they are easy to analyse and they can be used for comparison.
\subsection{Irreversibilities}
The factors that cause a process to be irreversible are called irreversibilities.
They include:
\begin{itemize}[noitemsep]
  \item Friction.
  \item Unrestrained expansion.
  \item Mixing of two fluids.
  \item Heat transfer across a finite temperature difference.
  \item Electrical resistance.
  \item Inelastic deformation of solids.
  \item Chemical reactions.
\end{itemize}
A reversible process includes none of these. There are three types of reversible process:
\begin{itemize}[noitemsep]
  \item Externally reversible.
  \item Internally reversible.
  \item Totally reversible.
\end{itemize}
\subsubsection{Externally reversible}
No irreversibilities exist in the surroundings. Heat transfer can occur between the system and the surroundings but only with an infinitesimal temperature difference. There may still be irreversibilities with the system.
\subsubsection{Internally reversible}
No irreversibilities exist within the system. The system moves slowly and without friction through a series of equilibrium states. Irreversibilities may exist in the surroundings usually due to heat transfer through a finite temperature difference.
\subsubsection{Totally reversible}
A process is called totally reversible, or simply reversible, if it involves no irreversibilities within the system or its surroundings. A totally reversible process involves no heat transfer through a finite temperature difference, no non quasi-equilibrium changes and no friction or other dissipative effects.
\subsection{The Carnot cycle}
A quick recap of the content covered so far.

Heat engines are cyclic devices. The working fluid returns back to its original state at the end of each cycle. In part one (of the cycle), there is work done by the fluid and in part two, work is done on the working fluid. This difference between these two parts is the net work delivered by the engine. Cycle efficiency can be maximised by using processes that require the least amount of work. This is achieved by using \emph{reversible processes}. Reversible cycles cannot be achieved in practice because the irreversibilities associated with each process cannot be eliminated. However, they provide upper limits to the performance of real cycles. The most famous reversible cycle is the Carnot cycle. A theoretical heat engine that operates on a Carnot cycle is called a Carnot heat engine. It is composed of four reversible processes: two isothermal and two adiabatic. It can be executed either in a closed system or a steady flow system.

Consider a system that consists of a gas contained in an adiabatic piston-cylinder device. The insulation of the cylinder head is such that it may be removed to bring the cylinder into contact with reservoirs to provide heat transfer. The four reversible processes that make up the Carnot cycle are as follows:
\subsubsection{A: Reversible isothermal expansion (Process 1-2, constant $T_H$)}
\begin{enumerate}[noitemsep]
  \item The gas is at temperature $T_H$.
  \item The cylinder head is in close contact with a source at temperature $T_H$.
  \item The gas is allowed to expand slowly and does work on the surrounding.
  \item But as the temperature drops by $dT$, some heat is transferred from the source to the gas, raising the temperature back to $T_H$. Thus, the temperature remains constant at $T_H$.
\end{enumerate}
Since the temperature difference never exceeds $dT$, it is a reversible heat transfer process. It continues until reaching position 2. The total amount of heat transferred to the gas during this process is $Q_H$.
\subsubsection{B: Reversible adiabatic expansion (Process 2-3, temperature drops from $T_H \rightarrow T_L$)}
\begin{enumerate}[noitemsep]
  \item The heat source in A is removed and the insulation is added, so the system becomes adiabatic.
  \item The gas expands slowly, doing work on the surroundings until its temperature drops from $T_H$ to $T_L$.
\end{enumerate}
The piston is assumed to be frictionless and the process to be quasistatic equilibrium, so the process is reversible and adiabatic.
\subsubsection{C: Reversible isothermal compression (Process 3-4, $T_L =$ constant)}
\begin{enumerate}[noitemsep]
  \item Insulation is removed and replaced with a low temperature sink at temp $T_L$.
  \item The piston is pushed inwards by an external force, doing work on the gas.
  \item As the gas is compressed, its temperature tends to rise.
  \item If it rises by $dT$, it drops back to $T_L$ as heat is transferred from the gas to the sink.
  \item Thus, temperature remains constant at $T_L$.
\end{enumerate}
This is a reversible heat transfer process as the temperature difference never exceeds $dT$. The amount of heat rejected is $Q_L$.
\subsubsection{D: Reversible adiabatic compression (Process 4-1, temperature rises from $T_L \rightarrow T_H$)}
\begin{enumerate}[noitemsep]
  \item Insulation is added back.
  \item Gas is compressed.
  \item The gas returns to its original state.
  \item Temperature rises from $T_L$ to $T_H$ which completes the cycle.
\end{enumerate}
All of the steps of the Carnot cycle can be summarised on a P-V graph (Figure \ref{carnotcycle}). Being a reversible cycle, the Carnot cycle is the most efficient cycle operating between two specified temperature limits.
\begin{figure}[H]
  \centering
  \includegraphics[width = \textwidth]{./img/CarnotCycle}
  \caption{Carnot cycle}
  \label{carnotcycle}
\end{figure}
\subsection{Reversed Carnot cycle} \label{reversecarnot}
The Carnot cycle is totally reversible. Thus, if each process is reversed, it becomes the Carnot refrigeration cycle. Heat and work interactions are reversed.
\begin{figure}[H]
  \centering
  \includegraphics[width = \textwidth]{./img/ReversedCarnotCycle}
  \caption{Reversed Carnot cycle}
\end{figure}
\subsection{The Carnot principles}
The second law of thermodynamics puts limits on the operation of cyclic devices as expressed by the Kelvin-Planck and Clausius statements. A heat engine cannot operate by exchanging heat with a single reservoir and a refrigerator cannot operate without a net energy input from an external source. These bring rise to the Carnot principles:
\begin{enumerate}[noitemsep]
  \item The efficiency of an irreversible heat engine is always less than the efficiency of a reversible one operating between the same two reservoirs.
  \item The efficiency of all reversible heat engines operating between the same two reservoirs are the same.
\end{enumerate}
\subsubsection{Proof for statement 1 (of Carnot principles)}
insert proof
\subsection{The thermodynamic temperature cycle}
insert proof
\begin{equation}
  \left( \frac{Q_M}{Q_L} \right)_{rev} = \frac{T_H}{T_L} \leftarrow \textrm{use Kelvin}
\end{equation}
\subsection{The Carnot heat engines}
Any heat engine has a thermal efficiency of $\eta_{th}$ given by the formula
\begin{equation}
  \eta_{th} = 1 - \frac{Q_L}{Q_H}
\end{equation}
For reversible heat engines, such as a Carnot engine, this becomes:
\begin{equation}
  \eta_{th, \ rev} = 1 - \frac{T_L}{T_H}
\end{equation}
Using the formula shown in the previous part. This us called the Carnot efficiency, since the Carnot heat engine is the best known reversible engine. This is the highest efficiency a heat engine operating between two thermal energy reservoirs at temperature $T_H$ and $T_L$ can have. If you calculate the $\eta_{th}$ of a heat engine, their nature can be deduced as follows:
\begin{align}
  \eta_{th} < \eta_{th, \ rev} & \rightarrow \textrm{Irreversible heat engine} \\
  \eta_{th} = \eta_{th, \ rev} & \rightarrow \textrm{Reversible heat engine}   \\
  \eta_{th} > \eta_{th, \ rev} & \rightarrow \textrm{Impossible heat engine}
\end{align}
From the equation $\eta_{th, \ rev} = 1 - \frac{T_L}{T_H}$, we can see that the efficiency of a Carnot heat engine increases as $T_H$ is increased, or as $T_L$ is decreased. As $T_L$ decreases so does the amount of heat rejected. The same is also true for actual heat engines. The thermal efficiency of actual heat engines can be maximised by:
\begin{itemize}[noitemsep]
  \item Supplying heat to the engine at the highest possible temperature (limited by material strength).
  \item And rejecting heat from the engine at the lowest possible temperature (limited by the temperature of the cooling medium such as rivers, lakes or the atmosphere).
\end{itemize}
\subsection{The Carnot refrigerator and heat pump}
The COP of any heat pump and refrigerator is given by:
\begin{align}
  \textrm{COP}_R = \frac{1}{\frac{Q_H}{Q_L} - 1} \\
  \textrm{COP}_{HP} = \frac{1}{1 - \frac{Q_L}{Q_H}}
\end{align}
Using the thermodynamic scale formulas, the COP for heat pumps and refrigerators running on a reverse Carnot cycle is:
\begin{align}
  \textrm{COP}_{R, \ rev} = \frac{1}{\frac{T_H}{T_L} - 1} \\
  \textrm{COP}_{HP, \ rev} = \frac{1}{1 - \frac{T_L}{T_H}}
\end{align}
These are the highest COP that a refrigerator or a heat pump operating between the temperature limits at $T_L$ and $T_H$ can have.
\begin{equation}
  \textrm{COP}_{R} =
  \begin{cases}
    < \textrm{COP}_{R, \ rev} \\
    = \textrm{COP}_{R, \ rev} \\\
    > \textrm{COP}_{R, \ rev}
  \end{cases}
  \textrm{\& COP}_{HP} =
  \begin{cases}
    < \textrm{COP}_{HP, \ rev} \rightarrow \textrm{Irreversible} \\
    = \textrm{COP}_{HP, \ rev} \rightarrow \textrm{Reversible}   \\
    > \textrm{COP}_{HP, \ rev} \rightarrow \textrm{Impossible}
  \end{cases}
\end{equation}