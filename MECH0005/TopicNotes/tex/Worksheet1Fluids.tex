\section{Introduction to Fluid Dynamics}
\subsection{Question 1}
\subsubsection{a}
Let us start with the equation for viscosity:
\begin{equation}
  \tau = \mu \frac{\partial u_y}{\partial x}
\end{equation}
Rearranging the following equation for \(\partial u_y / \partial x \) gives us:
\[ \frac{\partial u_y}{\partial x} = \frac{\tau}{\mu}\]
Inputting our variables:
\[ \frac{\partial u_y}{\partial x} = \frac{0.4}{5} = 0.08 \ \si{\per\second}\]

\subsubsection{b}
We need to find the shear stress exerted by the block on the oil, found using the following equation:
\begin{equation}
  \tau = \frac{F_t}{A}
\end{equation}
The force that the block exerts tangentially on the oil covered inclined plane surface is given by:
\[ F_t = 10\cdot g \cdot sin(20) \ \si{\newton} \]
The area of contact between the oil and the block is
\[ A = 0.1 \ \si{\meter \squared}\]
Inputting into equation (2):
\[ \tau = 100\cdot g \cdot sin(20) \]
We are told the velocity distribution through the oil is linear and hence can make the following simplification:
\[ \frac{\partial u_x}{\partial y} = \frac{u_{block}}{D} \]
\[ \tau = \mu \frac{u_{block}}{D} \]
\[ u_{block} = \frac{\tau D}{\mu} \]
\[ u_{block} = \frac{100 \cdot 9.81 \cdot sin(20) \cdot 1\times 10^{-4}}{5} \]
\[ u_{block} = 0.0067 \ \si{\meter \per \second} \]
\subsection{Question 2}