\documentclass[class=report, crop=false, 12pt,a4paper]{standalone}
\usepackage{enumitem}
\usepackage{multicol}
\usepackage{etoolbox}
\AtBeginEnvironment{quote}{\singlespacing\small}
\usepackage{setspace}
\onehalfspacing
\usepackage{graphicx}
\usepackage{float}
\usepackage{amsmath}
\usepackage{amssymb}
\usepackage{siunitx}
\sisetup{detect-all}
\begin{document}
\section{Gas power cycles}
\subsection{Thermodynamic cycles}
Two important areas of application of thermodynamics are power generation and refrigeration. Both are accomplished by systems that operate on a thermodynamic cycle.
\begin{itemize}[noitemsep]
  \item Power cycles: operated as engines.
  \item Refrigeration cycles: operated as refrigerators, air conditioners, heat pumps, etc.
  \item Gas cycles: the working fluid remains in the gaseous phase throughout the entire cycle.
  \item Vapour cycles: the working fluid exists in the vapour phase during part of the cycle and in the liquid phase during another part. 
  \item Closed cycles: the working fluid is returned to the initial state at the end of the cycle and is recirculated.
  \item Open cycles: the working fluid is renewed at the end of the cycled instead of being recirculated.
\end{itemize}
\subsection{Heat engines}
In automobile engines, the combustion gases are exhausted and replaced by the fresh air-fuel mixture at the end of each cycle. The engine operates on a mechanical cycle but the working fluid does not go through a complete thermodynamic cycle. 
\begin{itemize}[noitemsep]
  \item Internal combustion engine: heat is supplied by burning the fuel within the system boundaries e.g. automobile engines.
  \item External combustion engine: heat is supplied to the working fluid from an external source such as a furnace e.g. steam power plants.
\end{itemize}
\subsection{Basic considerations in the analysis of power cycles}
The cycles encountered in actual devices are difficult to analyse because of the presence of complicating effects such as friction and the absence of sufficient time for establishment of the equilibrium conditions during the cycle. Thus, we utilise some idealisations. An ideal cycle is an actual cycle stripped of all its internal irreversibilities, producing a graph made up totally of internally reversible processes. The ideal cycle can be used to show the general characteristics. Recall,
\begin{align}
  \eta_{th} &= \frac{W_{net}}{Q_{in}} \textrm{ or}\\
  \eta_{th} &- \frac{w_{net}}{q_{in}} 
\end{align}
\begin{figure}[h]
  \centering
  \includegraphics[width = 0.5\textwidth]{../img/actualvsidealcycle}
  \caption{How an ideal cycle compares to the actual cycle.}
  \label{fig:actualvsidealcycle}
\end{figure}
Also, heat engines on a totally reversible cycle (such as the Carnot cycle) have the highest thermal efficiency of all heat engines operating between the same temperature levels. The question arises: why do we not use the Carnot cycle? This is due to hardware. It is difficult to implement it. Thus, other ideal cycles are used.
\begin{equation}
  \eta_{th, \textrm{ totally reversible}} > \eta_{th, \textrm{ internally reversible}} = \eta_{th, \textrm{ ideal gas}}
\end{equation}
This applies between the same temperature limits. We also have:
\begin{equation}
  \eta_{th, \textrm{ ideal cycle}} > \eta_{th, \textrm{ actual cycle}}
\end{equation}
The idealisations and simplifications commonly employed in the analysis of power cycles can be summarised as follows:
\begin{enumerate}[noitemsep]
  \item The cycle does not involve any friction. Thus, the working fluid does not experience any pressure drop as it flows in pipes or devices such as heat exchangers.
  \item All expansion and compression processes take place in a quasi - equilibrium manner.
  \item The pipes connecting various components of a system are well insulated and heat transfer through them is negligible.
\end{enumerate}
Kinetic energy and potential energy are also generally neglected as their changes are negligible.
\begin{align}
  \textrm{KE and PE neglected for} & \begin{cases}
    \textrm{turbines}\\
    \textrm{compressors}\\
    \textrm{pumps}
  \end{cases}\\
  \textrm{KE neglected for} & \begin{cases}
    \textrm{condensers}\\
    \textrm{boilers}\\
    \textrm{mixing chambers}
  \end{cases}\\
  \textrm{KE not neglected for} & \begin{cases}
    \textrm{nozzles}\\
    \textrm{diffusers}
  \end{cases}
\end{align}
Also, note that for P-V and T-S diagrams, the area enclosed, for both, is the net work produced during a cycle.
\subsection{T-S diagram tips and tricks}
\begin{itemize}[noitemsep]
  \item A heat - addition process proceeds in the direction of increasing entropy. 
  \item A heat - rejection process proceeds in the direction of decreasing entropy. 
  \item An isentropic (internally reversible, adiabatic) process proceeds at constant entropy. 
  \item The area under the process curve on a T-S diagram represents the heat transfer for that process.
\end{itemize}
\subsection{The Carnot cycle and its value in engineering}
\begin{equation}
  \eta_{th, \textrm{ Carnot}} = 1 - \frac{Q_L}{Q_H} = 1 - \frac{T_L}{T_H}
  \label{carnotefficiency}
\end{equation}
Thermal efficiency increases with an increase in the average temperature at which heat is supplied to the system, or with a decrease in the average temperature at which heat is rejected from the system. The highest temperature in the cycle is limited by the maximum temperature that the components of the heat engine, such as the piston or the turbine blades, can withstand. The lowest temperature is limited by the temperature of the cooling medium utilised in the cycle such as a lake, a river or the atmospheric air.
\subsubsection{Proof question}
Show that the thermal efficiency of a Carnot cycle operating between the temperature limits $T_H$ and $T_L$ is solely a function of these two temperature and is given by equation \ref{carnotefficiency}.

Carnot cycle is:
\begin{enumerate}[noitemsep]
  \item Isothermal heat addition.
  \item Isentropic expansion.
  \item Isothermal heat rejection.
  \item Isentropic compression.
\end{enumerate}
\begin{figure}
  \centering
  \includegraphics[width = 0.5\textwidth]{../img/carnotq}
  \caption{T-S diagram for Carnot cycle}
\end{figure}
\begin{align}
  q_{in} &= \textrm{Area under graph as processes are totally reversible}\\
  &= T_H(S_2 - S_1) = T_H(S_3 - S_4)\\
  q_{out} &= \textrm{Area under process 3-4}\\
  &= T_L(S_3 - S_4) = T_L(S_2 - S_1)\\
  \eta_{th} &= 1 - \frac{q_{out}}{q_{in}} = 1 - \frac{T_L(S_2 - S_1)}{T_H(S_2 - S_1)}\\
  \eta_{th, \textrm{ const}} &= 1 - \frac{T_L}{T_H}
\end{align}
\subsection{Air-standard assumptions}
Even though internal combustion engines operate on a mechanical cycle (the piston returns to its starting position at the end of each revolution), the working fluid does not undergo a complete thermodynamic cycle. It is thrown out of the engine at some point in the cycle (as exhaust gases) instead of being returned to its initial state. Working on an open cycle is the characteristic of all internal combustion engines. To reduce the analysis to a manageable level, we utilise the following approximations, commonly known as the \emph{air-standard assumptions.} Those cycles that run this are called air-standard cycles.
\begin{enumerate}[noitemsep]
  \item The working fluid is air, which continuously circulates in a closed loop and always behaves as an ideal cycle.
  \item All the processes are internally reversible.
  \item The combustion process is replaced by heat-addition process from an external source.
  \item The exhaust process is replaced by a heat rejection process that restores the working fluid to its initial state.
\end{enumerate}
Also we assume that air has constant specific heats whose value is determined at room temperature (25\si{\celsius}). When this assumption is applied, true air-standard assumptions are called the cold-air-standard assumptions.
\subsection{An overview of reciprocating engines}
The basic components of a reciprocating engine are shown in the diagrams.
\begin{figure}
  \centering
  \includegraphics[width = \textwidth]{../img/ReciprocatingEngineDiagram}
  \caption{Basic components of a reciprocating engine.}
\end{figure}
\begin{itemize}[noitemsep]
  \item TDC: top dead center - position of the piston where it forms the smallest volume.
  \item BDC: bottom dead center - position of the piston where it forms the largest volume.
  \item Stroke: distance between TDC and BDC.
  \item Bore: diameter of the piston (internal diameter of the cylinder).
  \item Clearance volume: minimum volume formed.
  \item Displacement volume: the volume displaced by the piston as it moves between TDC and BDC.
\end{itemize}
The ratio of the maximum volume formed in the cylinder to the (minimum) clearance volume is called the compression ratio, r, of the engine.
\begin{gather}
  \textrm{Compression ratio: } r = \frac{V_{max}}{V_{min}} = \frac{V_{BDC}}{V_{TDC}}\\
  \textrm{Displacement volume} = \textrm{no. of cylinders} \times \textrm{stroke length} \times \textrm{bore area}
\end{gather}
The MEP (mean effective pressure) is a constant theoretical pressure, that if it acts on the piston to produce work, it would produce the same amount as that during an actual cycle. 
\begin{gather}
  \therefore W_{net} = \textrm{MEP} \times \textrm{piston area} \times \textrm{stroke}\\
  \textrm{or } W_{net} = \textrm{MEP} \times \textrm{displacement volume}\\
\end{gather}
or rearranging for MEP gives:
\begin{equation}
  \textrm{MEP} = \frac{W_{net}}{(V_{max} - V_{min})} = \frac{w_{net}}{(v_{max} - v_{min})}
\end{equation}
Engines with larger MEP values deliver more net work per cycle and thus perform better. 
\begin{itemize}[noitemsep]
  \item Reciprocating engines
    \begin{itemize}[noitemsep]
      \item SI (spark ignition) engines - combustion is initiated using a spark plug. An example of this is the Otto cycle.
      \item CI (compression ignition) engines - self-ignited as compression of gas (air-fuel) mixture raises the temperature. an example of this is the Diesel cycle.
    \end{itemize}
\end{itemize}
\subsection{Otto cycle: ideal cycle for SI engines}
In most SI engines, the piston executes four complete strokes (two mechanical cycles) within the cylinder and the crankshaft completes two revolutions for each thermodynamic cycle. These engines are called four-stroke IC engines.
\begin{figure}
  \centering
  \includegraphics[width = \textwidth]{../img/OttoCycleDiagram}
  \caption{T-S and P-V diagrams for Otto cycle, alongside mechanical cycle breakdown for the ideal and actual cases.} 
\end{figure}
The air-standard Otto cycle is described below in terms of processes.
\begin{itemize}[noitemsep]
  \item Process 1-2: isentropic compression.
  \item Process 2-3: constant volume heat addition.
  \item Process 3-4: isentropic expansion.
  \item Process 4-1: constant volume heat removal
\end{itemize}
In these diagrams, the Otto cycle replaces the combustion process of the actual engine operation with the constant - volume heat addition process. The constant-volume heat removal process replaces the exhaust blowdown. The original figure of the P-V diagram only shows two strokes of the engine. The intake and exhaust strokes are the missing ones. That is accounted for by drawing the graph with a process 0-1 added (the blue line).
\subsection{Efficiency of the Otto cycles}
Lets apply the first law to an Otto cycle during each process separately:
\begin{align}
  \Delta U &= \Delta Q + \Delta W \textrm{ assuming } \Delta PE = \Delta KE \approx 0\\
  \Delta u &= \Delta q + \delta w\\
  \Delta u &= (q_{in} - q_{out}) + (w_{in} - w_{out})
\end{align}
For the heat transfer processes (process 2-3 and process 4-1), $\Delta w = 0$ as there is no boundary work as it is a constant volume process. Therefore for process 4-1 and 2-3:
\begin{equation}
  \Delta u = q_{in} - q_{out}
\end{equation}
For process 4-1: heat removal
\begin{align}
  \Delta u &= -q_{out}\\
  u_1 - u_4 &= - q_{out}\\
  q_{out} &= u_4 - u_1\\
  q_{out} &= c_V(T_4 - T_1)
\end{align}
For process 2-3: heat addition
\begin{align}
  \Delta u &= q_{in}\\
  q_{in} &= u_3 - u_2\\
  q_{in} &= c_V(T_3 - T_2)\\
  \therefore \eta_{th , \ otto} = \frac{w_{net}}{q_{in}} = 1 - \frac{q_{out}}{q_{in}} &= 1 - \frac{T_4 - T_1}{T_3 - T_2} = 1 - \frac{T_4 \left( \frac{T_4}{T_1} -1 \right)}{T_3 \left( \frac{T_3}{T_4} - 1 \right)}
\end{align}
But processes 1-2 and 3-4 are isentropic (adiabatic, internally reversible) and $v_2 = v_3$ and $v_4 = v_1$. From the definition of adiabatic processes:
\begin{align}
  \frac{T_1}{T_2} &= \left( \frac{v_2}{v_1} \right)^{\gamma -1}\\
  \frac{T_4}{T_3} &= \left( \frac{v_3}{v_4} \right)^{\gamma -1}\\
  \therefore \frac{T_1}{T_2} &= \frac{T_4}{T_3}\\
  \therefore \eta-{th, \ otto} &= 1 - \frac{T_4 \left( \frac{T_4}{T_1} -1 \right)}{T_3 \left( \frac{T_3}{T_4} - 1 \right)}\\
  \eta_{th, \ otto} &= 1 - \frac{T_3}{T_4}\left[ \frac{\frac{T_4}{T_3} \frac{T_3}{T_2} \frac{T_2}{T_1} - 1}{\frac{T_3}{T_2} -1} \right]
\end{align}
The compression ratio r is
\begin{equation}
  r = \frac{V_{max}}{V_{min}} = \frac{V_1}{V_2} = \frac{V_4}{V_3}
\end{equation}
Since it is a fixed mass:
\begin{equation}
  r = \frac{v_1}{v_2} = \frac{v_4}{v_3}
\end{equation}
Remember process 2-3 and 4-1  are constant volume processes. Hence, 
\begin{align}
  Pv &= mRt\\
  \frac{v}{mR} &= \frac{T}{P}\\
  \therefore \frac{T_2}{P_2} &= \frac{T_3}{P_3} \rightarrow \frac{P_3}{P_2} = \frac{T_3}{T_2} \textrm{ and}\\
  \therefore \frac{T_4}{P_4} &= \frac{T_1}{P_1} \rightarrow \frac{P_1}{P_4} = \frac{T_1}{T_4}
\end{align}
\end{document}