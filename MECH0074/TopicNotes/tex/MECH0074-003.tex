\chapter{Large Spatial and Temporal Variations of Temperature}
\section{Introduction}
Many different engineering materials are subject to intense heating and cooling in localised regions. As with our previous discussion about materials, the temperature might be spatially or temporally variable. In Lecture 13, we looked at the thermoelastic response of materials subject to small temperature variations. Their response can be dealt with via a linear elastic model. In this chapter, we look at the effect of a large temperature applied to a material which are sufficient to generate a plastic response and how the spatial and temporal variation affects the material properties.

The incandescent light build initially failed due to the thermal fatigue and melting problems. This was largely a material selection and corrosion problem. Turning a light on and off generates enormous thermal stresses, but keeping it on continuously is fine. The filament is made of tungsten which has a high melting point. The inert gases around the filament stop evaporation. Most of energy dissipated is thermal.
\subsubsection{Types of heating processes}
\begin{itemize}
    \item \textbf{Mechanical heating}, usually by friction
    \item \textbf{Electrical heating}, using the material itself for energy release (e.g. induction heating), or more commonly by external means with an electrical resistance made of Nichrome (60\% Ni, 25\% Fe, 15\% Cr) or Kanthal (70\%, 24\% Cr, 5\% Al)
    \item \textbf{Radiation heating}, either with microwaves, infrared radiation from heated wires protected inside a quartz-glass (wires can be made of tungsten, carbon, Kanthal or Nichrome; naked Nichrome coiled wire was also used in the past), or using visible radiation (with a laser).
    \item \textbf{Chemical heating}, mainly by combustion, but also by hydrogen formation after atomic hydrogen is produced in an electric car, for instance.
    \item \textbf{Nuclear heating}, by nuclear fission or fusion
\end{itemize}
\begin{figure}[H]
    \centering
    \includegraphics[width = 0.8\textwidth]{img/figure41.png}
    \caption{Heating processes.}
\end{figure}
\begin{figure}[H]
    \centering
    \includegraphics[width = 0.8\textwidth]{img/figure40.png}
    \caption{Localised heating processes.}
\end{figure}
\section{Practical application of heat to a material}
Any time a material is heated, the heat is applied spatially and temporally. The characteristics scales have quite different effects on the material and structure.
\begin{table}[H]
    \centering
    \begin{tabular}{@{}lll@{}}
        \toprule
             & \textbf{Localised} & \textbf{Uniform} \\
        \midrule
        Slow & Welding            & Heat treatment   \\
        Fast & Thermal shocking   & Quenching        \\
        \bottomrule
    \end{tabular}
    \caption{Spatial and temporal heat application.}
\end{table}
\subsection{Welding}
\begin{itemize}
    \item TIG - Tungsten gas - electrode is tungsten. You do not need a metal filter. Need a gas tank to protect the weld - most often applied to stainless steels and light metals
    \item Flux-cored Arc Welding - similar to MIG. Uses a wire to serve as an electrode and a metal filler fed through the wand. Wire has a flux that creates the gas shield. Tends to have slag left so usually needs a clean-up
    \item Stick (Shielded Metal Arc Welding). Replaceable electrode stick t hat forms the filler metal. Arc is created at the end of the electrode. Stick is coated in flux that protects the metal from oxidation
    \item MIG welding (metal inert gas). Filler metal is consumable wire that acts as an electrode
    \item Laser beam welding - used on a few metals with laser providing the heat
    \item Plasma Arc Welding - uses a smaller arc with a high pressurised gas that is ionised and electrically conductive
\end{itemize}
Small amount of molten metal are introduced in the gap between two components to solidify the body. Major regions are:
\begin{enumerate}
    \item Fusion where the parts of the metal have melted and combined with filler material
    \item Heat affected zone - region next to steel that have undergone microstructural changes
\end{enumerate}
\subsection{Friction welding}
\url{https://www.youtube.com/watch?v=RTEP9QdTn5k}
\begin{figure}[H]
    \centering
    \includegraphics[width = 0.8\textwidth]{img/figure44.png}
    \caption{Friction welding.}
\end{figure}
\subsection{Radiation heating}
\begin{figure}[H]
    \centering
    \includegraphics[width = 0.8\textwidth]{img/figure42.png}
    \caption{Radiation heating.}
\end{figure}
\subsection{Laser heating}
\begin{figure}[H]
    \centering
    \includegraphics[width = 0.8\textwidth]{img/figure45.png}
    \caption{Laser heating.}
\end{figure}
\begin{figure}[H]
    \centering
    \includegraphics[width = 0.8\textwidth]{img/figure46.png}
    \caption{Close-up of laser heating.}
\end{figure}
\begin{figure}[H]
    \centering
    \includegraphics[width = 0.8\textwidth]{img/figure47.png}
    \caption{Close-up of laser heating.}
\end{figure}
The heat spreads out through diffusion from a moving source. The temperature distribution can be analysed using simple mathematical models of a moving source and this is discussed in Worksheet 15.
\section{Structural changes in the matter}
\url{https://www.youtube.com/watch?v=uG35D_euM-0&authuser=0}
\subsection{Microstructural changes}
Metals are comprised of a symmetrical structure of atoms known as an allotrope. Heating the metal will displace atoms from their position and the displaced atoms form a new structure. This process is known as allotropic phase transformation. Allotropic phase transformation alters the hardness, strength and ductility of the metal. The most important allotropic phase transformation is undergone by iron. When iron is heated past \SI{912}{\degree C} it is able to absorb more carbon which is essential for the manufacture of stainless steel.
\begin{figure}[H]
    \centering
    \includegraphics[width = 0.8\textwidth]{img/figure43.png}
    \caption{Effect on microstructure from cold rolling and then annealing.}
\end{figure}
\subsubsection{Heat treatment}
\textbf{Annealing} is used to soften metals including iron, steel, copper, brass and silver. The process involves heating the metal to a specific temperature then allowing it to cool slowly at a controlled rate. Annealing alters the physical and chemical properties of the metal to increase ductility and reduce hardness. This facilitates shaping, stamping or forming processes, and allows the metal to be cut more easily. Annealing also enhances electrical conductivity.

\textbf{Normalising} is applied to alloys to provide uniformity in grain size and composition. The metal is heated to a predefined temperature then cooled by air. The resulting metal is free of undesirable impurities and exhibits greater strength and hardness. Normalising is often used to produce a harder and stronger steel, albeit one that is less ductile than that produced by annealing. Typically, the normalising process is performed on materials that will be subjected to machining, because the process has improved this attribute.

\textbf{Hardening} is applied to steel and other alloys to improve their mechanical properties. During hardening, the metal is heated at a high temperature and this temperature is maintained until a proportion of carbon has been dissolved. Next the metal will is quenched, which involves rapidly cooling it in oil or water. Hardening will produce an alloy which has high strength and and wear resistance. However, hardening will also increase brittleness and is not suitable for engineering applications. When there is a need to have the surface of the component hard enough to resist wear and erosion, while maintaining ductility and toughness to withstand impact and shock loading - surface hardening would be used.

\textbf{Tempering} is applied to steel where ductility is desired. Untempered steel is very hard but too brittle for most practical applications. Tempering is a low temperature heat treatment process normally performed after hardening (neutral hardening, double hardening, atmospheric carburising, carbonitriding, or induction hardening) in order to reach a desired hardness/toughness ratio. the process involves heating steel to a lower temperature to reduce some of the excess hardness. The metal is then allowed to cool in still air which results in a tougher and less brittle steel.
\subsection{Macroscopic changes}
Large temperature variations lead to inelastic and non-recoverable deformations. For plastic deformation require about 0.2\% residual strain. Since the strain generated a temperature difference of $\Delta T$ scales as $\epsilon \sim \Delta T \alpha$. With a typical value of a $\alpha \sim \SI{1e-5}{\per\kelvin}$, we only need $\Delta T \sim \SI{200}{\kelvin}$ to generate this strain. Large temperatures leads to melting, rearrangement of bonds and this is what is used in casting and welding. Ductile and malleable materials can absorb changes while brittle materials fracture.
\section{Soldering, brazing, welding}
\begin{itemize}
    \item Soldering is a low-temperature process (60-\SI{400}{\degree C}) that uses a low-melting metal (a base of tin combined with lead, silver, antimony, bismuth, indium) to join similar or dissimilar metals; it is mainly used in electronic boards
    \item Brazing is a mid-temperature process (450-\SI{1200}{\degree C}) that uses a high-melting metal (a base of silver combined with nickel, copper, zinc) to join similar or dissimilar metals; it is mainly used in copper piping and jewellery
    \item Welding is a high temperature process (800-\SI{2000}{\degree C}) that uses a powerful heat source to locally melt and join similar metals; it is mainly used in iron and steel work
\end{itemize}
\subsubsection{Influence of localised heating}
Close to the weld there is a heat affected region where the microstructure is affected by the heat. The metal in this area is generally weaker than the base material and the fusion zone. This is where the residual stresses are found.
\begin{figure}[H]
    \centering
    \includegraphics[width = 0.8\textwidth]{img/figure48.png}
    \caption{Influence of localised heating on a weld.}
\end{figure}
\subsubsection{Heat affected zone}
This is the ring that surrounds the weld which affects the alloy. If thermal diffusivity is high, cooling rate is high so the HAZ is smaller. If thermal diffusivity is low, cooling rate is low so the HAZ is bigger. Other measures are used, such as rate of heat input for weld where:
\begin{equation}
    Q = \frac{60VI}{1000U}\times \textrm{efficiency}
\end{equation}
\begin{itemize}
    \item $Q$ is the heat input (\si{\kilo\joule\per\milli\meter})
    \item $VI$ is the electrical power
    \item $U$ is the speed of the weld
\end{itemize}
Usually we need $Q \approx 10-\SI{25}{\kilo\joule\per\milli\meter}$.
\begin{table}[H]
    \centering
    \begin{tabular}{@{}ll@{}}
        \toprule
        \textbf{Weld} & \textbf{Efficiency} \\
        \midrule
        PAW           & 0.46                \\
        GTAW          & 0.65                \\
        Gas metal arc & 0.83                \\
        \bottomrule
    \end{tabular}
    \caption{Welding efficiencies.}
\end{table}
\subsubsection{Thermoplastic shrinkage}
When a plate is heated, there is an elastic convex deformation that fades as it is cooled and a plastic concave deformation. This is exploited in ship manufacturing techniques to create curved sheets.
\begin{figure}[H]
    \centering
    \includegraphics[width = 0.8\textwidth]{img/figure49.png}
    \caption{Thermoplastic shrinkage.}
\end{figure}
\begin{figure}[H]
    \centering
    \includegraphics[width = 0.8\textwidth]{img/figure50.png}
    \caption{Weld shrinkage.}
\end{figure}
Weld shrinkage - generated by localised stresses caused by heating and distortion of the heated material. Usually leads to transverse and longitudinal shrinkage.
\subsubsection{Process of plate heating}
The process is known as heat line technique or line heating method of plate bending; it is applied mainly to mild-steel plates, and was started in the 1970s in shipbuilding. It consists of the following steps.
\begin{enumerate}
    \item Initial heating. It forces the heated mass to expand against the rest of the material, creating great stresses and a very small convex elastic deformation due to the temperature gradient
    \item High heating. Up to \SI{1200}{\kelvin} (but usually limited to $<$\SI{995}{\kelvin} to avoid the mild-steel phase transition). It lowers the strength of the heated mass so much, that plastic-yield takes place, that the side material forces the heated mass to bulge in the hottest region
    \item After cooling. Forced cooling (usually by water) increases the temperature gradient that forces the heated mass to recover its original strength but not its original shape, because the plastic deformation is not reversible, causing a shrinkage that pulls in from the rest of the material (i.e. in the whole it is not a thermal push but a thermal pull), causing a concave bending (and perhaps some cracks), and minor in-plane deformations due to the point-wise application (instead of the whole line at a time).
\end{enumerate}
\begin{figure}[H]
    \centering
    \includegraphics[width = 0.8\textwidth]{img/figure51.png}
    \caption{Plate heating.}
\end{figure}
\section{Rapid changes in temperature}
When a material is cooled or heated very rapidly over a small region, thermal stresses can be generated. This occurs due to uneven heating/cooling. Assume top thin layer is rapidly cooled from $T_1$ to $T_2$, the stress generated is:
\begin{equation}
    \sigma = - E \alpha \left(T_1 - T_2\right)
\end{equation}
The rate of heating is characterised by the Biot number,
\begin{equation}
    Bi = \frac{hH}{k}
\end{equation}
In contrast to small temperature changes, a hot shock is not the same as a cold shock since the highest stress occurs in different locations. Rapid cooling and heating is used to generate a stress in safety glass that causes it to fracture catastrophically (and safely).
\begin{figure}[H]
    \centering
    \includegraphics[width = 0.8\textwidth]{img/figure52.png}
    \caption{Rapid changes in temperature characterised.}
\end{figure}
\subsection{Thermal shock}
Thermal shock occurs when a thermal gradient causes different parts of an object to expand by different amounts. This differential expansion can be understood in terms of stress or of strain, equivalently. At some point, this stress can exceed the strength of the material, causing a crack to form. If nothing stops this crack from propagating through the material, it will cause the object's structure to fail (read Lu \& Fleck 1998).
\begin{figure}[H]
    \centering
    \includegraphics[width = 0.8\textwidth]{img/figure53.png}
    \caption{Thermal shock.}
\end{figure}
\subsection{Stress criterion}
The usual measure is the thermal shock resistance parameter:
\begin{equation}
    TSR = \frac{\sigma_fk}{E\alpha}
\end{equation}
where:
\begin{itemize}
    \item $\sigma_f$ is fracture strength of a material
    \item $\alpha$ is the expansivity
    \item $k$ is the thermal conductivity of material
    \item $E$ is Young's modulus of elasticity
\end{itemize}
When $TSR>$ critical value, there is a failure. The stress distribution in a thin sheet is:
\begin{equation}
    \sigma_{xx} = -E\alpha\left(T-T_i\right) + \frac{E\alpha}{2H}\int_{-H}^H\left(T-T_i\right)\textrm{d}z
\end{equation}
Kingery (1995) derived two relations.
\begin{equation}
    R = \frac{\sigma_f \left(1- v\right)}{E\alpha}
\end{equation}
Useful for instantaneous change in surface temperature (infinite $h$).
\begin{equation}
    R' = \frac{\sigma_f \left(1-v\right)k}{E\alpha}
\end{equation}
Relatively low Biot modulus. For constant heating:
\begin{equation}
    R' = \frac{\sigma_f\left(1-v\right)a}{E\alpha}
\end{equation}
Where $a$ is the thermal diffusivity.
\subsection{Strain criterion}
One way to look at for brittle materials is to define:
\begin{equation}
    \theta = \frac{\epsilon_c}{\alpha}
\end{equation}
where $\epsilon_c$ is the critical strain for failure. This is appropriate for a uniform temperature and high conductivity. If this is not available, then we would use:
\begin{equation}
    \epsilon_c = \frac{\sigma_u}{E}
\end{equation}
\subsubsection{Reducing failure from shocking}
Failure due to thermal shock can be prevented by reducing the thermal gradient seen by the object, by changing its temperature more slowly or increasing the material's thermal conductivity ($k$).
\begin{itemize}
    \item Reducing the material's coefficient of thermal expansion ($\alpha$)
    \item Increasing its strength ($\sigma_f$)
    \item Introducing built-in compressive stress, as for example in tempered glass
    \item Decreasing its Young's modulus ($E$)
    \item Increasing its toughness, by crack tip blunting (i.e., plasticity or phase transformation) or crack deflection (smaller cracks)
\end{itemize}
\subsection{Fraction criterion}
Hasselman (1969) developed a unified theory of thermal shock for brittle ceramics. It was `unified' in the sense that the conditions for the fracture initiation were combined with a consideration of crack propagation. For an infinitely fast quench, from a higher temperature, a thermal stress, $\sigma_{TS}$, is generated in the ceramic and is given by:
\begin{equation}
    \sigma_{TS} = \frac{\alpha E \Delta T}{\left(1-2v\right)}
\end{equation}
where $E$ is the Young's modulus, $\Delta T$ is the difference in temperature between the initial temperature of the ceramic and the quenching medium, $\alpha$ is the coefficient of linear thermal expansion and $v$ is the Poisson's ratio.

Mechanical model is based on an energy argument. For Griffith microcracks ($N$ per unit volume), spread uniformly in the material,
\begin{equation}
    W_t = \frac{3\left(\alpha\Delta T\right)^2E}{2\left(1-2v\right)}\left(1 + \frac{16\left(1-v^2\right)Nl^3}{9\left(1-2v\right)}\right)^{-1} + 2\pi Nl^2 G
\end{equation}
where $l$ is the crack length and $2\pi Nl^2 G$ is the energy associated with circular crack of radius $l$. According to Griffith, the limit of the cracks that are unstable are those where:
\begin{equation}
    \frac{\dif W_t}{\dif l} = 0
\end{equation}
Energy of a crack is:
\begin{equation}
    \int \sigma \dif \epsilon = \frac{3\sigma^2}{2\epsilon} \qquad \epsilon = \alpha \Delta T
\end{equation}
\begin{figure}[H]
    \centering
    \includegraphics[width = 0.6\textwidth]{img/figure54.png}
    \caption{Crack lengths against temperature difference.}
\end{figure}
For short cracks:
\begin{equation}
    \Delta T_c = \sqrt{\frac{\pi G \left(1 - 2v\right)^2}{2E_0\alpha^2\left(1-v^2\right)l}}
\end{equation}
For long cracks:
\begin{equation}
    \Delta T_c = \sqrt{\frac{128\pi G \left(1-v^2\right)N^2l^5}{81\alpha^2 E_0}}
\end{equation}
The results can be expressed in terms of the critical fracture stress:
\begin{equation}
    S_t = \sqrt{\frac{GE_0}{2l_0\left(1-v^2\right)}}
\end{equation}
The dimensionless group is:
\begin{equation}
    \frac{S_t}{\sqrt{GE}}
\end{equation}
The temperature difference inside the material is:
\begin{equation}
    \Delta T_i \sim \Delta T \frac{k}{ah}
\end{equation}
Then:
\begin{equation}
    \frac{\alpha \Delta T E k \sqrt{l}}{ha\sqrt{\pi G E}} \sim 1
\end{equation}
\begin{figure}[H]
    \centering
    \includegraphics[width = \textwidth]{img/figure55.png}
    \caption{Crack lengths , temperature difference, strength and fracture stress.}
\end{figure}
\subsection{Material selection}
Among the best thermomechanical materials, there are alumina, zirconia, tungsten alloys, silicon nitride, silicon carbide, boron carbide, and some stainless steels. Reinforced carbon-carbon is extremely resistant to thermal shock, due to graphite's extremely high thermal conductivity and low expansion coefficient, the high strength of carbon fibre and a reasonable ability to deflect cracks within the structure.
\section{Conclusions}
The influence of large changes in temperature can be exploited to change the properties of materials and as part of a manufacturing process. In some instances, sudden changes in temperature, can lead to failure either through a yield, strain or cracking. There are approximate models of these processes.