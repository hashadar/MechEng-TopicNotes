\chapter{Road Mapping}
\section{Overview}
\subsection{Context}
Objective of year 4 is to bring together all your knowledge from your undergraduate courses and see how it can be applied to real practical problems. Graduate level courses put the emphasis on the student to find and read around the material.
\begin{itemize}
    \item First, we will explore what we mean by the words ``extreme pressure'', ``extreme chemistry'' and ``extreme temperature''
          \begin{itemize}
              \item \dots by looking at a number of examples
          \end{itemize}
    \item Structure of the courses - the role of the professor and your role
    \item Formalisation of these definitions and classifications
    \item How does industry deal with these problems
\end{itemize}
\subsection{Course assessment}
\textbf{Assessments}
\begin{enumerate}
    \item Coursework (30\%) - January
    \item Group Project (40\%) - end of March
    \item Laboratory class (30\%) - February
\end{enumerate}
1, 2, 3 will cover the three themes of pressure, temperature and chemistry.
\begin{table}[htbp]
    \centering
    \begin{tabular}{@{}llll@{}}
        \toprule
                      & \textbf{2018-2019}    & \textbf{2019-2020}            & \textbf{2020-2021}            \\
        \midrule
        Exam          & Extreme Pressure      & Extreme Temperature           & CW Extreme Temperature        \\
        Group Project & Extreme Temperature   & Extreme Pressure              & Extreme Pressure              \\
        Laboratory    & Corrosion (chemistry) & Extreme chemistry (corrosion) & Extreme chemistry (corrosion) \\
        \bottomrule
    \end{tabular}
    \caption{Table to show themes of assessments in previous years.}
\end{table}
\subsection{Group project - extreme pressure 2019 / 2020}
You must choose which problem you are interested in and we will organise you into groups. We will have a fortnightly session to discuss through the term and give you help. We will be running training sessions so that everyone has a chance to learn a bit about CFD. Need to raise the game from last year - tended to have a late start.
\subsection{Course structure}
4 teaching blocks:
\subsubsection{Block 1}
Lectures 1-4:
\begin{itemize}
    \item Setting the scene, framework and terminology
    \item Investigative tools used to solve hard problems
    \item Key concepts for molecular scales (chemistry and materials)
    \item Key concepts for continuum scales (fluid mechanics, solid mechanics and structures)
\end{itemize}
\subsubsection{Block 2}
Lectures 5-8: Extreme Pressure.
\begin{itemize}
    \item Static cases
          \begin{enumerate}
              \item High pressure: bottom of the ocean, pressure vessels
              \item Low pressure: space, cavitation
          \end{enumerate}
    \item Unsteady cases
          \begin{enumerate}
              \item High pressure: gas flows, AIV, explosions (nuclear), liquid flows, VIV, TIV
              \item Low pressure: cavitation
          \end{enumerate}
\end{itemize}
\subsubsection{Block 3}
Lecture 9-12: Extreme Chemistry.
\begin{itemize}
    \item Slow chemistry
          \begin{enumerate}
              \item Corrosion
          \end{enumerate}
    \item Fast chemistry
          \begin{enumerate}
              \item Fuel cells
              \item Combustion
          \end{enumerate}
    \item Application section: manufacturing, design and testing
\end{itemize}
\subsubsection{Block 4}
Lecture 13-16: Extreme Temperature.
\begin{itemize}
    \item Static cases
          \begin{enumerate}
              \item High temperature: fire, manufacturing, coatings
              \item Low temperature: arctic, LNG transport
          \end{enumerate}
    \item Unsteady cases
          \begin{enumerate}
              \item Thermal shocks
          \end{enumerate}
\end{itemize}
\section{Engineering in extreme environments}
What do we mean by these words?
\begin{quoting}
    The phrase "engineering in extreme environments" relates to the study, design, and implementation of engineering solutions that can withstand and function optimally under extremely harsh or atypical conditions. These conditions are characterized by extreme pressure, extreme temperature, and extreme chemical environments, all of which present unique challenges that can significantly affect the performance and durability of materials and systems.

    Specifically:

    Extreme Pressure: Engineering in high-pressure environments often involves deep-sea, subterranean or aerospace applications, where the equipment is subjected to extraordinary pressures. This requires an in-depth understanding of material properties and structural integrity under such conditions, including factors like compression, shear stress, and deformation.

    Extreme Temperature: This refers to the ability to design and implement engineering solutions in environments with very high or very low temperatures. This might include deep-sea vents, cryogenic systems, combustion engines, or spacecraft re-entry systems. The challenges here primarily relate to thermal stress, material expansion/contraction, and maintaining operational efficiency under these temperature ranges.

    Extreme Chemistry: Engineering in extreme chemical environments requires developing materials and systems that can resist highly corrosive or reactive environments, which can include certain industrial processes, chemical plants, or hostile extraterrestrial environments. An understanding of material science, including the resistance of different materials to corrosion, oxidation, and other chemical reactions, is crucial.

    This aspect of mechanical engineering seeks to unite the foundational knowledge of mechanical engineering — such as fluid mechanics, thermodynamics, materials science, and structural analysis — with a deeper understanding of how these principles apply under extreme conditions. It typically involves rigorous testing and modelling, the use of advanced materials and coatings, and innovative design principles to ensure safety, efficiency, and longevity of engineering systems under extreme environments.

    (ChatGPT-4 2023-07-03)
\end{quoting}
\subsection{Engineering}
Engineering - a branch of science and technology concerned with the design, building, and use of engines, machines, and structures.

Focussed on the discrete elements or components that sit in an environment. We will refer to a structure throughout the course, through what we really mean is anything (engine, blade, casing, tool, system, person).
\subsection{Classification of environment}
Environment - this is just the surroundings in which an objects sits or is exposed to. There are two types of environments:
\begin{enumerate}
    \item Natural
    \item Man-made
\end{enumerate}
There has to be a physical way to communicate between the environment and the engineering structure which is usually through a surface.
\subsection{Classification of interactions}
\begin{itemize}
    \item Internal or closed problems e.g. pipe flow, IC engine
    \item Open problems e.g. fire outside, atmospheric
          \begin{itemize}
              \item Open problems represent challenges in terms of boundary conditions
          \end{itemize}
\end{itemize}
\subsubsection{Types of interactions}
\begin{itemize}
    \item Interactions are with the interfaces. What processes act on an interface?
    \item Solid surface - affected by:
          \begin{itemize}
              \item normal / tangential forces / pressure / viscous forces (high / low pressure)
          \end{itemize}
    \item Thermal effects (heating / cooling)
    \item Chemical effects (burning, corrosion)
    \item Structural changes due to forces / loading
    \item Fatigue
\end{itemize}
\subsubsection{Interfacial processes}
\begin{itemize}
    \item Temperature
    \item Stresses - solid and fluid forces
    \item Chemistry
\end{itemize}
\subsection{Classification of extremes}
Extreme has quite a few interpretations. Extreme may be something that is an outlier. The definition that we will work around is based on two types:
\begin{enumerate}
    \item based on magnitude (of a scalar or vector)
    \item how it is spread and varies spatially or temporally
\end{enumerate}
We try to tackle these broad questions using regime diagrams (Figures \ref{regimeDiagram1}, \ref{regimeDiagram2}, \ref{regimeDiagram3}).
\begin{figure}[htbp]
    \centering
    \includegraphics[width = 0.8\textwidth]{./img/figure78.png}
    \caption{Regime diagram to show different pressure environments.}
    \label{regimeDiagram1}
\end{figure}
\begin{figure}[htbp]
    \centering
    \includegraphics[width = 0.8\textwidth]{./img/figure79.png}
    \caption{Regime diagram to show different temperature environments.}
    \label{regimeDiagram2}
\end{figure}
\begin{figure}[htbp]
    \centering
    \includegraphics[width = 0.8\textwidth]{./img/figure79.png}
    \caption{Regime diagram to show different chemical environments.}
    \label{regimeDiagram3}
\end{figure}
\section{Extreme natural environments}
\begin{table}[htbp]
    \centering
    \begin{tabular}{@{}ll@{}}
        \toprule
        \textbf{Under extreme} & \textbf{Examples}            \\
        \midrule
        Pressure               & Tsunami, earthquakes, storms \\
        Temperature            & Fires, arctic                \\
        Chemistry              & Corrosion, biofouling        \\
        \bottomrule
    \end{tabular}
    \caption{Table to show example naturally extreme environments.}
\end{table}
\subsubsection{Challenges in extreme natural environments}
Main challenges are:
\begin{enumerate}
    \item Living in these external environments
    \item Working in these external environments
\end{enumerate}
\subsection{Extreme pressure}
\subsubsection{Example: deep water (high pressure)}
\begin{quoting}
    Deep underwater environments, such as those over 2000 meters below sea level, experience extreme hydrostatic pressure. This pressure can pose severe challenges to structures, equipment, and human life. For instance, habitats in deep oceans and lakes must be designed to withstand these pressures. A case in point is the Deepwater Horizon oil drilling rig, which suffered a catastrophic failure in 2010. The disaster was partly attributed to a mismanagement of high-pressure conditions approximately 1000 meters below sea level, highlighting the critical role that pressure management plays in these environments.

    (ChatGPT-4 2023-07-03)
\end{quoting}
\subsubsection{Example: hurricanes (high pressure and gustiness)}
\begin{quoting}
    Hurricanes present a convergence of extreme conditions. The high wind pressure and gustiness can lead to severe infrastructural damage. These natural disasters are formed from a convergence of tropical thunderstorms, leading to an area of low pressure with high winds blowing into the area. Over 2000 deaths have occurred due to hurricanes in the US since 2005. The combination of high pressure, high winds, and water hazards like storm surges and flooding make them a multifaceted challenge.

    (ChatGPT-4 2023-07-03)
\end{quoting}
\subsubsection{Example: earthquakes (high solid mechanical pressure)}
\begin{quoting}
    Earthquakes generate extreme solid mechanical pressures resulting from seismic activity. The vibration of the ground due to surface waves can lead to the failure and collapse of structures, making it a significant challenge for civil and structural engineers to design buildings that can withstand these forces. On average, earthquakes result in around 10,000 fatalities per year, indicating the severity of their impact and the importance of engineering designs that consider seismic activity.

    (ChatGPT-4 2023-07-03)
\end{quoting}
\subsubsection{Example: tsunami (high fluid pressure)}
\begin{quoting}
    Tsunamis represent a high-impact hazard that involves enormous fluid pressures. Generated by underwater seismic activity, landslides, or volcanic eruptions, tsunamis can travel across ocean basins and cause massive destruction when they make landfall. A devastating example of this is the 2004 Indian Ocean tsunami, which resulted in the deaths of over 230,000 people. Designing for tsunami resistance, especially in coastal areas, involves understanding and accounting for these extreme fluid pressures in order to protect structures and lives.

    (ChatGPT-4 2023-07-03)
\end{quoting}
\subsection{Extreme temperature}
\subsubsection{Example: fire on land}
\begin{quoting}
    Areas that consistently experience temperatures in excess of \SI{40}{\degree C}, often in combination with dry conditions, are prone to land fires. These high-temperature environments can cause severe challenges for both human and animal habitation, and can also lead to significant environmental damage. Land fires often result in casualties and cause extensive damage to wildlife and forests. They pose unique challenges for engineers and firefighters, who must design and implement fire prevention, containment, and extinguishing strategies. These may include fire-resistant building materials and designs, effective emergency evacuation plans, and the use of technology to predict and manage wildfires.

    (ChatGPT-4 2023-07-04)
\end{quoting}
\subsubsection{Example: arctic and antarctic}
\begin{quoting}
    These regions present extreme cold conditions that provide a unique set of challenges for both natural and human-engineered systems. Life forms such as organisms living in the ice, zooplankton and phytoplankton, fish and marine mammals, birds, land animals, and human societies have to adapt to these frigid environments. In the context of engineering, these conditions necessitate the design of structures and equipment that can withstand the low temperatures, high winds, and icy conditions prevalent in these regions. This might involve the use of special materials that retain their structural integrity at low temperatures, insulation technologies to protect against the cold, and heating systems that can operate efficiently in these conditions. Infrastructure must be designed to accommodate the challenges of ice movement and permafrost, while ensuring the safety and comfort of human inhabitants.

    (ChatGPT-4 2023-07-04)
\end{quoting}
\subsection{Extreme chemistry}
\subsubsection{Example: corrosion}
\begin{quoting}
    Corrosion is a common problem in environments where metal components are exposed to corrosive agents such as salt. Salt, particularly in the form of sea spray or road de-icing materials, can accelerate corrosion processes, leading to the deterioration and failure of metal structures and equipment. To combat this, engineers employ various protective strategies. These may include the use of special corrosion-resistant materials, protective coatings, or the application of an electro-motive force (EMF) to protect against the corrosive effects of salt.

    (ChatGPT-4 2023-07-04)
\end{quoting}
\subsubsection{Example: extreme pH (alkaline and acidic)}
\begin{quoting}
    Certain environments can present extreme pH conditions, either persistently or intermittently, that pose challenges for both natural life and engineered systems:
    \begin{itemize}
        \item Alkaline Environments: Habitats with a pH above 9, such as some soils, waters, and industrial processes, can be corrosive to certain materials and can disrupt biological processes. One interesting example is Mono Lake in California's Eastern Sierras, characterized by its highly alkaline waters. A unique feature of this environment is a soft, gelatinous microbial mat that forms on the lake's surface, demonstrating the ability of life to adapt to even these extreme conditions. For engineers, dealing with such environments might involve the use of materials that resist alkaline corrosion, or the design of systems that neutralize or manage the alkaline conditions.
        \item Acidic Environments: Habitats with a pH below 5, such as acid mine drainage or some peat bogs, present their own set of challenges. Acidic conditions can be highly corrosive, posing risks to infrastructure and equipment, and can also disrupt biological activity. Engineering solutions for these environments may include the use of acid-resistant materials, coatings, or pH buffering systems to manage the acidic conditions.
    \end{itemize}

    (ChatGPT-4 2023-07-04)
\end{quoting}
\section{Industrial accidents}
Industrial accidents are usually a combination of failure and stupidity
\subsubsection{Example disaster: Enschede}
\begin{quoting}
    In this tragedy that occurred in Enschede, Netherlands on May 13, 2000, a fire ignited approximately \SI{900}{\kilo\gram} of fireworks stored in a central building depot. This led to a chain reaction explosion of 177 tons of fireworks stored in nearby containers. The disaster, resulting in the deaths of 23 people, injuring 940, and causing the destruction of over 1500 buildings, underscored the immense potential hazards of improperly stored explosives. Notably, the depot was inspected and deemed safe by authorities just a week before the disaster, highlighting a critical failure in regulatory oversight and safety protocol enforcement.

    (ChatGPT-4 2023-07-04)
\end{quoting}
\subsubsection{Example disaster: Exxon Valdez}
\begin{quoting}
    This environmental catastrophe occurred on March 24, 1989, when the Exxon Valdez oil tanker hit a reef in Prince William Sound, Alaska, spilling hundreds of thousands of barrels of crude oil into the surrounding environment. The immediate impact included the death of up to 250,000 seabirds, at least 2800 sea otters, and many other marine creatures. Long-term effects were also profound, with significant quantities of oil remaining in the environment decades later and ongoing harm to local ecosystems. This disaster demonstrated the dire consequences of inadequate navigation safety protocols and environmental protection measures in oil transport operations.

    (ChatGPT-4 2023-07-04)
\end{quoting}
\subsubsection{Example disaster: Chernobyl}
\begin{quoting}
    On April 26, 1986, a catastrophic failure occurred at reactor number four of the Chernobyl Nuclear Power Plant in Ukraine. This failure, resulting from a series of steam explosions during an attempted emergency shutdown, released substantial quantities of radioactive material into the environment, resulting in immediate and long-term fatalities and significant environmental damage. The disaster highlighted the catastrophic risks associated with nuclear power generation, particularly in the context of inadequate safety measures and procedural controls.

    (ChatGPT-4 2023-07-04)
\end{quoting}
\subsubsection{Example disaster: Savar building collapse}
\begin{quoting}
    This structural failure, which occurred in Savar, Bangladesh on April 24, 2013, is the deadliest building collapse in history. Despite visible cracks and official warnings, garment factories in the building continued operations, leading to the death of approximately 1129 people and injuries to over 2500. This disaster underscored the severe potential consequences of ignoring structural safety warnings, as well as the need for more effective enforcement of building safety regulations.

    (ChatGPT-4 2023-07-04)
\end{quoting}
\subsubsection{Example disaster: The Banqiao Dam Collapse, China}
\begin{quoting}
    This dam collapse in China in August 1975 led to the deaths of approximately 171,000 people when it released about 15 billion cubic meters of water. The collapse, due to both natural (exceptional rainfall) and human (poor construction and engineering) factors, demonstrated the potential scale of disasters related to dam failure and highlighted the critical importance of robust design and construction practices for such significant infrastructure.

    (ChatGPT-4 2023-07-04)
\end{quoting}
\subsubsection{Example disaster: Bhopal Gas Tragedy, India}
\begin{quoting}
    This industrial disaster occurred on the night of December 2-3, 1984, when over 500,000 people were exposed to methyl isocyanate and other chemicals leaking from a pesticide plant in Bhopal, India. The event resulted in thousands of deaths and serious injuries, and was a result of negligent safety practices by the plant operator and inadequate enforcement by government authorities. The tragedy serves as a stark reminder of the potentially disastrous consequences of lax industrial safety standards and oversight.

    (ChatGPT-4 2023-07-04)
\end{quoting}
\section{Codes of practice}
\begin{itemize}
    \item Most disasters occur because people fail to follow codes of practice
    \item Everything that is constructed and designed must be done within the confines of best practice
    \item Insurance is not given unless best practice is followed
    \item Various bodies are responsible for regulating engineering design in normal and extreme environments
\end{itemize}
\subsection{Professional bodies}
\begin{table}[H]
    \centering
    \begin{tabular}{@{}lll@{}}
        \toprule
        \textbf{Society} &                                           & \textbf{Remit}                       \\
        \midrule
        ASTM             & American Society for Testing Materials    & Materials                            \\
        ASME             & American Society for Mechanical Engineers & Mechanical Engineering               \\
        ASCE             & American Society for Civil Engineers      & Civil Engineering                    \\
        ISO              & International Organisation for Standards  & Everything                           \\
        BSEN             & British Standards Institute (BSI)         & National Agency - follows EU and ISO \\
        Lloyds Registry  & Classification Society                    & Ships                                \\
        MARPOL           & Offshore Pollution IMO                    & Ships                                \\
        DNV-GL           & Classification Society                    & Ships                                \\
        \bottomrule
    \end{tabular}
    \caption{List of some professional bodies and their remits.}
\end{table}
\subsection{Example of codes of practice}
\subsubsection{ASTM}
American Society for Testing and Materials is an international standards organisation that develops and publishes voluntary consensus technical standards for a wide range of material, products, systems and services.
\begin{quoting}
    This specification covers high strength, cold drawn or cold rolled steel wire with rectangular cross-section and round mill edge. Intended for use in pre-stressed vessel and press frame windings. The steel shall be manufactured by basic oxygen, electric furnace, or vacuum induction processes and produced as ingot cast or continuous cast. Materials shall be tested for contents of carbon, manganese phosphorus, sulphur and silicon. A tension test shall also be performed to evaluate mechanical properties conformance. Guidelines for workmanship, finish and appearance are stated, as well as the inspection, certification of reports, packaging, marking, and loading for shipment.

    (ASTM 9505-04 (2017) Standard Specification for Steel Wire, Pressure Vessel Winding Active Standard (latest version))
\end{quoting}
\subsubsection{ASME}
There are a number of student resources about codes of practice. It is worth looking at these: \url{https://www.asme.org/codes-standards/about-standards}.
\begin{quoting}
    Code and standards are an essential part of design and repair engineering requirements. Engineers that understand code and regulation requirements are a must in my industry and will increasingly become more widespread in other industries, as regulatory issues change and become more stringent. All industries are becoming more aware of the benefits of adherence to Code and industry standards in ensuring reliability of systems, components and structures.

    (ASME Vision 2030 Survey Feedback)
\end{quoting}
\subsubsection{ASCE}
Our interest is on ASCE/SEI 7-16, Minimum Design Loads and Associated Criteria for Buildings and Other Structures. See: \url{https://asce7tsunami.online/}
\begin{quoting}
    ASCE/SEI 7-16, or "Minimum Design Loads and Associated Criteria for Buildings and Other Structures," is a particularly relevant standard developed by ASCE. This standard provides the complete set of load provisions, which includes provisions for dead, live, soil, flood, wind, snow, rain, ice, and earthquake loads, as well as their combinations that are suitable for inclusion in building codes and other documents. ASCE/SEI 7-16 applies to the design and construction of general structural systems and elements for buildings and other structures.

    One particular update in the 2016 revision of ASCE/SEI 7 was the inclusion of tsunami loads and effects. The changes were a response to the recognition of the risk posed by tsunamis to coastal structures, especially in areas such as the Pacific Northwest. These changes involved substantial work in understanding the complex forces exerted by tsunami waves and the resulting implications for structural design.

    (ChatGPT-4 2023-07-04)
\end{quoting}
\subsection{Contributing to codes of practice}
Codes of practice come out of the research activity from universities, committee memberships of various organisations.
\subsection{Example of impact}
The Marine Group in UCL MechEng have been working on the three main areas of environmental pollution of the seas:
\begin{enumerate}
    \item ballast water discharge
    \item acidic discharge
    \item sound generation
\end{enumerate}
Guidelines related to MARPOL Annex VI and the NOX Technical Code in accordance with the Action Plan endorsed by the MEPC 64.
\begin{quoting}
    Application of calculation-based methodology for verification of the wash water discharge criteria for pH in section 10.1.2.1(ii) of the 2009 Guidelines for Exhaust Gas Cleaning Systems, Professor I. Eames \& Professor A. Greig

    (Code MEPC 64 (MARPOL Annex VI))
\end{quoting}
\subsubsection{Example chemistry: wash water discharge}
Scrubbers are used to remove particulate and acidic gases from engine exhaust, they generate a buoyant and acidic discharge.
\begin{itemize}
    \item pH = $-\log_{10}\ce{H^+}$
    \item pH of discharge = 3-4
    \item pH of sea = 7.4-8.3
\end{itemize}
Code requires pH to recover to 6.0 about \SI{4}{\meter} from the exit.
\subsubsection{Example chemistry: physical process of jet}
To offer guidance we need to be able to say whether discharge criteria are met. This requires a predictive environment that is validated against test and simple enough to be understood. Complexity comes from the turbulent part:
\begin{itemize}
    \item Jet
    \item Plume
    \item Gravity current.
\end{itemize}
\subsubsection{What defines a jet / plume?}
A \textit{jet} is a stream of fluid that is forced into a surrounding medium, usually air or water, typically from some kind of a nozzle. On the other hand, a \textit{plume} is a column of one fluid moving through another, and is formed by the diffusion of the source fluid into the surrounding medium.
\begin{itemize}
    \item \(Q\) is the volume flux, which measures the quantity of volume flowing through a given cross-sectional area per unit time. In the equation \begin{equation}Q = \pi b^2 w\end{equation} \(b\) is the radius of the jet or plume and \(w\) is the velocity of the fluid.
    \item \(M\) is the momentum flux, which measures the rate of momentum flow across a given cross-sectional area. In the equation \begin{equation}M = \pi b^2 w^2\end{equation} \(b\) is again the radius and \(w\) is the velocity.
    \item \(B\) is the buoyancy flux, which describes the flow rate of buoyancy across a given cross-sectional area. In the equation \begin{equation}B = -\pi b^2\frac{w\Delta \rho}{\rho_0}\end{equation} \(b\) is the radius, \(w\) is the velocity, \(\Delta \rho\) is the difference in density between the two fluids, and \(\rho_0\) is the reference (usually ambient) density.
\end{itemize}
The equations for the jet radius \(R\) and the dimensionless volume flux \(\bar{Q}\) both involve \(R_0\) (the initial radius) \(\alpha\) (a constant of proportionality), and \(Z\) (the vertical height). In the context of these equations, \(\alpha Z\) represents the spreading of the jet or plume with height.

For the jet:
\begin{itemize}
    \item The radius increases linearly with height \begin{equation}R = R_0 + 2\alpha Z\end{equation}
    \item The dimensionless volume flux, which is the volume flux normalized by the initial volume flux, also increases linearly with height \begin{equation}\bar{Q} = \left(1 + \frac{2\alpha Z}{b_0}\right)\end{equation}
\end{itemize}
For the plume:
\begin{itemize}
    \item The radius increases more rapidly with height \begin{equation}R = R_0 + \frac{6}{5}\alpha Z\end{equation} reflecting the additional spreading due to buoyancy forces.
    \item The dimensionless volume flux for the plume also increases more rapidly with height \begin{equation}\bar{Q} = \left(1 + \frac{6\alpha Z}{5b_0}\right)^{\frac{5}{3}}\end{equation} again due to the effects of buoyancy.
\end{itemize}
\subsubsection{Example: chemistry of strong acids and bases}
Conservation of charge:
\begin{equation}
    [\ce{H^+}] + [\ce{M^+}] = [\ce{OH^-}] + [A^-]
\end{equation}
Conservation of mass:
\begin{align}
    C_b^0D & = \left([\ce{MOH}] + [\ce{M^+}]\right)\left(1 +D \right) \\
    C^0_a  & = [\ce{A^-}]\left(1+D\right)
\end{align}
Note $D$ is the dilution of the acid. Dissociation of water and base ($K_b = 1$, strong base):
\begin{align}
    K_w & = [\ce{OH^-}][\ce{H^+}]                     \\
    K_b & = \dfrac{[\ce{M^+}][\ce{OH^-}]}{[\ce{MOH}]}
\end{align}
Note $K_w = \SI{1e-14}{\mol^2 / dm^{-6}}$ is temperature dependent giving a point of neutralisation which varies from 7.47 at \SI{0}{\degree C} to 6.92 at \SI{30}{\degree C}. This leads to:
\begin{equation}
    \frac{1}{D} = \dfrac{\left(\dfrac{C^0_b K_b}{\frac{K_w}{[\ce{H^+}]+K_b}} + [\ce{H^+}]-\dfrac{K_w}{[\ce{H^+}]}\right)}{C^0_a - [\ce{H^+}]+\dfrac{K_w}{[\ce{H^+}]}}
\end{equation}
Explanation of above section:
\begin{quoting}
    Strong acids and bases fully dissociate in solution, meaning they split into their respective ions completely. Here are the given equations explained:

    The first equation is a conservation of charge equation. The sum of the charges of all species on both sides of the equation should be equal. This equation states that the sum of the hydrogen ion concentration ([\ce{H^+}]) and metal ion concentration ([\ce{M^+}]) equals the sum of the hydroxide ion concentration ([\ce{OH^-}]) and the anion concentration ([\ce{A^-}]).

    The next two equations deal with conservation of mass. In the first, $C_b^0D$ represents the initial molar concentration of the base multiplied by the dilution factor. This equals the sum of the concentrations of metal hydroxide ([\ce{MOH}]) and the metal ion ([\ce{M^+}]), each multiplied by the term (1+D), where D represents dilution of the acid. The second equation states that the initial molar concentration of the acid $C_a^0$ equals the anion concentration ([\ce{A^-}]) times the term (1+D).

    The dissociation of water and base is represented by two equilibrium expressions. The first, $K_w = [\ce{OH^-}][\ce{H^+}]$, is the equilibrium expression for the self-ionization of water, where $K_w$ is the ion-product of water. The second equation, $K_b = \frac{[\ce{M^+}][\ce{OH^-}]}{[\ce{MOH}]}$, is the equilibrium expression for the base, where $K_b$ is the base dissociation constant.

    $K_w = \SI{1e-14}{\mol^2 / dm^{-6}}$ are the units for the ion product of water at 25°C. However, it's worth noting that it is temperature-dependent. For instance, the pH of neutral water isn't always 7, but varies from 7.47 at \SI{0}{\degree C} to 6.92 at \SI{30}{\degree C}. This is due to changes in $K_w$ with temperature.

    The final equation provided is a more complex expression that includes both terms for the conservation of charge and conservation of mass, as well as equilibrium constants. This equation essentially allows you to calculate the dilution factor, given the known initial concentrations of the acid and base, and the equilibrium constants.

    (ChatGPT-4 2023-07-04)
\end{quoting}
\subsubsection{Fluid mechanical model}
This section presents a fluid mechanical model that describes the reaction of an acid and a base in a flowing system. The model is based on several key principles of fluid dynamics, including conservation of mass (acid and base), conservation of volume flux, conservation of momentum flux, and conservation of buoyancy flux.

The first equation represents the conservation of acid in the flow:
\begin{equation}
    \frac{\dif}{\dif z}\left(\pi b^2 w[\ce{A^-}]\right) = 0
\end{equation}
This equation states that the rate of change of the acid concentration ([\ce{A^-}]) with respect to position ($z$) is zero, indicating that the acid concentration in the fluid is constant along the flow.

The second equation represents the conservation of base in the flow:
\begin{equation}
    \frac{\dif}{\dif z}\left(\pi b^2 w \left([\ce{MOH}]+[\ce{M^+}]\right)\right) = 2\alpha \pi bwC^0_b
\end{equation}
This equation states that the rate of change of the sum of the concentrations of metal hydroxide ([\ce{MOH}]) and metal ion ([\ce{M^+}]) with respect to position ($z$) is equal to twice the product of the initial base concentration, the radius of the flow tube, and a proportionality constant $\alpha$.

The third equation represents conservation of volume flux in the flow:
\begin{equation}
    \frac{\dif}{\dif z}\left(\pi b^2 w\right) = 2\alpha \pi bw
\end{equation}
This equation shows that the rate of change of volume flux with respect to position ($z$) is equal to twice the product of the radius of the flow tube, the velocity of the fluid, and a proportionality constant $\alpha$.

The fourth equation represents the conservation of momentum flux in the flow:
\begin{equation}
    \frac{\dif}{\dif z}\left(\pi b^2 w^2\right) = -\frac{\Delta \rho g \pi b^2}{\rho_0}
\end{equation}
This equation states that the rate of change of momentum flux with respect to position ($z$) is equal to the negative of the gravitational acceleration times the change in density times the cross-sectional area of the flow tube divided by the initial fluid density.

The fifth equation represents the conservation of buoyancy flux in the flow:
\begin{equation}
    \frac{\dif}{\dif z}\frac{\left(\pi b^2 w \Delta \rho\right)}{\rho_0} = 0
\end{equation}
This equation indicates that the rate of change of the buoyancy flux with respect to position ($z$) is zero, implying that the buoyancy flux is conserved in the fluid flow.

The final three equations can be derived from the above equations and they capture the changes in the volume flux ($Q$), momentum flux ($M$), and the acid concentration with respect to position ($z$) in the fluid flow:
\begin{align}
    \frac{\dif Q}{\dif z}    & = 2\alpha \pi^{\frac{1}{2}} M^{\frac{1}{2}} \\
    \frac{\dif M}{\dif z}    & = \frac{BM}{Q}                              \\
    \frac{[\ce{A^-}]}{C^a_0} & = \frac{Q_0}{Q}
\end{align}
\subsubsection{Titration curves}
\begin{figure}[htbp]
    \centering
    \includegraphics[width = 0.5 \textwidth]{./img/figure81.png}
    \caption{Figure to show titration curve of \ce{HNO3 + NaOH}.}
    \label{titration}
\end{figure}
Figure \ref{titration}: the dashed and solid lines show theoretical results while the points and crosses are experimental. Source of disagreement is most likely contamination (very difficult problem).
\subsubsection{Example chemistry: jet experiment}
Figures \ref{jet1}, \ref{jet2} show an example jet experiment.
\begin{figure}[htbp]
    \centering
    \includegraphics[width = 0.5 \textwidth]{./img/figure82.png}
    \caption{Figure to show jet experiment of acid and litmus.}
    \label{jet1}
\end{figure}
\begin{figure}[htbp]
    \centering
    \includegraphics[width = 0.5 \textwidth]{./img/figure83.png}
    \caption{Figure to show validation between predicted and experimental results for acid and litmus experiment.}
    \label{jet2}
\end{figure}
\subsubsection{Example of tsunami loading on buildings: ASCE6}
\begin{itemize}
    \item Computational and experimental study
    \item Measurements of surface elevations and forces on buildings
    \item Simulation and experiments at UCL's mechanical engineering department
\end{itemize}
Figures \ref{tsunami1}, \ref{tsunami2} show some physical experiments.
\begin{figure}[htbp]
    \centering
    \includegraphics[width = 0.8 \textwidth]{./img/figure84.png}
    \caption{UCL Mechanical Engineering tsunami experiment.}
    \label{tsunami1}
\end{figure}
\begin{figure}[htbp]
    \centering
    \includegraphics[width = 0.8 \textwidth]{./img/figure85.png}
    \caption{City scale tsunami experiment / simulation.}
    \label{tsunami2}
\end{figure}