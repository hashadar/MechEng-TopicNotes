\section{Influence of temperature on the response of a straight pipe}
\subsection{Meaning and significance of stress terms}
\subsubsection{Principal stress}
Principal stress is a measure which defines the maximum normal stress which may be applied to a body of interest and where that stress is located. Principal stress acts on the principal plane (an oblique plane at some angle $\theta$) and has the condition that there is zero shear stress on this plane. The resultant normal stresses acting on the principal plane, $\sigma_n$, is the principal stress. The normal stress can take a maximum or minimum value.

In the 3D case, we find that there exist three principal planes (where the shear stress is zero), which are orthogonal and each have their own maximum / minimum normal stresses. 

\url{https://www.informit.com/articles/article.aspx?p=1729271&seqNum=4}
\url{https://mechcontent.com/principal-stress/#What_is_Principal_stress}
\subsubsection{Von Mises stress}
\subsubsection{Stress magnitude}
\subsection{Relationship between axial stress and temperature of length constrained 3D pipe}\label{part1b}
\subsection{3D pipe ANSYS}\label{part1c}
\subsubsection{Relationship between maximum stress in temperature range}
\subsubsection{Comparison against theoretical result and discussion}
\subsection{1D pipe ANSYS}\label{part1d}
\subsubsection{Relationship between maximum stress in temperature range}
\subsubsection{Comparison against theoretical result and discussion}
\subsection{Discussion}
\subsubsection{Difference and similarities between results from \ref{part1b}, \ref{part1c}, \ref{part1d} as pipe slenderness changes}
\subsubsection{Model errors encountered using ANSYS Static Structural model}