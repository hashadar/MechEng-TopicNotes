\documentclass[11pt]{article}
\usepackage{graphicx}
\usepackage{hyperref}
\usepackage{appendix}
\usepackage{amsmath}
\usepackage[version = 4]{mhchem}
\usepackage{amsthm}
\usepackage{amssymb}
\usepackage{float}
\usepackage{multirow}
\usepackage{commath}
\usepackage{booktabs}
\renewcommand{\arraystretch}{1.2}
\usepackage{siunitx}
\sisetup{detect-all}
\usepackage{listings}
\usepackage{color} %red, green, blue, yellow, cyan, magenta, black, white
\definecolor{mygreen}{RGB}{28,172,0} % color values Red, Green, Blue
\definecolor{mylilas}{RGB}{170,55,241}
\usepackage[a4paper,margin=20mm]{geometry}
\numberwithin{equation}{section}
\setlength{\parskip}{\baselineskip}
\setlength{\parindent}{0pt}
\hypersetup{
    colorlinks=true,
    linkcolor=black,
    filecolor=black,      
    urlcolor=black,
    citecolor=black
}
\urlstyle{same}
\begin{document}
\title{\textbf{UCL Mechanical Engineering 2020/2021}\\MECH0015 Coursework}
\author{NCWT3}
\maketitle
\tableofcontents
\listoffigures
\section{Question 1}
\subsection{Materials science and engineering of aero engine blades}
The materials and engineering for a turbine blade must be able to withstand extreme temperatures, high stresses and even vibrations. To contextualise these requirements, the temperatures inside a gas turbine can reach be in the range of \SI{1800}{\kelvin} REFFFF and for example, the Rolls Royce Trent engine compressor spins at 3300RPM REFFFF. Any spinning component is also bound to face vibrational forces from a variety of sources. 

Nickel superalloys are currently used for this application, as they have excellent properties at high temperatures. Let us first look at the chemical composition of these alloys, which are often complex. Nickel is used due to its versatile $\gamma '$ precipitates. It is paired with elements such as \ce{Cr}, \ce{Al}, \ce{Ti}, \ce{C} contributing to properties such as a corrosion and oxidisation resistance and strength. Refractory metals such as \ce{Re} and \ce{W} are added for solid solution strengthening and for their extremely high melting points. These alloys form a complex array of phases with a primary disordered FCC matrix ($\gamma$) of \ce{Ni} and other elements containing an ordered GCP FCC matrix ($\gamma '$) consistent of \ce{Ni3(Al, Ti)}. The $\gamma$ phase acts as a structure for the precipitates and provides ductility. The $\gamma '$ phase is the main strengthening phase. Other phases such as TCP and carbide phases can form, with TCP phases worsening mechanical properties and carbide providing strength and grain boundary stability. 

The microstructure of these alloys is extremely important in yielding the properties required. Techniques such as directional solidification and single crystal growth seek to form a microstructure contributing to enhanced mechanical properties. By forming the grain crystals in an elongated fashion, along only one axis, we can significantly improve the creep resistance parallel to the grain direction. This is beneficial as we know that if we align this with the centrifugal force direction, we can counteract the stress that the blade would experience. Single crystal growth further enhances the properties as it rids the microstructure of grain boundaries – which can change the properties of the material drastically. With no grain boundaries, we have a lack of defects. This would actually decrease our yield strength as there are no longer any slip prevention mechanisms in the structure, however we are also decreasing the amount of creep deformation, which is critical for this high temperature application. However, the manufacture of these single crystal blades is very expensive and requires and much additional processing. 

Despite the materials used and the microstructure formed, we still require additional mechanisms to prevent the turbine blades from failing. To provide cooling to the blades, we can incorporate methods such as a thermal barrier coating and internal/external air cooling. A coating of a ceramic with a low thermal conductivity could help reduce the turbine blade temperatures by a significant percentage. This coating also prevents oxidisation and corrosion of the blade. We can also incorporate a hollow turbine blade with cooling channels to internally cool the blade. This can be further supplemented by incorporating small holes along the length of the blade which allows the cooling air to flow along the boundary of the blade and form a protective layer of cool air against the high temperature gas in the compressor. These cooling channels can be incorporated using moulds and the holes can be machined using electrical discharge machining or lasers, which are suited to the small scales of the holes.

To conclude, a chemical composition and microstructure designed to be strong at high temperatures coupled with active cooling methods allow the blades to survive in temperatures above their melting points. Combined with higher-than-normal manufacturing tolerances and rigorous quality control, turbine blades are made to survive the harsh environment of a jet engine compressor.
\section{Question 2}
\subsection{Goodman plot}
The Goodman rule is a mechanism which allows us to reduce the testing requirement associated with plotting SN curves, by telling us the “shift” of  the curve for different mean stresses. If we have $\sigma_m = 0$, then we can "replot" for positive values of $\sigma_m$. To maintain $N_{f}$ (lifetime) for different stress values, we need to reduce $\sigma_a$, and for $\sigma_a = 0$, we find that $\sigma_u = \sigma_{uts}$. The Goodman plot is hence an equivalence diagram (of $\sigma_m$ and $\sigma_a$), showing a linear relationship between $\sigma_m$ and $\sigma_a$. We use it to find the shift for a given $\sigma_m$, and can use that to replot our SN curve. 
\subsection{Splash zone}
The splash zone is the region where the legs of the rig experience constant wetting due to waves washing on and off the metal. Tides also change the water level and expose different parts of the leg to seawater and air. The net result is a cycle of wet and dry for the legs. Seawater is a solution containing many different ions, which act as an electrolyte. As the legs can act as an anode, the net result is wet corrosion. We may also see some dry corrosion, as the PB ratio of Fe is above 2, this oxidisation will result in flaking and loss of material. Over time, the material will lose structural integrity and fail without adequate anti-corrosion measures. 
\subsection{Link between fatigue lifetime and yield stress}
The fatigue lifetime of a metal and its stress are linked through the SN curve, which can be used to determine after how many cycles a material would fail for a given stress. This can be approximated through Basquin's law: $\Delta \sigma N_f^a = C$. We also have two types of failure that can occur, dependent on the lifetime of the material. For less than $10^3$ cycles, we see that failure occurs above $\sigma_y$. This results in plastic deformation of the material at a high stress (high cycle fatigue - HCF). For more than $10^3$ cycles, we see that failure occurs below $\sigma_y$. This results in elastic deformation of the material at a low stress (low cycle fatigue - LCF). The yield stress is a testing point for the different fatigue cycles. 
\begin{thebibliography}{00}

\end{thebibliography}
\end{document}