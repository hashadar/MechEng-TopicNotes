\documentclass[11pt]{article}
\usepackage{graphicx}
\usepackage{hyperref}
\usepackage{amsmath}
\usepackage{amsthm}
\usepackage{amssymb}
\usepackage[all=normal,floats,leading,paragraphs,charwidths,tracking,wordspacing]{savetrees}
\usepackage{float}
%\usepackage[version = 4]{mhchem}
\usepackage{multirow}
\usepackage{commath}
\usepackage{booktabs}
\usepackage{adjustbox}
\usepackage{subcaption}
\usepackage{rotating}
\usepackage{natbib}
%\renewcommand{\arraystretch}{1.2}
\usepackage[nottoc,numbib]{tocbibind}
\usepackage{siunitx}
\sisetup{detect-all}
\DeclareSIUnit{\atm}{atm}
\usepackage{listings}
\usepackage{color} %red, green, blue, yellow, cyan, magenta, black, white
\definecolor{mygreen}{RGB}{28,172,0} % color values Red, Green, Blue
\definecolor{mylilas}{RGB}{170,55,241}
\usepackage[a4paper,margin=20mm]{geometry}
%\numberwithin{equation}{section}
\setlength{\parskip}{\baselineskip}
\setlength{\parindent}{0pt}
\NewDocumentCommand\Pounds{o}{%
\pounds\IfNoValueTF{#1}%
{\relax}{\,\num[]{#1}}}
\hypersetup{
    colorlinks=true,
    linkcolor=black,
    filecolor=black,      
    urlcolor=black,
    citecolor=black
}
\urlstyle{same}
\lstset{language=Matlab,%
    %basicstyle=\color{red},
    breaklines=true,%
    morekeywords={matlab2tikz},
    keywordstyle=\color{blue},%
    morekeywords=[2]{1}, keywordstyle=[2]{\color{black}},
    identifierstyle=\color{black},%
    stringstyle=\color{mylilas},
    commentstyle=\color{mygreen},%
    showstringspaces=false,%without this there will be a symbol in the places where there is a space
    numbers=left,%
    numberstyle={\tiny \color{black}},% size of the numbers
    numbersep=9pt, % this defines how far the numbers are from the text
    emph=[1]{for,end,break},emphstyle=[1]\color{red}, %some words to emphasise
    %emph=[2]{word1,word2}, emphstyle=[2]{style},    
}
\begin{document}
\begin{titlepage}
    \begin{center}
        \vspace*{1cm}
             
        MSIN0068 Project Management\\
        2021/22
 
        \vspace{1.5cm}

        {\LARGE \textbf{Individual Coursework} \par}
             
        \vspace{1.5cm}
 
        Anonymous
        
        \vfill

        Word count: 511

        \vspace{1.5cm}

        University College London\\
        Torrington Place\\
        LONDON WC1E 7JE
             
    \end{center}
 \end{titlepage}
\newpage
\section*{Question B1}
\section*{Question B2}
Chipmaker's fab projects utilise a staggered approach to their construction. This consists of six phases which are: Programming, Design, Base-build, Fit-out, Tooling and Ramp-up. Staggering these phases can bring advantages and disadvantages. 

As can be seen from Exhibit 4, both the TD and HVM projects stagger many of the phases, with no more than three phases occurring simultaneously. Primarily, the programming, design and base-build occur simultaneously, whilst fit-up, tooling and ramp-up are staggered consecutively. Combining the programming and design phases can be advantageous in this scenario as many of the work has previously been done. As Chipmaker has implemented a `Design Reuse' policy, overarching design choices need not be made as there is a pre-existing foundation for new projects to be built upon. As stated in the report, `equipment, space and routing layouts' are to be reused in new projects. Hence, less time is taken to design these aspects. Programming is integral to this design process in the first place, as this is where the core requirements of what needs to be built are decided. If these are agreed upon early in the programming phase, the design team can get started on those aspects in advance, rather than waiting for niche, unrelated details of the project to be finalised. However, this system has the inherent disadvantage that any changes made in this early `malleable' period, will affect the productivity of the project, as designers will have to restart any work on certain aspects. This may be especially problematic if the base-build has reached a stage where rebuilding is no longer possible (e.g. due to cost or time). 

The question that arises is whether it is more efficient to stagger the programming and design, risking having to redesign parts of the project, or to finalise programming before working on the design phase. An argument in favour of staggering the programming and design phases is that many aspects of Chipmaker's fab projects utilise the aforementioned `Design Reuse' policy, increasing the efficiency of the programming and design phase by providing a template. An argument against the staggered approach is that Chipmaker's needs are changing in the market (as can be seen from the need to change Chipmaker's business strategy after market events). These changing needs can manifest themselves at any point during the project's timescale. Hence, if one were to not utilise the staggered approach, one would be able to iron out these hiccups without affecting further development phases (which if had been started concurrently, would waste more resources). 

The decision to stagger the fit-out, tooling and ramp-up is a sensible one as many of the systems require testing within their environment of use. For example, if there are problems with the fit-up, they may only be discovered in the tooling phase, which would require there to be a secondary fit-up, cleaning up any errors. If the tooling phase has completed certain aspects of the production line, then ramp-up and testing of those parts of the system can take place concurrently with the tooling of other parts of the project, increasing the overall efficiency. 

\section*{Question B3}
The design reuse policy is a centralised process for implementing changes to HVM fabs. Due to the inherently ever-changing nature of the R\&D based TD fabs, new technologies and frameworks would cost more money and resources to implement in HVM fabs. Chipmakers solution to this is to reuse designs where possible, implementing changes where tangible benefit can be ascertained. 

The changes are overseen by a board, who scrutinise a proposed change. This scrutinization may be seen as a potential drawback by a project manager as they would like to adhere to a certain timescale. Hence, they may choose to make certain decisions in the programming and design phase that avoid making changes that may positively influence the project. This is partly what the Design Reuse policy is trying to achieve, as it would like to reduce the time and resources spent on designing certain elements of the fab. However, this can also work against the project, as these effectively limit the fab to work on a constrained basis of inputs and guides. Of course, if the project manager can demonstrate that the changes made are substantive and will produce better results, the board should have no problem in approving these changes. Despite this, the scope of each fab must be considered individually. As Chipmaker is an international firm, with fab projects across the world, certain practices may not be directly compatible with a particular environment. Demonstrably, making changes to suit the environment may only give a marginal benefit but would improve the efficiency and compatibility of the fab project in that environment. An example of this may be utilising different building materials or utilising a different method for delivery of the product (rail vs road). These changes would not necessarily contribute to the technology manufactured inside the plant but would nonetheless be a `quality of work' improvement. Hence, it would be beneficial to implement preliminary case studies for each fab where improvements and designs can be streamlined based on the context of the project. This would allow for the fab projects to broaden their scope to the environment and take advantage of available resources, whilst simultaneously combating problems that may arise due to a particular environment. 

To conclude, the board's aim is to ensure that the rate of technology transfer is not affected. This may be seen by some project managers as limiting in the case that they must adhere to their pre-existing guidelines and design their projects to conform. However, due to the global scope of the fab projects, where the scope is contextual based on a variety of external factors, it may be unreasonable to assume that designs may be reused in all scenarios. The bureaucratic nature of the board may also be seen as a hurdle by some project managers, with some choosing to avoid the process altogether. Streamlining the process would help in reducing the barrier to implementing changes. This may be achieved through working with project managers directly, rather than requiring audit reports and research to be done beforehand and scrutinised afterwards. With direct cooperation at the programming and design phase levels, changes may be more easily approved and implemented.

\section*{Question B4}
\end{document}