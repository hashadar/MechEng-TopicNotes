%%%%%%%%%%%%%%%%%%%%%%%%%%%%%%%%%%%%%%%%%
% Lachaise Assignment
% LaTeX Template
% Version 1.0 (26/6/2018)
%
% This template originates from:
% http://www.LaTeXTemplates.com
%
% Authors:
% Marion Lachaise & François Févotte
% Vel (vel@LaTeXTemplates.com)
%
% License:
% CC BY-NC-SA 3.0 (http://creativecommons.org/licenses/by-nc-sa/3.0/)
% 
%%%%%%%%%%%%%%%%%%%%%%%%%%%%%%%%%%%%%%%%%

%----------------------------------------------------------------------------------------
%	PACKAGES AND OTHER DOCUMENT CONFIGURATIONS
%----------------------------------------------------------------------------------------

\documentclass{article}

\input{structure.tex} % Include the file specifying the document structure and custom commands

%----------------------------------------------------------------------------------------
%	ASSIGNMENT INFORMATION
%----------------------------------------------------------------------------------------

\title{Designing a Lap Simulator for the Shell Eco-Marathon\\Intermediate Report} % Title of the assignment

\author{Supervisor: Tim Baker\\Student: Hasha Dar} % Author name and email address

\date{University College London --- \today} % University, school and/or department name(s) and a date

%----------------------------------------------------------------------------------------
\usepackage{listings}
\usepackage{color} %red, green, blue, yellow, cyan, magenta, black, white
\definecolor{mygreen}{RGB}{28,172,0} % color values Red, Green, Blue
\definecolor{mylilas}{RGB}{170,55,241}
\lstset{language=Matlab,%
    %basicstyle=\color{red},
    breaklines=true,%
    morekeywords={matlab2tikz},
    keywordstyle=\color{blue},%
    morekeywords=[2]{1}, keywordstyle=[2]{\color{black}},
    identifierstyle=\color{black},%
    stringstyle=\color{mylilas},
    commentstyle=\color{mygreen},%
    showstringspaces=false,%without this there will be a symbol in the places where there is a space
    numbers=left,%
    numberstyle={\tiny \color{black}},% size of the numbers
    numbersep=9pt, % this defines how far the numbers are from the text
    emph=[1]{for,end,break},emphstyle=[1]\color{red}, %some words to emphasise
    %emph=[2]{word1,word2}, emphstyle=[2]{style},    
}

\begin{document}

\maketitle % Print the title

%----------------------------------------------------------------------------------------
%	INTRODUCTION
%----------------------------------------------------------------------------------------
\section{Project Background \& Problem}
The Shell Eco-Marathon is a competition at the school and university level for students focusing on energy optimisation in vehicles. The aim of the competition is to develop new innovations in energy efficiency for vehicles on the road with the idea of reducing carbon emissions \citep{ShellEco2021}. I will be focusing on the 'Prototype' category for the competition, which involves designing a hydrogen-powered single-seater vehicle with a focus on attaining ultra-efficiency through the optimisation of the aerodynamic and performance characteristics of the vehicle.

For an ultra-energy-efficient vehicle, there are many sources of inefficiency. Some of them are aerodynamic drag (air resistance), friction drag (from the road surface) and losses in the powertrain (electricity generation, mechanical losses) \citep{Wei2019}. The design itself of the car must be optimised to increase its performance such as the weight of the vehicle \citep{Tsirogiannis2019}.

A problem that arises when investigating and modelling these variables is that there is no way of testing certain performance characteristics within the holistic context of the competition virtually. Currently, to test newly designed components and its overall impact on the performance of the vehicle, the part must be prototyped and installed on the vehicle, and then experimentally tested. This presents a problem as this is costly, time-consuming and constrains the number of prototypes that can be made.

A transient, dynamic lap simulator will allow the analysis of the vehicle's various performance characteristics. This will allow the team to optimise and prototype designs much more effectively. The team would be able to test the vehicle on the final test track, without the cost of being there physically.
\section{Project Aim}
During the design phase of the vehicle, a lot of work must go into modelling performance variables of the vehicle. This may be done through mathematical analysis and through the use of software such as Computational Fluid Dynamics (CFD) and Finite Element Analysis (FEA). These methods allow us to test whether our design meets our performance criteria (in a virtual sense). This project will focus on bringing together the various performance characteristics of the vehicle to be able to analyse the car in a contextual manner. A lap simulator will allow the team to see how their prototype design will function in a real-world scenario, rather than as a discrete set of performance variables. Hence, with a lap simulator the team will be able to:
\begin{itemize}
	\item unify the various computational models that assess parts of the vehicles performance
	\item see where the primary energy inefficiencies are in a transient, dynamics test, allowing targeted improvement of the vehicle's performance
	\item reduce the number of physical tests for the vehicle,
	      \begin{itemize}
		      \item lowering cost
		      \item reducing development time
		      \item minimising the number of prototype parts manufactured
	      \end{itemize}
\end{itemize}
\section{Methodology}
\begin{enumerate}
	\item Research and select a race track for modelling
	\item Build racetrack analysis code (to find turning angles at any moment)
	\item Research vehicle performance parameters (motor, drivetrain, vehicle)
	\item Integrate vehicle performance parameters into MATLAB model
	\item Integrate vehicle constraints (friction limits, cornering force, electrical power)
	\item Design Simulink model to link performance parameters together
	\item Design motor control loops (dependent on race strategy)
	\item Generate vehicle data (displacement, velocity, acceleration)
	\item Generate lap time for vehicle
	\item Optimise race strategy to improve lap time
	\item Improve fidelity of Simulink model
	\item Develop GUI for simulation
\end{enumerate}
\subsection{Track data and vehicle parameters}
The following MATLAB code takes racetrack racing line coordinate data and turns adjacent points into vectors. The angle between adjacent vectors are then calculated and stored in an array. This gives us the instantaneous racing line track curvature at any point. Next, we see that the performance parameters for the various vehicle subsystems are stored. 
\lstinputlisting{../MCode/model1.m}
Next, a Simulink model was created to model the various performance parameters
\bibliographystyle{agsm}
\bibliography{./bib/export}

%----------------------------------------------------------------------------------------

\end{document}
