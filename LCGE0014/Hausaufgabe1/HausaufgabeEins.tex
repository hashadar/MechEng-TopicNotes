\documentclass[12pt]{article}
%% Language and font encodings
\usepackage[ngerman]{babel}
\usepackage[utf8x]{inputenc}
\usepackage[T1]{fontenc}
\usepackage{fontspec}
%\setmainfont{Times New Roman}

%% Sets page size and margins
\usepackage[a4paper,top=2cm,bottom=2cm,left=3cm,right=3cm,marginparwidth=1.75cm]{geometry}

%% Useful packages
\usepackage{amsmath}
\usepackage{graphicx}
\usepackage[colorinlistoftodos]{todonotes}
\usepackage[colorlinks=true, allcolors=blue]{hyperref}

\title{LCGE0014 Hausaufgabe Eins}
\author{Hasha Dar}

\begin{document}
\maketitle

Duzen im gesch{\"a}ftlichen Umfeld ist ein relatives neues Konzept im Deutschland. Der Artikel sagt, dass manche Mitarbeiter von ausl{\"a}ndischen Firmen inspirierend sind, um das Du miteinander zu benutzen. Die Firmen, die eine offene und lockere Arbeitsplatz unterst{\"u}tzen, tendenziell Neuunternehmen oder Firm in IT- und Internetbranche sein. Hier kann man sehen, dass das Du hilft, die traditionelle Arbeitsplatz Hierarchien aufzubrechen. Wenn man in einem Neuunternehmen arbeitet, ist es wichtig, eine enge Verbindung mit den Kollegen zu haben. Deshalb gibt es ein starkes Argument, um zu duzen. Manche Firmen gehen noch einen Schritt weiter und wollen, dass ihre Mitarbeiter nur duzen m{\"u}ssen oder die Verwendung vom Du von der Spitze der Hierarchie aktiv gef{\"o}rdert haben. Dies kann ein Vorteil sein, weil es erm{\"o}glicht, die Barrieren zwischen Junior- und Senior-Mitarbeitern zu durchbrechen. Wenn in einem wettbewerbsorientieren Arbeitsfeld die M{\"o}glichkeit besteht, die Zusammenarbeit innerhalb einer Firma zu verst{\"a}rken, liegt es im Interesse der Firma, den Einsatz vom das Du zu f{\"o}rdern.\\

Anderseits sieht man, dass immer noch duzen im Arbeitsplatz nicht {\"u}blich ist. Nat{\"u}rlich beg{\"u}nstigen Menschen an der pr{\"a}existent sozialen Hierarchie von Heute festzuklammern. Der Artikel sagt, dass diese {\"a}nderungen kein Vorteil f{\"u}r die Angestellten vorangebracht haben, sondern schadete die zwischenmenschliche Interaktion. Wenn eine Firma einen Duz-Zwang vorstellt, haben Forscher gesagt, dass dies f{\"u}r die Kollegen ungem{\"u}tlich sein k{\"o}nnte. Vielleicht die Firma muss mit Klienten verhandeln, wo das Du in einem kulturellen/sozialen Kontext als unangemessen angesehen w{\"u}rde. Ein weiteres Beispiel ist der Tonunterschied zwischen Duzen und Siezen in der Beziehung mit einem Feedback an einen Kollegen. Wenn man duzt, wird Feedback scharfer und pers{\"o}nlicher klingen, w{\"a}hrend wenn man siezt, ist es formeller und h{\"o}flicher. \\

Meine Meinung ist voreingenommen, weil Englisch meine Muttersprache ist, wo das Konzept von duzen und siezen nur in der Antike existiert. In der britische Arbeitskultur sind F{\"o}rmlichkeit und Professionalit{\"a}t Eckpunkte aber wegen der Coronapandemie ist eine Verlagerung zum Arbeiten von zu Hause aus der Norm. Mit dem Durchbrechen dieser sozialen Barrieren habe ich eine entspannte Atmosph{\"a}re im Arbeitsplatz erlebt. Daher bin ich der Meinung, dass dass eine Entspannung im Arbeitsumfeld und somit „duzen im Job“ dauern wird. Es f{\"o}rdert die Zusammenarbeit und es erm{\"o}glicht bessere Beziehungen zu Ihren Kollegen aufzubauen. Wenn man Kollegen, die man auch als Freunde bezeichnen kann, w{\"u}rde dies meiner Meinung nach nur eine freundlichere und freiere Arbeitsatmosph{\"a}re schaffen und das Unternehmen als Ganzes verbessern. 

\end{document}
