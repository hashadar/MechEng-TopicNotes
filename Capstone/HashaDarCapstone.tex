\documentclass[12pt]{article}
\usepackage{multicol}
\usepackage{graphicx}
\usepackage{natbib}
\newcommand{\citeprimsec}[2]{\citep[][cited by \citealp{#2}]{#1}}
\usepackage{amsmath}
\usepackage{amssymb}
\usepackage{mathtools}
\usepackage[english]{babel}
\usepackage{siunitx}
\sisetup{detect-all}
\usepackage[a4paper,width=150mm,top=25mm,bottom=25mm]{geometry}
\numberwithin{equation}{section}

\setlength{\parskip}{\baselineskip}%
\setlength{\parindent}{0pt}%
\begin{document}

\title{Mechanical Engineering: Year One Capstone Assesment}
\date{2019/2020}
\author{University College London}
\maketitle
\begin{flushleft}

\section{Blades}
\subsection{For the blades of the wind turbine, composite materials are usually employed. Why is this the case?}
A composite material may be employed for their favourable properties. Composites materials can have variable (favourable) properties depending on their composition; yielding better fatigue strength, elasticity and corrosion resistance than an alternative e.g. an aluminium alloy. The orientation of the fibres in the composites matrix can be specifically arranged to combat stress (in this case the force of the wind on the blade), reducing the probability of failure (cracking, deformation) in the structure.

\subsection{What common composites might be employed in this application, and what are their relative merits and benefits in comparison to each other?}
The composite used is likely to be of a fiber-reinforced matrix. The fiber used is likely to be glass or carbon based. Alternatives such as basalt fibers have also been used. Most common are 'E-Glass' fibers, used for their stiffness, tensile and compressive strength. Glass fibers with modified compositions, yielding higher strength, have been developed but are seldom used due to much greater cost.

The matrix material is likely to be a thermoset plastic rather than a thermoplastic. This is due to thermosets having more favourable production characteristics: lower curing temperatures/times and lower viscosity, leading to high processing speed. The most common thermosets used are epoxy and polyester resins. However, an advantage to using a thermoplastic is their recyclability. \citeprimsec{windTurbineMaterial2}{windTurbineMaterial} 

\subsection{What significant issues can you see with using composites in this engineering application? [For 
example, you could consider economic or environmental
challenges]}
A common manufacturing technique to produce turbine blades is called vacuum assister resin transfer molding. This is where fiber sheets are placed and aligned in a mold, covered in a vacuum bag and a resin injected. The resin is then left to cure \citep{VARTM}. This process may be labour intensive as the placing and direction of the fibers is a delicate process, requiring human input. Furthermore, once a blade has been manufactured, the blade msut undergo significant testing (in some cases several months), incurring more cost. If a blade is rejected from testing, the material in the blade may be quite difficult to extract and reuse. 

Building on reuse, recovered fibers prevent a cost barrier as in most cases, recovered fibers are more expensive than new fibers, to use on industrial scales. However, their reuse can be found in other fields, such as cement production \citep{fiberCement}.

\subsection{The power (W in J/s) produced by a wind turbine depends on blade length (B), the incoming wind speed (V), and air density (\(\rho\)). Derive one dimensionless number relevant to the problem using W as the dependent parameter. Use this dimensionless number to comment on the implications of doubling the blade length.}
Using Buckingham Pi:
\begin{align}
  [W] &= ML^2T^{-3}\\
  [B] &= L\\
  [V] &= LT^{-1}\\
  [\rho] &= ML^{-3}
\end{align}
\begin{align}
  W &= B^a V^b \rho^c\\
  ML^2T^{-3} &= L^a L^b T^{-b} M^c L^{-3c}\\
  c = 1, b &= 3, a = 2\\
  W &= B^2 V^3 \rho
\end{align}
\begin{gather}
  k = \frac{W_1}{B^2 V^3 \rho} \textrm{ and } k = \frac{W_2}{4B^2 V^3 \rho}\\
  W_1 = \frac{W_2}{4} \\
  4W_1 = W_2
\end{gather}
From this we can see that doubling the blade length (\(B\)), quadruples the power output of the wind turbine.

\subsection{Considering the answer above, discuss the trade-offs associated with choosing longer blades for a turbine of a fixed height.}
Naturally, choosing larger blades for a turbine of a fixed height creates a limit to how large the blades can be before the turbine's tower would not be able to structurally support the weight of the blades. Hence, using larger blades requires the use of stronger materials in order to support their weight. Using stronger materials is more expensive to procure and manufacture, driving up initial costs. If the turbine cannot produce enough power to become econmically viable over its lifetime, this would cause problems for the manufacturer.

\section{Gearbox (dynamics)}
\subsection{Derive a simple relationship for the gear ratio expressed as a function of number of teeth in the sun and ring gears of an epicyclic (or planetary) gear train.}
Let us define,
\begin{itemize} 
  \item The gear ratio $i$
  \item The sun gear with subscript $S$
  \item The planet gear(s) with subscript $P$ 
  \item The ring gear with subscript $R$
  \item The carrier with subscript $C$
  \item The number of teeth $z$
  \item The modulus of the gear(s) $m$
\end{itemize}

The diameter of a gear is $d = mz$ and for two gears to mesh, their module must be the same. Hence, we can derive
\begin{gather}
  m_1 = m_2\\
  \frac{d_1}{z_1} = \frac{d_2}{z_2}\\
  \frac{z_2}{z_1} = \frac{d_2}{d_1} = \frac{r_2}{r_1}
\end{gather}
From Figure (\ref{SystemRadii}), we can see that a constraint on our system is,
\begin{equation}
  r_R = r_S + 2r_P
  \label{radiiRelationship}
\end{equation}

\begin{figure}[h]
  \centering
  \includegraphics[width = 0.7 \textwidth]{./img/GearRadii.jpg}
  \caption{Planetary gearbox radii}
  \label{SystemRadii}
\end{figure}

The gear ratio of a planetary gearbox is given by the following formula.
\begin{equation}
  i = \frac{\omega_{\textrm{output}}}{\omega_{\textrm{input}}}
\end{equation}
Considering that the sun gear is connected to the input shaft and the carrier of the planet gears is connected to the output shaft, we can simply say,
\begin{equation}
  i = \frac{\omega_S}{\omega_C}
\end{equation}
The linear velocity of the carrier is,
\begin{equation}
  v_C = \omega_C (r_S + r_P)
\end{equation}
The velocity of the point of contact between the sun and the planet gear is,
\begin{equation}
  v_S = \omega_S r_S
\end{equation}
By considering the point of contact between the planet and the ring gear as having 0 relative velocity, we can derive that the point of contact between the sun and the planet is simply $2v_C$. Hence,
\begin{align}
  v_S = 2v_C &= 2\omega_C(r_S + r_P)\\
  \omega_S r_S &= 2\omega_C (r_S + r_P)\\
  \frac{\omega_S}{\omega_C} &= \frac{2(r_S + r_P)}{r_S}\\
  &= \frac{2r_S + 2r_P}{r_S} \label{gearRatio1}
\end{align}
Rearranging equation (\ref{radiiRelationship}) and substituting into (\ref{gearRatio1}), we get,
\begin{align}
  \frac{\omega_S}{\omega_C} &= \frac{2r_S + r_R - r_S}{r_S}\\
  &= \frac{r_S + r_R}{r_S}
\end{align}
This simplifies to,
\begin{equation}
  i = 1 + \frac{r_R}{r_S}
\end{equation}
Since the number of teeth is proportional to the radius of the gear, we can substitute $z$ into our equation,
\begin{equation}
  i = 1 + \frac{z_R}{z_S}
\end{equation}
\subsection{Perform a conceptual design of an epicyclic gear system for a 1.5 MW wind turbine if the three blades spin at a design speed of 12 rpm and the high-speed shaft in the generator needs to spin at 1680 rpm. Provide information on the configuration of your proposed planetary gear set (note: the 5 laws of planetary gearing – see the provided videos) and the input/output torque ratio that can be achieved by your system. Neglect friction and assume that the angular acceleration of the gears (which are rigid and non-deformable) is zero. You must indicate the number of teeth in each gear and provide a schematic drawing.}
\subsection{Wind gusts and turbulence lead to misalignment of the drive train and premature failure of the gear components. How could this be mitigated?}
\subsection{Comment on the advantages/disadvantages of an epicyclic gear system in the context of a wind turbine gear box.}

\section{Gearbox (materials)}
\subsection{For the gears in the gearbox of the wind turbine, steel would normally be the material of choice. Why is this the case?}
\subsection{There are many different grades of steel available – what particular properties of the steel might be required for a gearbox application, and what sorts of steel would be suitable therefore?}
A gear must be manufactured to have certain properties,
\begin{itemize}
  \item High tensile strength - prevent failure when a torque (static load) is applied to the gear.
  \item High endurance strength - withstand the dynamic loads, when the gear is in use.
  \item Low coefficient of friction - reduce mechanical losses in the system.
  \item Good manufacturability - to reduce cost.
\end{itemize}
\subsection{The gears will be enclosed in a housing to help hold the mechanism together and prevent the ingress of contaminants. Suggest suitable materials for this housing, ensuring you provide justification for your suggestions (taking into account a range of factors including properties, and economic issues). Given your suggestions above, qualify these by providing consideration for how such an enclosure could be manufactured. What manufacturing processes might principally be required?}

\section{Tower}
\subsection{The tower in the picture is a single tube which is also normally made of steel. Explain why this is likely to be manufactured from a different grade of steel to that used in the gearbox. What properties are needed in this particular context?}

\section{Energy generation}
\subsection{Designers are considering the maximum power that could be generated by this turbine in two theoretical case studies. In case A, the incoming wind speed is 25 m/s and the air speed after passing through the blades is 15 m/s. In case B, the incoming wind speed is 20 m/s and the final speed is 12 m/s. Describe the concepts needed to estimate the maximum theoretical output of a wind turbine, and calculate the maximum theoretical power output for both these cases when the blade length is 37 m. Take the value of 1.2 kg/m3 for air density.}

\subsection{Consider the various terms in the energy equation as they relate to a wind turbine. Describe briefly which terms are important for the wind turbine, and connect terms in the equation to the major sources of energy loss. Comment on the implications for wind turbine efficiency.}
\subsection{The world’s most powerful commercial wind turbine today has a blade length of 82m and is rated at 9.5 MW. Comment briefly on the reasons why the numbers calculated using your theoretical approach above are far larger than this actual capacity.}

\section{Energy storage}
\subsection{In an offshore wind turbine facility, the excess energy generated is stored using a simple compressed air storage system. The wind turbine is mechanically coupled to a compressor that has a compression ratio of 200. The compressor takes in air from the surrounding at ambient pressure and temperature conditions (p0, T0) and performs a reversible adiabatic compression process. The output air from the compressor (at p1, T1) undergoes a reversible isobaric heat removal process using a heat exchanger in order to reduce the temperature to T0. The air is then stored in a high-pressure storage facility.}
\subsection{Determine:}
\subsubsection{heat and work transfers per unit mass for both the compression and heat removal processes}
\subsubsection{the total entropy change per unit mass undergone by air}
\subsection{The heat removal in the above system is achieved by a refrigerator operating on a simple vapour compression cycle. This simple vapour compression refrigerator uses ammonia (NH3) as the working fluid. The evaporator pressure is pe and the condenser pressure is pc (where pc/pe>=6). The working fluid leaves the evaporator dry-saturated and enters the compressor, where it is compressed reversibly and adiabatically. Condensation at constant pressure then takes place until the saturated liquid state is reached. This is followed by throttling to evaporator pressure.}
\subsection{Sketch the cycle on T-s and p-h diagrams}
\subsubsection{the compressor delivery temperature, and}
\subsubsection{the mass flow rate of the refrigerant}

\section{Final discussion}
\subsection{Consider all the aspects covered above and compare the trade-offs in each of these categories during wind turbine design. You may also wish to consider how they are affected by the environment: water depth, ocean waves and seafloor structure. Reflect on the consequent additional challenges in building and maintaining offshore wind turbines.}
test

\bibliographystyle{agsm}
\bibliography{references}
\end{flushleft}
\end{document}