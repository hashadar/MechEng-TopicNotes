\documentclass[11pt]{article}
\usepackage{graphicx}
\usepackage{hyperref}
\usepackage{appendix}
\usepackage{amsmath}
\usepackage{amsthm}
\usepackage[version=3]{mhchem}
\usepackage{amssymb}
\usepackage{float}
\usepackage{multirow}
\usepackage{commath}
\usepackage{booktabs}
\renewcommand{\arraystretch}{1.2}
\usepackage{siunitx}
\sisetup{detect-all}
\usepackage{listings}
\usepackage{color} %red, green, blue, yellow, cyan, magenta, black, white
\definecolor{mygreen}{RGB}{28,172,0} % color values Red, Green, Blue
\definecolor{mylilas}{RGB}{170,55,241}
\usepackage[a4paper,margin=20mm]{geometry}
\numberwithin{equation}{section}
\setlength{\parskip}{\baselineskip}
\setlength{\parindent}{0pt}
\hypersetup{
    colorlinks=true,
    linkcolor=black,
    filecolor=black,      
    urlcolor=black,
    citecolor=black
}
\urlstyle{same}
\begin{document}
\title{\textbf{UCL Mechanical Engineering 2021/2022}\\MECH0026 Problem Sheet 1 Solutions}
\author{HD}
\maketitle
\section{}
\subsection{}
For plane stress, our simplification is valid when one dimension of an object (e.g. $z$-direction) is very small compared to others, e.g. a thin sheet, loaded perpendicular to the surface. Stress tensors relating to the $z$-direction are virtually 0 ($\sigma_z = \tau_{yz} = \tau_{xz} =0$) and no loads (body or boundary) in $z$-direction. We can use compliance matrix to find out the components of our stress field.

For plane strain, our simplification is valid when one dimension of an object (e.g. in $z$-direction) is very large compared to others e.g. a long cylindrical or prismatic body loaded perpendicular to the length. The conditions of plane strain are:
\begin{enumerate}
    \item Everything is constant in the $z$-direction $\frac{\partial ()}{\partial z} = 0$
    \item $w = 0$
    \item No loads (body or boundary) in $z$-direction
\end{enumerate}
Hence, $\epsilon_z = \delta_{yz} = \delta_{xz} =0$. We can use stiffness matrix to find components of our strain field
\section{}
\begin{align}
    \phi      & = \dfrac{P}{20h^3} \left(15h^2x^2y-5x^2y^3-2h^2y^3+y^5\right)                                    \\
    \sigma_x  & = \dfrac{\partial^2 \phi}{\partial y^2} = \dfrac{P}{20h^3}\left(20y^3 -30x^2y-12h^2y\right)      \\
    \sigma_y  & = \dfrac{\partial^2 \phi}{\partial x^2} = \dfrac{P}{20h^3}\left(30h^2xy-10xy^3\right)            \\
    \tau_{xy} & = -\dfrac{\partial^2 \phi}{\partial x\partial y} = - \dfrac{P}{20h^3} \left(20h^2x-30xy^2\right)
\end{align}
satisfies harmonic relationship
\end{document}