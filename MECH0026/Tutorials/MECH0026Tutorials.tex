\documentclass[11pt]{article}
\usepackage{graphicx}
\usepackage{hyperref}
\usepackage{appendix}
\usepackage{amsmath}
\usepackage{amsthm}
\usepackage[version=3]{mhchem}
\usepackage{amssymb}
\usepackage{float}
\usepackage{multirow}
\usepackage{commath}
\usepackage{booktabs}
\renewcommand{\arraystretch}{1.2}
\usepackage{siunitx}
\sisetup{detect-all}
\usepackage{listings}
\usepackage{color} %red, green, blue, yellow, cyan, magenta, black, white
\definecolor{mygreen}{RGB}{28,172,0} % color values Red, Green, Blue
\definecolor{mylilas}{RGB}{170,55,241}
\usepackage[a4paper,margin=20mm]{geometry}
\numberwithin{equation}{section}
\setlength{\parskip}{\baselineskip}
\setlength{\parindent}{0pt}
\hypersetup{
    colorlinks=true,
    linkcolor=black,
    filecolor=black,      
    urlcolor=black,
    citecolor=black
}
\urlstyle{same}
\begin{document}
\title{\textbf{UCL Mechanical Engineering 2021/2022}\\MECH0026 Problem Sheet Solutions}
\author{HD}
\maketitle
\section{Problem Sheet 1}
\subsection{Q1}
\subsubsection{a}
For plane stress, our simplification is valid when one dimension of an object (e.g. $z$-direction) is very small compared to others, e.g. a thin sheet, loaded perpendicular to the surface. Stress tensors relating to the $z$-direction are virtually 0 ($\sigma_z = \tau_{yz} = \tau_{xz} =0$) and no loads (body or boundary) in $z$-direction. We can use compliance matrix to find out the components of our stress field.

For plane strain, our simplification is valid when one dimension of an object (e.g. in $z$-direction) is very large compared to others e.g. a long cylindrical or prismatic body loaded perpendicular to the length. The conditions of plane strain are:
\begin{enumerate}
    \item Everything is constant in the $z$-direction $\frac{\partial ()}{\partial z} = 0$
    \item $w = 0$
    \item No loads (body or boundary) in $z$-direction
\end{enumerate}
Hence, $\epsilon_z = \delta_{yz} = \delta_{xz} =0$. We can use stiffness matrix to find components of our strain field.
\subsubsection{b}
\begin{table}
    \begin{center}
        \begin{tabular}{ c c c }
            \toprule
                          & Plane Strain                              & Plane strain                              \\
            \midrule
            Stress Tensor & $\sigma = \begin{pmatrix}
                    \sigma_x  & \tau_{xy} & 0 \\
                    \tau_{yx} & \sigma_y  & 0 \\
                    0         & 0         & 0
                \end{pmatrix}$      & $\sigma = \begin{pmatrix}
                    \sigma_x  & \tau_{xy} & 0        \\
                    \tau_{yx} & \sigma_y  & 0        \\
                    0         & 0         & \sigma_z
                \end{pmatrix}$      \\
            Strain tensor & $\varepsilon = \begin{pmatrix}
                    \varepsilon_x & \gamma_{xy}   & 0             \\
                    \gamma_{yx}   & \varepsilon_y & 0             \\
                    0             & 0             & \varepsilon_z
                \end{pmatrix}$ & $\varepsilon = \begin{pmatrix}
                    \varepsilon_x & \gamma_{xy}   & 0 \\
                    \gamma_{yx}   & \varepsilon_y & 0 \\
                    0             & 0             & 0
                \end{pmatrix}$
        \end{tabular}
        \caption{Table to show stress/strain tensors in plane stress/strain.}
    \end{center}
\end{table}
\subsubsection{c}
Compliance matrix $[S]$:
\begin{equation}
    \begin{Bmatrix}
        \epsilon_{xx} \\
        \epsilon_{yy} \\
        \epsilon_{zz} \\
        \gamma_{xy}   \\
        \gamma_{yz}   \\
        \gamma_{zx}   \\
    \end{Bmatrix} = \frac{1}{E}
    \begin{Bmatrix}
        1  & -v & -v & 0      & 0      & 0      \\
        -v & 1  & -v & 0      & 0      & 0      \\
        -v & -v & 1  & 0      & 0      & 0      \\
        0  & 0  & 0  & 2(1+v) & 0      & 0      \\
        0  & 0  & 0  & 0      & 2(1+v) & 0      \\
        0  & 0  & 0  & 0      & 0      & 2(1+v) \\
    \end{Bmatrix}
    \begin{Bmatrix}
        \sigma_{xx} \\
        \sigma_{yy} \\
        \sigma_{zz} \\
        \tau_{xy}   \\
        \tau_{yz}   \\
        \tau_{zx}
    \end{Bmatrix}
\end{equation}
Stiffness matrix $[C]$:
\begin{equation}
    \begin{Bmatrix}
        \sigma_{xx} \\
        \sigma_{yy} \\
        \sigma_{zz} \\
        \tau_{xy}   \\
        \tau_{yz}   \\
        \tau_{zx}
    \end{Bmatrix} = \frac{E\left(1-v\right)}{\left(1+v\right)\left(1-2v\right)}
    \begin{Bmatrix}
        1             & \frac{v}{1+v} & \frac{v}{1+v} & 0                   & 0                   & 0                   \\
        \frac{v}{1+v} & 1             & \frac{v}{1+v} & 0                   & 0                   & 0                   \\
        \frac{v}{1+v} & \frac{v}{1+v} & 1             & 0                   & 0                   & 0                   \\
        0             & 0             & 0             & \frac{1-2v}{2(1-v)} & 0                   & 0                   \\
        0             & 0             & 0             & 0                   & \frac{1-2v}{2(1-v)} & 0                   \\
        0             & 0             & 0             & 0                   & 0                   & \frac{1-2v}{2(1-v)} \\
    \end{Bmatrix}
    \begin{Bmatrix}
        \epsilon_{xx} \\
        \epsilon_{yy} \\
        \epsilon_{zz} \\
        \gamma_{xy}   \\
        \gamma_{yz}   \\
        \gamma_{zx}   \\
    \end{Bmatrix}
\end{equation}
\subsection{Q2}
\begin{align}
    \phi      & = \dfrac{p}{20h^3} \left(15h^2x^2y-5x^2y^3-2h^2y^3+y^5\right)                                    \\
    \sigma_x  & = \dfrac{\partial^2 \phi}{\partial y^2} = \dfrac{p}{20h^3}\left(20y^3 -30x^2y-12h^2y\right)      \\
    \sigma_y  & = \dfrac{\partial^2 \phi}{\partial x^2} = \dfrac{p}{20h^3}\left(30h^2xy-10xy^3\right)            \\
    \tau_{xy} & = -\dfrac{\partial^2 \phi}{\partial x\partial y} = - \dfrac{p}{20h^3} \left(20h^2x-30xy^2\right)
\end{align}
satisfies harmonic relationship. Boundary conditions:
\begin{align}
    \tau_{xy}   & = 0 \textrm{ at } y = \pm h                                 \\
    \sigma_{yy} & = -p \textrm{ at } y = -h                                   \\
    \sigma_{yy} & = p \textrm{ at } y = h                                     \\
    \sigma_{xx} & = 0 \textrm{ at } x = 0                                     \\
    u           & = v = \frac{\partial v}{\partial u} = 0 \textrm{ at } x = L
\end{align}
\end{document}