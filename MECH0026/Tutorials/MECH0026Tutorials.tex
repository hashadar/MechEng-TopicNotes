\documentclass[11pt]{article}
\usepackage{graphicx}
\usepackage{hyperref}
\usepackage{appendix}
\usepackage{amsmath}
\usepackage{amsthm}
\usepackage[version=3]{mhchem}
\usepackage{amssymb}
\usepackage{float}
\usepackage{multirow}
\usepackage{commath}
\usepackage{booktabs}
\renewcommand{\arraystretch}{1.2}
\usepackage{siunitx}
\sisetup{detect-all}
\usepackage{listings}
\usepackage{color} %red, green, blue, yellow, cyan, magenta, black, white
\definecolor{mygreen}{RGB}{28,172,0} % color values Red, Green, Blue
\definecolor{mylilas}{RGB}{170,55,241}
\usepackage[a4paper,margin=20mm]{geometry}
\numberwithin{equation}{section}
\setlength{\parskip}{\baselineskip}
\setlength{\parindent}{0pt}
\hypersetup{
    colorlinks=true,
    linkcolor=black,
    filecolor=black,      
    urlcolor=black,
    citecolor=black
}
\urlstyle{same}
\begin{document}
\title{\textbf{UCL Mechanical Engineering 2021/2022}\\MECH0026 Problem Sheet 1 Solutions}
\author{HD}
\maketitle
\section{}
\subsection{}
For plane stress, our assumption is that our stress tensors relating to the $z$-direction are 0. For plane strain, our assumption is that there is no displacement in the $z$-direction.
\section{}
\begin{align}
    \phi      & = \dfrac{P}{20h^3} \left(15h^2x^2y-5x^2y^3-2h^2y^3+y^5\right)                                    \\
    \sigma_x  & = \dfrac{\partial^2 \phi}{\partial y^2} = \dfrac{P}{20h^3}\left(20y^3 -30x^2y-12h^2y\right)      \\
    \sigma_y  & = \dfrac{\partial^2 \phi}{\partial x^2} = \dfrac{P}{20h^3}\left(30h^2xy-10xy^3\right)            \\
    \tau_{xy} & = -\dfrac{\partial^2 \phi}{\partial x\partial y} = - \dfrac{P}{20h^3} \left(20h^2x-30xy^2\right)
\end{align}
\end{document}