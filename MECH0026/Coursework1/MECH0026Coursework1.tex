\documentclass[11pt]{article}
\usepackage{graphicx}
\usepackage{hyperref}
\usepackage{appendix}
\usepackage{amsmath}
\usepackage{amsthm}
\usepackage{amssymb}
\usepackage{float}
\usepackage{multirow}
\usepackage{commath}
\usepackage{booktabs}
\renewcommand{\arraystretch}{1.2}
\usepackage{siunitx}
\sisetup{detect-all}
\usepackage{listings}
\usepackage{color} %red, green, blue, yellow, cyan, magenta, black, white
\definecolor{mygreen}{RGB}{28,172,0} % color values Red, Green, Blue
\definecolor{mylilas}{RGB}{170,55,241}
\usepackage[a4paper,margin=20mm]{geometry}
\numberwithin{equation}{section}
\setlength{\parskip}{\baselineskip}
\setlength{\parindent}{0pt}
\hypersetup{
    colorlinks=true,
    linkcolor=black,
    filecolor=black,      
    urlcolor=black,
    citecolor=black
}
\urlstyle{same}
\begin{document}
\title{\textbf{UCL Mechanical Engineering 2021/2022}\\MECH0026 Coursework One}
\author{Hasha Dar}
\date{\today}
\maketitle
\tableofcontents
\listoffigures
\section{Description of the finite element model setup}
\section{Analytical}
Let us analyse a case of uniaxial tension and use the concept of superposition to find our stress distribution.

The biharmonic equation written in polar coordinates is:
\begin{gather}
    \left(\dfrac{\partial^2}{\partial r^2} + \dfrac{1}{r}\dfrac{\partial}{\partial r}+\dfrac{1}{r^2}\dfrac{\partial^2}{\partial \theta^2}\right)\left(\dfrac{\partial^2\phi}{\partial r^2}+\dfrac{1}{r}\dfrac{\partial\phi}{\partial r}+\dfrac{1}{r^2}\dfrac{\partial^2 \phi}{\partial\theta^2}\right) = 0
\end{gather}
We know our boundary conditions which are:
\begin{gather}
    \sigma_r, \, \tau_{r\theta} = 0, \, r = a\\
    \dfrac{\sigma_y}{\sigma_x} \rightarrow 2.5; \, r \rightarrow \infty\\
    \tau_{xy} \rightarrow 0; \, r \rightarrow \infty
\end{gather}
We can use separation of variables to obtain a solution of the form:
\begin{gather}
    \phi = R(r)\Theta (\theta)
\end{gather}
One solution is:
\begin{gather}
    \phi = \left(Ar^2 + Br^4 \dfrac{C}{r^2} + D\right)\cos \left(2\theta\right)\label{airy1}
\end{gather}
Let us also consider:
\begin{align}
    \phi = A \ln r + Cr^2 \label{airy2}
\end{align}
A linear combination of \ref{airy1} and \ref{airy2} yields a stress function that can be shown to represent the stress distribution in a large plate with a circular hole subjected to a uniform tensile stress $\sigma$. By inputting our boundary conditions we can come to the following stress function:
\begin{gather}
    \phi = \dfrac{\sigma}{2}\left(2r^2 - a^2 \ln r\right) - \dfrac{\sigma}{4}\left(r^2+\dfrac{a^4}{r^2}-2a^2\right)\cos \left(2\theta\right)
\end{gather}
Now we can find our stresses:
\begin{gather}
    \sigma_r = \dfrac{1}{r}\dfrac{\partial \phi}{\partial r} + \dfrac{1}{r^2}\dfrac{\partial^2\phi}{\partial \theta^2} = \dfrac{\sigma}{2}\left(1 - \dfrac{a^2}{r^2}\right) + \dfrac{\sigma}{2}\left(1+\dfrac{3a^4}{r^4}-\dfrac{4a^2}{r^2}\right)\cos\left(2\theta\right)\\
    \sigma_{\theta} = \dfrac{\partial^2\phi}{\partial r^2} = \dfrac{\sigma}{2}\left(1 + \dfrac{a^2}{r^2}\right) - \dfrac{\sigma}{2}\left(1 + \dfrac{3a^4}{r^4}\right)\cos \left(2\theta\right)\\
    \tau_{r\theta} = -\dfrac{\partial}{\partial r}\left(\dfrac{1}{r}\dfrac{\partial \phi}{\partial \theta}\right) = -\dfrac{\sigma}{2}\left(1 - \dfrac{3a^4}{r^4}+\dfrac{2a^2}{r^2}\right)\sin\left(2\theta\right)
\end{gather}
Superimposing a tensile stress at a \SI{90}{\degree} angle simply involves adding a $\dfrac{\pi}{2}$ term to our $\theta$ components. Let us denote the stress in the vertical direction as $\sigma_y$ and the stress in the horizontal direction as $\sigma_x$:
\begin{multline}
    \sigma_r = \dfrac{\sigma_x}{2}\left(1 - \dfrac{a^2}{r^2}\right) + \dfrac{\sigma_x}{2}\left(1+\dfrac{3a^4}{r^4}-\dfrac{4a^2}{r^2}\right)\cos\left(2\theta\right) + \\\dfrac{\sigma_y}{2}\left(1 - \dfrac{a^2}{r^2}\right) + \dfrac{\sigma_y}{2}\left(1+\dfrac{3a^4}{r^4}-\dfrac{4a^2}{r^2}\right)\cos\left(2\left(\theta + \dfrac{\pi}{2}\right)\right)
\end{multline}
\begin{multline}
    \sigma_{\theta} = \dfrac{\sigma_x}{2}\left(1 + \dfrac{a^2}{r^2}\right) - \dfrac{\sigma_x}{2}\left(1 + \dfrac{3a^4}{r^4}\right)\cos \left(2\theta\right) + \\ \dfrac{\sigma_y}{2}\left(1 + \dfrac{a^2}{r^2}\right) - \dfrac{\sigma_y}{2}\left(1 + \dfrac{3a^4}{r^4}\right)\cos \left(2\left(\theta + \dfrac{\pi}{2}\right)\right)
\end{multline}
\begin{gather}
    \tau_{r\theta} = -\dfrac{\sigma_x}{2}\left(1 - \dfrac{3a^4}{r^4}+\dfrac{2a^2}{r^2}\right)\sin\left(2\theta\right) -\dfrac{\sigma_y}{2}\left(1 - \dfrac{3a^4}{r^4}+\dfrac{2a^2}{r^2}\right)\sin\left(2\left(\theta + \dfrac{\pi}{2}\right)\right)
\end{gather}
Simplifying and inputting our boundary conditions:
\begin{gather}
    \sigma_r = \dfrac{\sigma_x + \sigma_y}{2}\left(1 - \dfrac{a^2}{r^2}\right) + \dfrac{\sigma_x + \sigma_y}{2}\left(1 + \dfrac{3a^4}{r^4}- \dfrac{4a^2}{r^2}\right)\cos \left(2\theta\right)\\
    \sigma_{\theta} = \dfrac{\sigma_x - \sigma_y}{2}\left(1 + \dfrac{a^2}{r^2}\right) - \dfrac{\sigma_x - \sigma_y}{2}\left(1 + \dfrac{3a^4}{r^4}\right)\cos\left(2\theta\right)\\
    \tau_{r\theta} = \dfrac{\sigma_y - \sigma_x}{2}\left(1 -\dfrac{3a^4}{r^4}+ \dfrac{2a^2}{r^2}\right)\sin\left(2\theta\right)
\end{gather}
At $r =a$:
\begin{gather}
    \sigma_r = 0\\
    \sigma_{\theta} = \sigma_x + \sigma_y - 2\left(\sigma_x - \sigma_y\right)\cos\left(2\theta\right)\\
    \tau_{r\theta} = 0
\end{gather}
$\sigma_{\theta}$ takes a maximum value at $\theta = \dfrac{\pi}{2}$ and $\dfrac{3\pi}{2}$:
\begin{gather}
    \sigma_{\theta} = 3\left(\sigma_x + \sigma_y\right)
\end{gather}
\end{document}