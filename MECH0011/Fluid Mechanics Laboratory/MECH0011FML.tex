\documentclass[11pt]{article}
\usepackage{graphicx}
\usepackage{hyperref}
%\usepackage{appendix}
\usepackage{amsmath}
\usepackage{amsthm}
\usepackage{wrapfig}
\usepackage{amssymb}
\usepackage{float}
\usepackage{commath}
%\usepackage{siunitx}
%\sisetup{detect-all}
\usepackage[a4paper,margin=20mm]{geometry}
\numberwithin{equation}{section}
\setlength{\parskip}{\baselineskip}
\setlength{\parindent}{0pt}
\hypersetup{
    colorlinks=true,
    linkcolor=magenta,
    filecolor=magenta,      
    urlcolor=magenta,
}
\urlstyle{same}
\begin{document}
\title{\textbf{UCL Mechanical Engineering 2020/2021}\\MECH0011 Fluid Mechanics Laboratory\\
Lift of a NACA-0012 Aerofoil}
\author{\textbf{Group 21}\\Haochen Yang, Muhammad Naveed, Yikai Zhang, Ryan El Khoury, Hasha Dar}
\maketitle
\section{Introduction to Lift and Drag}
\subsection{Describe what lift and drag are}
A fluid, for example air in our case, that flows around an object will act a force on it. In our case, this object is the airfoil. Lift is the effect of the force applied by the fluid in the perpendicular direction to the flow direction. THe lift force can be calculated with known pressure distribution.
\begin{align}
  L = \oint \left(p n k \right)\dif S
\end{align}
Where $p$ is the pressure on the surface of the airfoil, $n$ is the normal unit vector pointing into the wing, $k$ is the vertical unit vector, normal to the airfoil direction. Or it can be calculated by using lift coefficient:
\begin{align}
  L = \frac{1}{2} \rho V^2 A C_l
\end{align}
Where $\rho$ is the density of the fluid (air), $V$ is the velocity of the flow and $A$ is the wing area. Drag is also called fluid resistance because it acts like a resistance force. Drag acts opposite to the motion of any object with respect to the fluid that flows around it. Drag can also be calculated as:
\begin{align}
  D = \frac{1}{2} \rho V^2 A C_d 
\end{align}
Where $C_d$ is the drag coefficient.
\begin{figure}[H]
  \centering
  \includegraphics[width = 0.5 \textwidth]{./img/diagram1.png}
  \caption{}
\end{figure}
\subsection{Describe the difference between form and friction drags, and how their contribution differ in streamlined and bluff bodies}
\begin{figure}[H]
  \centering
  \includegraphics[width = 0.5 \textwidth]{./img/diagram1.png}
  \caption{}
\end{figure}
Force generated by pressure can be written as:
\begin{align}
  F = p \dif S
\end{align}
Where $p$ is the pressure on the surface area $\dif S$. Form drag is the normal component of the force normal to the surface. Hence, the form drag can be calculated by integrating the force component in the stream direction iver the whole surface area.
\begin{align}
  D_f = \int_S \left(p\right)\dif S
\end{align}
The skin friction is the tangential component of the force normal to the surface. Hence, skin friction can be calculated by integrating the shear force, $\tau_w \dif S$, over the whole area.
\begin{align}
  F_s = \int_S \left(\tau_w\right)\dif S
\end{align}
\subsection{How do you determine the lift coefficient of a finite wing using the coefficient of an infinite one? explain why this is necessary and mention measures to improve the efficiency of a wing}
The lift coefficient of the finite wing can be calculated with given lift coefficient of a finite wing by using lifting-line theory.
\begin{figure}[H]
  \centering
  \includegraphics[width = 0.35 \textwidth]{./img/diagram3.png}
  \caption{For a given wing with wing span $b$ and chord length $c$}
\end{figure}
\begin{figure}[H]
  \centering
  \includegraphics[width = 0.5 \textwidth]{./img/diagram4.png}
  \caption{}
  \label{GeometricAngleOfAttack}
\end{figure}
As shown in the graph, the geometric angle of attack is $\alpha$ and $\alpha_e$ is the effective geometry of angle of attack. $\alpha_i$ is the induced angle of attack. $V_{\infty}$ is the free stream flow velocity, $W$ is the downwash caused by the flow and $\alpha_e = \alpha - \alpha_i$. 

Also, by applying the \textit{Biot-Savart} Law,  for a given vortex filament with strength $\Gamma$, the velocity of a certain point $P$ which is created by this vortex
\begin{figure}[H]
  \centering
  \includegraphics[width=0.2\textwidth]{./img/diagram5.png}
  \caption{}
\end{figure}
\begin{align}
  \left| \vec{V} \right| &= \frac{\Gamma}{4\pi} \int_{-\infty}^{\infty} \frac{\left| \dif \vec{l} \cdot \vec{r} \right| }{ \left| r \right|^3}\\
  l &= r\cos{\theta}\\
  h &= r\sin{\theta}\\
  \dif l &= -\frac{h}{\sin{\theta}}\\
  \left| \vec{V} \right| &= \frac{\Gamma}{4\pi} \int_{\pi}^0 \left(\frac{-\sin{\theta}}{h}\right) \dif \theta\\
\end{align}
So the downwash at a certain point on the wing will be:
\begin{align}
  W = \frac{\Gamma}{4\pi h} \int_{\frac{\pi}{2}}^0\left(-\sin{\theta}\right)\dif \theta = \frac{\Gamma}{4\pi h}
\end{align}
The lift per unit span:
\begin{align}
  L_v &= \rho V_{\infty} \Gamma = \frac{1}{2}\rho V_{\infty}^2 c C_l = \frac{1}{2} \rho V_{\infty}^2 c \cdot 2\pi \alpha_e\\
  \Gamma &= cV_{\infty}\pi \left(\alpha - \alpha_i\right)\\
  \alpha_i &= \frac{-w\left(y\right)}{V_{\infty}}
\end{align}
However when we are considering the downwash at the wing tips where $h=0$. The angle of attack is approach infinity, but that's impossible. So, we introduce the lifting-line theory. We suggest that $\gamma$ is a function of $y$. But according to \textit{Helmholtz's} first theorem, the strength of a vortex filament is a constant along its length. So, in order to let $\Gamma$ to change $y$ along y, we create an infinitely large number of trailing vortexes which all have slightly different strengths.
\begin{align}
  \dif \Gamma &= \frac{\dif \Gamma}{\dif y} \dif y\\
  \therefore W\left(y_0\right) &= \int_{\frac{-b}{2}}^{\frac{b}{2}} \left(\frac{\frac{\dif \Gamma}{\dif y}}{4\pi \left(y-y_0\right)}\right) \dif y\\
  C_l &= \frac{2\Gamma\left(y_0\right)}{V_{\infty}c\left(y_0\right)} = 2\pi \alpha_e = 2\pi\left(\alpha\left(y_0\right) - \alpha_i \left(y_0\right)\right)\\
  \alpha_i \left( y_0 \right) &= -\frac{w\left( y_0 \right)}{V_{\infty}} = -\frac{1}{4\pi V_{\infty}} \int_{\frac{-b}{2}}^{\frac{b}{2}} \left( \frac{\frac{\dif \Gamma}{\dif y}\dif y}{y - y_0} \right)\\
  \Gamma \left(y\right) &= \Gamma \left(y_0\right) \sqrt{1 - \left(\frac{2y}{b}\right)^2}\\
  \frac{\dif \Gamma}{\dif y} &= -\frac{4\Gamma_0 y}{b^2} \left(\sqrt{1-\left(\frac{2y}{b}\right)^2}\right)^{-1}\\
  \therefore w\left(y_0\right) &= -\frac{\Gamma_0}{2b}
\end{align}
Where - point is the mid point of the wingspan. The lift of the finite wing is:
\begin{align}
  L &= \int_{\frac{-b}{2}}^{\frac{b}{2}} \left(L\left(y\right)\right)\dif y = \rho V_{\infty} \Gamma_0 \int_{\frac{-b}{2}}^{\frac{b}{2}} \sqrt{1-\left(\frac{2y}{b}\right)^2} \dif y = \frac{1}{4}\rho \Gamma_0 b\pi\\
  C_L &= \frac{L}{\frac{1}{2}\rho V_{\infty}^2 S} = \frac{\Gamma_0 b \pi}{2V_{\infty S}}
\end{align}
It is usually hard to calculated the lift force of the whole wing in the three dimensional world. So, it is much more easier just to consider slicing the wing into cross-sections. But we can’t just add up all the independently calculated forces of each cross-sections because this approximation is incorrect. In practice, each cross-section of the wing will influence its neighbouring one. The lifting-line theory correct part of the errors by introducing the interactions between the wing slices. 

To maximize the efficiency of the wing, we must maximize the lift force exerted on the wing surface. To achieve that, we must minimize the downwash at the wing tips. One of the way to do so is to change the wing tip by adding a winglet.
\end{document}