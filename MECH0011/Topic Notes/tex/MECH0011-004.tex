\documentclass[class=report, crop=false, 12pt,a4paper]{standalone}
\usepackage{enumitem}
\usepackage{float}
\usepackage[normalem]{ulem}
\usepackage{graphicx}
\usepackage{amsmath}
\usepackage{amssymb}
\usepackage{siunitx}
\usepackage{commath}
\usepackage{tikz}
\usetikzlibrary{positioning, fit, calc}   
\tikzset{block/.style={draw, thick, text width=3cm ,minimum height=1.3cm, align=center},   
line/.style={-latex}     
}  
\begin{document}
\section{Exercise 2 - Fluid tutorial group B}
A viscous fluid of constant density, $\rho$, and kinematic viscosity, $\nu$, flows over a flat plate inclined at an angle $\alpha$ and moving with a constant velocity $V_w$. The flow is stationary and no pressure gradients are applied. The only body force acting on the fluid is due to gravity, $g$. You can assume zero velocity component in the direction orthogonal to the plate and negligible air resistance at the interface between the viscous fluid and air $(y = h)$.
\end{document}