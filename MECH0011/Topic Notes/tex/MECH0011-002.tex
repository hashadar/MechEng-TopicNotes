\documentclass[class=report, crop=false, 12pt,a4paper]{standalone}
\usepackage{enumitem}
\usepackage{multicol}
\usepackage{graphicx}
\usepackage{float}
\usepackage{amsmath}
\usepackage{amssymb}
\usepackage{mathtools}
\usepackage{siunitx}
\usepackage{commath}
\usepackage{array}
\usepackage{natbib}
\usepackage{tikz}
\usepackage[a4paper,width=150mm,top=25mm,bottom=25mm]{geometry}
\usetikzlibrary{positioning, fit, calc}   
\tikzset{block/.style={draw, thick, text width=3cm ,minimum height=1.3cm, align=center},   
line/.style={-latex}     
}  
\setlength{\parindent}{0pt}
\begin{document}
\begin{center}
    13/10/2020
\end{center}
\section{Constitutive Equations}
We want to find a way to link the stress tensor $\tau$ with the velocity field i.e. $\tau = f(u, \ v, \ w)$.
\begin{figure}[H]
  \centering
  \includegraphics[width = 0.8\textwidth]{../img/diagram1.png}
\end{figure}
The angle of deformation $\Delta \theta$ can be used to derive the following:
\begin{gather}
  \tan{\Delta \theta} = \frac{u \cdot \Delta T}{H}\\
  \tan{\Delta \theta} = \dif \theta = \frac{u\cdot t}{H} \rightarrow \frac{\dif \theta}{\dif t} = \frac{u}{H}\\
  \tau = \frac{F}{A} \propto \frac{\dif \theta}{\dif t} = \frac{u}{H}\\
  \tau = \mu \frac{\dif \theta}{\dif t} = \mu \frac{u}{H}\\
  \tau = \mu \frac{\dif u}{\dif y}
\end{gather}
\begin{itemize}
  \item $\tau$ is the shear stress
  \item $\frac{\dif u}{\dif y}$ is the shear rate
  \item $\mu$ is the dynamic viscosity and has units \si{\newton\second\per\meter\squared}
  \item $\nu = \frac{\mu}{\rho}$ is the kinematic viscosity and has units \si{\meter\squared\per\second}
\end{itemize}
For Newtonian fluids, $\mu$ is constant. In the case above, our stress tensor is $\tau_{yx}$, hence:
\begin{equation}
  \tau_{yx} = \mu \frac{\partial u}{\partial y}
\end{equation}
Our velocity gradient can be defined as:
\begin{equation}
  \nabla \vec{V} = \begin{bmatrix}
    \frac{\partial u}{\partial x} & \frac{\partial v}{\partial x} & \frac{\partial w}{\partial x}\\
    \frac{\partial u}{\partial y} & \frac{\partial v}{\partial y} & \frac{\partial w}{\partial y}\\
    \frac{\partial u}{\partial z} & \frac{\partial v}{\partial z} & \frac{\partial w}{\partial z}
  \end{bmatrix}
\end{equation}
The left diagonal components are the normal deformation, orthogonal to the surface.
\begin{figure}[H]
  \centering
  \includegraphics[width = 0.8\textwidth]{../img/diagram2.png}
\end{figure}
A simplified way of writing these left diagonal terms is
\begin{equation}
  \nabla \cdot \vec{V} = \frac{\partial u_i}{\partial x_i} = \frac{\partial u}{\partial x} + \frac{\partial v}{\partial y} + \frac{\partial w}{\partial z} 
\end{equation}
The repeated index $i$ means sum in the x, y and z directions.
\begin{figure}[H]
  \centering
  \includegraphics[width = 0.5\textwidth]{../img/diagram3.png}
  \caption{$1/3$ symbolises the average deformation in x, y and z.}
\end{figure}
To find $\frac{\partial u}{\partial x}$, we can do the following
\begin{equation}
  \frac{\partial u}{\partial x} = \frac{1}{3}\frac{\partial u_i}{\partial x_i} + \left( \frac{\partial u}{\partial x} - \frac{1}{3}\frac{\partial u_i}{\partial x_i} \right)
\end{equation}
\begin{figure}[H]
  \centering
  \includegraphics[width = \textwidth]{../img/diagram4.png}
\end{figure}
This can be also done for the other two orthogonal directions
\begin{gather}
  \frac{\partial v}{\partial y} = \frac{1}{3}\frac{\partial u_i}{\partial x_i} + \left( \frac{\partial v}{\partial y} - \frac{1}{3}\frac{\partial u_i}{\partial x_i} \right)\\
  \frac{\partial w}{\partial z} = \frac{1}{3}\frac{\partial u_i}{\partial x_i} + \left( \frac{\partial w}{\partial z} - \frac{1}{3}\frac{\partial u_i}{\partial x_i} \right)
\end{gather}
Let us consider another term, such as $\frac{\partial u}{\partial y}$. We can define this as a component of deformation and rotation of the fluid particle.
\begin{equation}
  \frac{\partial u}{\partial y}=\frac{1}{2} \left(\frac{\partial u}{\partial y} + \frac{\partial v}{\partial x}\right) + \left(\frac{\partial u}{\partial y} - \frac{\partial v}{\partial x}\right)
\end{equation}
\begin{figure}[H]
  \centering
  \includegraphics[width = \textwidth]{../img/diagram5.png}
\end{figure}
\subsubsection{Example}
\begin{gather}
  \frac{\partial u}{\partial y}=3=\frac{1}{2} \left(\frac{\partial u}{\partial y} + \frac{\partial v}{\partial x}\right) + \left(\frac{\partial u}{\partial y} - \frac{\partial v}{\partial x}\right) = 2.5 + 0.5\\
  \frac{\partial v}{\partial x}=2=\frac{1}{2} \left(\frac{\partial u}{\partial y} + \frac{\partial v}{\partial x}\right) + \left(\frac{\partial v}{\partial x} - \frac{\partial u}{\partial y}\right) = 2.5 - 0.5
\end{gather}
\begin{figure}[H]
  \centering
  \includegraphics[width = \textwidth]{../img/diagram6.png}
\end{figure}
\begin{figure}[H]
  \centering
  \includegraphics[width = 0.8 \textwidth]{../img/diagram7.png}
\end{figure}
\subsection{Strain Rate Tensor}
\begin{equation}
  s = \begin{bmatrix}
    \frac{\partial u}{\partial x} & \frac{1}{2} \left[ \frac{\partial u}{\partial y} + \frac{1}{2} \frac{\partial v}{\partial x} \right] & \left[ \frac{\partial u}{\partial z} + \frac{\partial w}{\partial x} \right]\\
    \frac{1}{2} \left[ \frac{\partial u}{\partial y} + \frac{\partial v}{\partial x} \right] & \frac{\partial v}{\partial y} & \frac{1}{2} \left[ \frac{\partial v}{\partial z} + \frac{\partial w}{\partial y} \right] \\
    \frac{1}{2} \left[ \frac{\partial u}{\partial z} + \frac{\partial w}{\partial x} \right] & \frac{1}{2} \left[ \frac{\partial v}{\partial z} + \frac{\partial w}{\partial y} \right] & \frac{\partial w}{\partial z}
  \end{bmatrix} =
\end{equation}
\begin{multline}
  \begin{bmatrix}
    \frac{1}{3} \nabla \cdot \vec{V} & 0 & 0\\
    0 & \frac{1}{3} \nabla \cdot \vec{V} & 0\\
    0 & 0 & \frac{1}{3} \nabla \cdot \vec{V}\\ 
  \end{bmatrix} + \\
  \begin{bmatrix}
    \left[\frac{\partial u}{\partial x} - \frac{1}{3} \nabla \cdot \vec{V} \right] & \frac{1}{2} \left[ \frac{\partial u}{\partial y} + \frac{1}{2} \frac{\partial v}{\partial x} \right] & \frac{1}{2} \left[ \frac{\partial u}{\partial z} + \frac{\partial w}{\partial x} \right]\\
    \frac{1}{2} \left[ \frac{\partial u}{\partial y} + \frac{\partial v}{\partial x} \right] & \left[ \frac{\partial v}{\partial y} - \frac{1}{3} \nabla \cdot \vec{V} \right] & \frac{1}{2} \left[ \frac{\partial v}{\partial z} + \frac{\partial w}{\partial y} \right] \\
    \frac{1}{2} \left[ \frac{\partial u}{\partial z} + \frac{\partial w}{\partial x} \right] & \frac{1}{2} \left[ \frac{\partial v}{\partial z} + \frac{\partial w}{\partial y} \right] & \left[ \frac{\partial w}{\partial z} - \frac{1}{3} \nabla \cdot \vec{V} \right]
  \end{bmatrix}
\end{multline}
\begin{multline}
  \textrm{Deformation part which goes in pressure } \rho \ +\\ \textrm{Deformation part which goes in the stress tensor } T
\end{multline}
Compact notation of the strain rate tensor, indices $i, \ j = 1, \ 2,\ 3$
\begin{equation}
  s_{ij} = \frac{1}{2} \left( \frac{\partial u_i}{\partial x_j} + \frac{\partial u_j}{\partial x_i} \right)
\end{equation}
\subsection{Stress Tensor}
\begin{gather}
  T = \begin{bmatrix}
    \tau_xx & \tau_yx & \tau_zx\\
    \tau_xy & \tau_yy & \tau_zy\\
    \tau_xz & \tau_yz & \tau_zz
  \end{bmatrix} = \\ 
  \begin{bmatrix}
    \mu \left[ 2 \frac{\partial u}{\partial x} - \frac{2}{3} \nabla \cdot \vec{V} \right] & \mu \left[ \frac{\partial u}{\partial y} + \frac{1}{2} \frac{\partial v}{\partial x} \right] & \mu \left[ \frac{\partial u}{\partial z} + \frac{\partial w}{\partial x} \right]\\
    \mu \left[ \frac{\partial u}{\partial y} + \frac{\partial v}{\partial x} \right] & \mu\left[ 2\frac{\partial v}{\partial y} - \frac{2}{3}\nabla \cdot \vec{V} \right] & \mu \left[ \frac{\partial v}{\partial z} + \frac{\partial w}{\partial y} \right] \\
    \mu \left[ \frac{\partial u}{\partial z} + \frac{\partial w}{\partial x} \right] & \mu \left[ \frac{\partial v}{\partial z} + \frac{\partial w}{\partial y} \right] & \mu \left[ 2\frac{\partial w}{\partial z} - \frac{2}{3}\nabla \cdot \vec{V} \right]
  \end{bmatrix}
\end{gather}
Compact notation for constitutive equation:
\begin{gather}
  \tau_{ij} = 2\mu \left[ s_{ij} - \frac{1}{3}(\nabla \cdot \vec{V}) \delta_{ij} \right] = \mu \left[ \left(\frac{\partial u_i}{\partial x_j} + \frac{\partial u_j}{\partial x_i} \right) - \frac{2}{3}(\nabla \cdot \vec{V}) \delta_{ij} \right]\\
  \delta_{ij} = \begin{cases}
    1 \textrm{ if } i = j\\
    0 \textrm{ if } i \neq j
  \end{cases} 
\end{gather}
The stress tensor is always symmetric along the left diagonal.
\begin{gather}
  \tau_{xz} = \tau_{zx} = \frac{1}{2} \left(\frac{\partial u}{\partial z} + \frac{\partial w}{\partial x} \right)
\end{gather}
\begin{figure}[H]
  \centering
  \includegraphics[width = 0.8\textwidth]{../img/diagram8.png}
\end{figure}
We now have 10/12 equations to describe the fluid. The final two relate to the temperature and the energy of the fluid. We will not be considering the energies of the fluid in this course. Our state equation can be $p = \rho RT$, when the fluid is compressible and if our fluid is incompressible we take $\rho$ as constant. All in all, 11 variables and 11 equations to describe the fluid
\section{Navier-Stokes Equations}
Navier-Stokes equations are a system of equations that can be used to describe the behaviour of a fluid. They can be obtained through inserting the Constitutive Equations into the Conservation of Momentum Equations, rearranging, and simplifying them. The Navier-Stokes Equations for an incompressible fluid in 3D are as follows:
\subsubsection{Conservation of Mass:}
\begin{gather}
  \nabla \cdot \vec{V} = \frac{\partial u}{\partial x} + \frac{\partial v}{\partial y} + \frac{\partial w}{\partial z} = 0
\end{gather}
\subsubsection{Conservation of Momentum $(x,y,z)$:}
\begin{gather}
  \rho \left(\frac{\partial u}{\partial t} + u\frac{\partial u}{\partial x} + v\frac{\partial u}{\partial y} + w\frac{\partial u}{\partial z} \right) = -\frac{\partial p}{\partial x} + \mu \left(\frac{\partial^2 u}{\partial x^2} + \frac{\partial^2 u}{\partial y^2} + \frac{\partial^2 u}{\partial z^2} \right) \\
  \rho \left(\frac{\partial v}{\partial t} + u\frac{\partial v}{\partial x} + v\frac{\partial v}{\partial y} + w\frac{\partial v}{\partial z} \right) = -\frac{\partial p}{\partial y} + \mu \left(\frac{\partial^2 v}{\partial x^2} + \frac{\partial^2 v}{\partial y^2} + \frac{\partial^2 v}{\partial z^2} \right) \\
  \rho \left(\frac{\partial w}{\partial t} + u\frac{\partial w}{\partial x} + v\frac{\partial w}{\partial y} + w\frac{\partial w}{\partial z} \right) = -\frac{\partial p}{\partial z} + \mu \left(\frac{\partial^2 w}{\partial x^2} + \frac{\partial^2 w}{\partial y^2} + \frac{\partial^2 w}{\partial z^2} \right) - \rho g
\end{gather}
The Navier-Stokes Equations can be further simplified if the following occur:
\begin{itemize}
  \item Constant Density
  \item Assume Steady Flow (No Time-Dependent Terms)
  \item Assume No External Forces
  \item Assume Fluid is Incompressible
\end{itemize}
The Navier-Stokes Equations in 2D, with the above assumptions applied are below:
\subsubsection{Conservation of Mass (2D):}
\begin{gather}
  \nabla \cdot \vec{V} = \frac{\partial u}{\partial x} + \frac{\partial v}{\partial y} = 0
\end{gather}
\subsubsection{Conservation of Momentum $(x,y)$:}
\begin{gather}
  \rho \left(u\frac{\partial u}{\partial x} + v\frac{\partial u}{\partial y} \right) = -\frac{\partial p}{\partial x} + \mu \left(\frac{\partial^2 u}{\partial x^2} + \frac{\partial^2 u}{\partial y^2} \right) \\
  \rho \left(u\frac{\partial v}{\partial x} + v\frac{\partial v}{\partial y} \right) = -\frac{\partial p}{\partial y} + \mu \left(\frac{\partial^2 v}{\partial x^2} + \frac{\partial^2 v}{\partial y^2} \right) 
\end{gather}
\section{Lagrangian vs. Eulerian}
\begin{gather}
  \frac{\Dif}{\Dif t} = \frac{\partial}{\partial t} + \left(u\frac{\partial}{\partial x} + v\frac{\partial}{\partial y} + w\frac{\partial}{\partial z} \right)
\end{gather}
\begin{itemize}
  \item $\frac{\Dif}{\Dif t}$ - Lagrangian/Material Derivative: Variation in time of a property (for example temperature, density or velocity component) of a fluid particle. The reference system is moving with the fluid particle. 
  \item $\frac{\partial}{\partial t}$ - Eulerian Derivative: Variation in time of a property (for example temperature, density etc..) in a fixed point in space (x, y , z). Reference system fixed in space. 
  \item $u\frac{\partial}{\partial x} + v\frac{\partial}{\partial y} + w\frac{\partial}{\partial z}$ - Convection Terms in the $x$, $y$ and $z$: Variation of a property due to how the particle is moving in space.
\end{itemize}
\end{document}