\documentclass[class=report, crop=false, 12pt,a4paper]{standalone}
\usepackage{enumitem}
\usepackage{float}
\usepackage[normalem]{ulem}
\usepackage{graphicx}
\usepackage{amsmath}
\usepackage{siunitx}
\usepackage{commath}
\usepackage{tikz}
\usetikzlibrary{positioning, fit, calc}   
\tikzset{block/.style={draw, thick, text width=3cm ,minimum height=1.3cm, align=center},   
line/.style={-latex}     
}  
\begin{document}
\section{Tutorial Week 1: Continuity Equation}
\subsection{Exercise 2}
The system is steady due to the fact that our velocity field has no terms in $t$, meaning that our flow does not change with time - steady flow.

The fluid is compressible

\begin{gather}
  v = 4\ln{y} - 2y + 10\\
  \textrm{Volume dilatation} \rightarrow \frac{\partial v}{\partial y} = \frac{4}{y} - 2\\
  \textrm{Variation in the Volume dilatation} \rightarrow \frac{\partial ^2 v}{\partial y^2} = -\frac{4}{y^2}
\end{gather}
For the range $1 < y < 4$, the variation in the volume dilatation is negative, hence our fluid is compressible. 
\begin{gather}
  (0, \ 1) \rightarrow \frac{4}{1} - 2 = 2 \ \si{\per\second}\\
  (0, \ 3) \rightarrow \frac{4}{3} - 2 = -\frac{2}{3} \ \si{\per\second}
\end{gather}
\end{document}