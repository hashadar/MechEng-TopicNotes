\documentclass[class=report, crop=false, 12pt,a4paper]{standalone}
\usepackage{enumitem}
\usepackage{float}
\usepackage[normalem]{ulem}
\usepackage{graphicx}
\usepackage{amsmath}
\usepackage{amssymb}
\usepackage{siunitx}
\usepackage{commath}
\usepackage{tikz}
\usetikzlibrary{positioning, fit, calc}   
\tikzset{block/.style={draw, thick, text width=3cm ,minimum height=1.3cm, align=center},   
line/.style={-latex}     
}  
\begin{document}
\section{Constitutive equations}
We want to find a way to link the stress tensor $\tau$ with the velocity field i.e. $\tau = f(u, \ v, \ w)$.
\begin{figure}[H]
  \centering
  \includegraphics[width = 0.8\textwidth]{../img/diagram1.png}
\end{figure}
The angle of deformation $\Delta \theta$ can be used to derive the following:
\begin{gather}
  \tan{\Delta \theta} = \frac{u \cdot \Delta T}{H}\\
  \tan{\Delta \theta} = \dif \theta = \frac{u\cdot t}{H} \rightarrow \frac{\dif \theta}{\dif t} = \frac{u}{H}\\
  \tau = \frac{F}{A} \propto \frac{\dif \theta}{\dif t} = \frac{u}{H}\\
  \tau = \mu \frac{\dif \theta}{\dif t} = \mu \frac{u}{H}\\
  \tau = \mu \frac{\dif u}{\dif y}
\end{gather}
\begin{itemize}
  \item $\tau$ is the shear stress
  \item $\frac{\dif u}{\dif y}$ is the shear rate
  \item $\mu$ is the dynamic viscosity and has units \si{\newton\second\per\meter\squared}
  \item $\nu = \frac{\mu}{\rho}$ is the kinematic viscosity and has units \si{\meter\squared\per\second}
\end{itemize}
For Newtonian fluids, $\mu$ is constant. In the case above, our stress tensor is $\tau_{yx}$, hence:
\begin{equation}
  \tau_{yx} = \mu \frac{\partial u}{\partial y}
\end{equation}
Our velocity gradient can be defined as:
\begin{equation}
  \nabla \vec{V} = \begin{bmatrix}
    \frac{\partial u}{\partial x} & \frac{\partial v}{\partial x} & \frac{\partial w}{\partial x}\\
    \frac{\partial u}{\partial y} & \frac{\partial v}{\partial y} & \frac{\partial w}{\partial y}\\
    \frac{\partial u}{\partial z} & \frac{\partial v}{\partial z} & \frac{\partial w}{\partial z}
  \end{bmatrix}
\end{equation}
The left diagonal components are the normal deformation, orthogonal to the surface.
\begin{figure}[H]
  \centering
  \includegraphics[width = 0.8\textwidth]{../img/diagram2.png}
\end{figure}
A simplified way of writing these left diagonal terms is
\begin{equation}
  \nabla \cdot \vec{V} = \frac{\partial u_i}{\partial x_i} = \frac{\partial u}{\partial x} + \frac{\partial v}{\partial y} + \frac{\partial w}{\partial z} 
\end{equation}
The repeated index $i$ means sum in the x, y and z directions.
\begin{figure}[H]
  \centering
  \includegraphics[width = 0.5\textwidth]{../img/diagram3.png}
  \caption{$1/3$ symbolises the average deformation in x, y and z.}
\end{figure}
To find $\frac{\partial u}{\partial x}$, we can do the following
\begin{equation}
  \frac{\partial u}{\partial x} = \frac{1}{3}\frac{\partial u_i}{\partial x_i} + \left( \frac{\partial u}{\partial x} - \frac{1}{3}\frac{\partial u_i}{\partial x_i} \right)
\end{equation}
\begin{figure}[H]
  \centering
  \includegraphics[width = \textwidth]{../img/diagram4.png}
\end{figure}
This can be also done for the other two orthogonal directions
\begin{gather}
  \frac{\partial v}{\partial y} = \frac{1}{3}\frac{\partial u_i}{\partial x_i} + \left( \frac{\partial v}{\partial y} - \frac{1}{3}\frac{\partial u_i}{\partial x_i} \right)\\
  \frac{\partial w}{\partial z} = \frac{1}{3}\frac{\partial u_i}{\partial x_i} + \left( \frac{\partial w}{\partial z} - \frac{1}{3}\frac{\partial u_i}{\partial x_i} \right)
\end{gather}
Let us consider another term, such as $\frac{\partial u}{\partial y}$. We can define this as a component of deformation and rotation of the fluid particle.
\begin{equation}
  \frac{\partial u}{\partial y}=\frac{1}{2} \left(\frac{\partial u}{\partial y} + \frac{\partial v}{\partial x}\right) + \left(\frac{\partial u}{\partial y} - \frac{\partial v}{\partial x}\right)
\end{equation}
\begin{figure}[H]
  \centering
  \includegraphics[width = \textwidth]{../img/diagram5.png}
\end{figure}
\subsubsection{Example}
\begin{gather}
  \frac{\partial u}{\partial y}=3=\frac{1}{2} \left(\frac{\partial u}{\partial y} + \frac{\partial v}{\partial x}\right) + \left(\frac{\partial u}{\partial y} - \frac{\partial v}{\partial x}\right) = 2.5 + 0.5\\
  \frac{\partial v}{\partial x}=2=\frac{1}{2} \left(\frac{\partial u}{\partial y} + \frac{\partial v}{\partial x}\right) + \left(\frac{\partial v}{\partial x} - \frac{\partial u}{\partial y}\right) = 2.5 - 0.5
\end{gather}
\begin{figure}[H]
  \centering
  \includegraphics[width = \textwidth]{../img/diagram6.png}
\end{figure}
\end{document}