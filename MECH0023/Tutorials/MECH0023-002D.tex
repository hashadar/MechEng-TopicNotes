\documentclass[11pt]{article}
\usepackage{enumitem}
\usepackage{multicol}
\usepackage{graphicx}
\usepackage{float}
\usepackage{amsmath}
\usepackage{amssymb}
\usepackage{mathtools}
\usepackage{siunitx}
\usepackage{commath}
\usepackage{array}
\usepackage{natbib}
\usepackage[a4paper,width=150mm,top=25mm,bottom=25mm]{geometry}
\setlength{\parindent}{0pt}
\begin{document}
\title{\textbf{UCL Mechanical Engineering 2021/2022}\\MECH0023 Dynamics Tutorial}
\author{HD}
\date{21-10-12}
\maketitle
\section{What is the meaning of the complex number in the below equation? Do we have a negative natural frequency?}
\begin{equation}
    x(t) = a_1 e^{i\omega_n t} + a_2 e^{-i\omega_n t}
\end{equation}
\section{Solving differential equation}
\begin{equation}
    10\ddot{x} + 1000x = 0
\end{equation}
Boundary conditions:
\begin{equation}
    x_0 = \SI{0.01}{\meter} \textrm{ and } \dot{x}_0 = 0
\end{equation}
Let:
\begin{align}
    x        & = e^{mt}     \\
    \dot{x}  & = me^{mt}    \\
    \ddot{x} & = m^2 e^{mt}
\end{align}
Substituting:
\begin{align}
    e^{mt}\left(10m^2 + 1000\right) & =0 \\
    m = \pm 10i
\end{align}
General solution:
\begin{align}
    x       & = a_1 e^{10it} + a_2 e^{-10it}                             \\
    x       & = c_1 \cos \left(10t\right) + c_2 \sin\left(10t\right)     \\
    \dot{x} & = -10c_1 \sin\left(10t\right) + 10c_2 \cos\left(10t\right)
\end{align}
Inputting boundary conditions:
\begin{align}
    10  & = c_1 \cos \left(0\right) + c_2 \sin\left(0\right)        \\
    c_1 & = 10                                                      \\
    0   & = -10(0.01) \sin\left(0\right) + 10c_2 \cos\left(0\right) \\
    c_2 & = 0
\end{align}
Therefore:
\begin{equation}
    x = 10\cos\left(10t\right) \, (\si{\milli\meter})
\end{equation}
\section{Discussion on meaning of complex numbers in the polar form}
Complex numbers represent a means to solving differential equations (in the polar form). Hence, we can use complex numbers to solve differential equations to describe oscillatory behaviour. The complex numbers do not have an attributed physical characteristic and we can see that our solutions have only real components. Our natural frequency is both positive and negative, but negative natural frequency is simply the result of our mathematics and does not have any physical meaning.
\end{document}