\documentclass[11pt]{article}
\usepackage{enumitem}
\usepackage{multicol}
\usepackage{graphicx}
\usepackage{float}
\usepackage{amsmath}
\usepackage{amssymb}
\usepackage{mathtools}
\usepackage{siunitx}
\usepackage{commath}
\usepackage{array}
\usepackage{natbib}
\usepackage[a4paper,width=150mm,top=25mm,bottom=25mm]{geometry}
\setlength{\parindent}{0pt}
\begin{document}
\title{\textbf{UCL Mechanical Engineering 2021/2022}\\MECH0023 Dynamics Tutorial}
\author{HD}
\date{21-10-19}
\maketitle
\section*{Question 1}
8 oscillations, initial displacement \SI{6}{\centi\meter}, boundary condition: human hand.
\begin{gather}
    L \approx \SI{20}{\centi\meter}\\
    \delta = 16\pi \zeta\\
    \ln 2 = 16\pi\zeta\\
    \zeta = 0.0137
\end{gather}
\section*{Question 2}
Brandon counting oscillations also yielded 8 periods to half the amplitude
\section*{Question 3}
\begin{gather}
    \textrm{Period} = \frac{4.85}{8} = \SI{0.61}{\second}\\
    \omega_n = \frac{2\pi}{0.61} = \SI{10.36}{\radian \per \second}\\
    \omega_d = \omega_n \sqrt{1-\zeta^2} = \SI{10.27}{\radian \per \second}\\
    x(t) = X_0 e^{-\zeta \omega_n t}\sin\left(\omega_d t + \varphi\right)\\
    x(t) = 6 e^{-0.0137 \cdot 10.36 t}\sin\left(10.27 t + \varphi\right)
\end{gather}
Source of damping is dependent on a variety of sources. We can see that the ruler is elastic and energy is lost when the ruler elastically snaps back to its original shape. We also see that the ruler may bounce on the table and energy may dissipate throughout the whole length of the ruler; essentially the ruler may start to act about a pivot and we could feel the ruler trying to push up against our hand. This dissipation of energy through the material is called Hysteric damping. The ruler also displaces air and we can see that there is an element of drag against the motion of the ruler, so there is viscous damping in this case too.
\end{document}