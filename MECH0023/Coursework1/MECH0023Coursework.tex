\documentclass[11pt]{article}
\usepackage{graphicx}
\usepackage{hyperref}
\usepackage{appendix}
\usepackage{amsmath}
\usepackage{amsthm}
\usepackage{amssymb}
\usepackage{float}
\usepackage{multirow}
\usepackage{commath}
\usepackage{booktabs}
\usepackage{subcaption}
\renewcommand{\arraystretch}{1.2}
\usepackage{siunitx}
\sisetup{detect-all}
\usepackage{listings}
\usepackage{color} %red, green, blue, yellow, cyan, magenta, black, white
\definecolor{mygreen}{RGB}{28,172,0} % color values Red, Green, Blue
\definecolor{mylilas}{RGB}{170,55,241}
\usepackage[a4paper,margin=20mm]{geometry}
\numberwithin{equation}{section}
\setlength{\parskip}{\baselineskip}
\setlength{\parindent}{0pt}
\hypersetup{
    colorlinks=true,
    linkcolor=black,
    filecolor=black,      
    urlcolor=black,
    citecolor=black
}
\urlstyle{same}
\begin{document}
\title{\textbf{UCL Mechanical Engineering 2021/2022}\\MECH0023 Coursework}
\author{Anonymous}
\date{\today}
\maketitle
\tableofcontents
\listoffigures
\section{Satellite model}
\subsection{Description of model and assumptions}
\subsection{Minimum thickness of beam}
\subsection{Amplitude of displacement for steady-state oscillation of the sensor during rocket function}
\subsection{Limitations of analysis}
\subsection{Manoeuvre}
\subsubsection{Description of obtaining sensor response to the thrust force}
\subsubsection{Dispalcement response}
\section{Linear modelling}
\subsection{Description of non-linearity $y(t)$ \& $\theta(t)$}
The following function:
\begin{equation}
    \frac{\dif y\left( t\right)}{\dif t} = 150 \cos \left(\theta\left(t\right)\right) - 47
\end{equation}
has a non-linear relationship due to $\theta(t)$ being within a cosine function. There is also an offset value of $47$ which makes our relationship non-linear as well.
\subsection{Description of non-linear relationship in a real system}
An example of a non-linear relationship is car velocity and air resistance. Our input is the velocity of the car and our output is the drag force We can describe this relationship with the following equation.
\begin{equation}
    F_D = \frac{1}{2} \rho v^2 C_D A
\end{equation}
The relationship between the drag force and velocity is proportionally squared. This means that as velocity increases, our drag force will increase by four times. Non-linearity can also come in the form as headwind or tailwind. Mathematically, this would be represented as a constant term added to our equation. For example:
\begin{equation}
    F_D = \frac{1}{2} \rho v^2 C_D A + 100
\end{equation}
\section{Linear model performance and stability}
\subsection{Description of actuator switching on}
\subsection{Root locus and effect of proportional control}
\subsection{Negative velocity feedback controller}
\subsubsection{Plot of root locus}
\subsubsection{Areas where damping ratio is higher than 0.6}
\subsubsection{Range of $K$ where performance specifications are achieved}
\section{Controller techniques}
\subsection{Description of closed loop position control system}
\subsection{Specify disturbances}
\section{Inverted pendulum}
\subsection{Identification and description of assumptions and limitations for the model of a rotary inverted pendulum}
\subsection{Moving rocket to final position}
\section{Multi-Degree-of-Freedom systems}
\subsection{Example of a system with multiple degrees-of-freedom}
\end{document}