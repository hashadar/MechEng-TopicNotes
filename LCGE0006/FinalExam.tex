\documentclass[11pt]{article}
\usepackage{graphicx}
\usepackage{hyperref}
\usepackage{appendix}
\usepackage{amsmath}
\usepackage{amsthm}
\usepackage{amssymb}
\usepackage{float}
\usepackage{commath}
\usepackage{booktabs}
\renewcommand{\arraystretch}{1.2}
\usepackage{siunitx}
\sisetup{detect-all}
\usepackage{listings}
\usepackage{color} %red, green, blue, yellow, cyan, magenta, black, white
\definecolor{mygreen}{RGB}{28,172,0} % color values Red, Green, Blue
\definecolor{mylilas}{RGB}{170,55,241}
\usepackage[a4paper,margin=20mm]{geometry}
\numberwithin{equation}{section}
\setlength{\parskip}{\baselineskip}
\setlength{\parindent}{0pt}
\hypersetup{
    colorlinks=true,
    linkcolor=black,
    filecolor=black,      
    urlcolor=black,
    citecolor=black
}
\urlstyle{same}
\begin{document}
\title{\textbf{LCGE0006: German Level 5: Business and Current Affairs, Level 6, Undergraduate, 2020}}
\author{NCWT3}
\maketitle
\section*{Task 1}
Meiner Meinung nach sind die Vor- und Nachteile, die sich bei der Arbeit für die verschiedenen Organisationstypen ergeben können, subjektiv. Man muss zuerst die eigene Persönlichkeit, die Lebensziele und die Prioritäten berücksichtigen. Es kann gesagt werden, dass bestimmten Organisationstypen und Rollen innerhalb der Organisation für verschiedenen Personen attraktiv sind. All dies kann auf die Denkweise eines Individuums zurückgeführt werden, und darauf werde ich hier näher eingehen.

Natürlich hat die Arbeit für ein großes und etabliertes Unternehmen wie BMW viele Vergünstigungen. Der Job ist sicher und das Gehalt großartig. Zum Beispiel kann Andreas Rauth vielleicht besondere Vergünstigungen wie einen BMW-Firmenwagen genießen. Dies kann jedoch zu Lasten des beruflichen Aufstiegs gehen. BMW beschäftigt Zehntausende von Menschen und die Möglichkeit, die Karriereleiter zu erklimmen, erfordert extrem Kompetenz, das Kenntnis der richtigen Leute und Glück. Man kann also davon ausgehen, dass Rauth noch lange als Fertigungsmeister für BMW arbeiten würde. Aber man kann auch voraussehen, dass man bei der Arbeit in einem etablierten Unternehmen viele anderen Möglichkeiten hat, zum Beispiel Lösungen zur Optimierung und Rationalisierung der Produkte des Unternehmens.

Im Vergleich dazu ist ein Start-up ganz anders. Die Arbeitsplatzsicherheit ist häufig ein Problem, da das Unternehmen eine höhere Ausfallrisiko hat. Von den Mitarbeitern kann auch erwartet werden, dass sie weniger und länger arbeiten. Weil der Cashflow für viele Start-ups ein Problem darstellt, gibt es für einen Mitarbeiter häufig keinen zusätzlichen Vergünstigungen. Was macht Start-ups für Menschen attraktiv? Aufgrund meiner eigenen Forschung und mein Engagement für Start-ups in der Automobilbranche sind sie viel experimenteller und entwickeln Spitzentechnologie. Die Beziehung zwischen den Mitarbeitern ist auch persönlicher. Im Gegensatz zu den traditionelleren deutschen Automobilgiganten mit ihren bewährten Designs und großen Büroteams kann die Arbeit für einen Einzelnen viel ansprechender und erfüllender sein, als nur ein Zahnrad in einer Maschine zu sein. Ein weiterer motivierender Faktor könnte sein, dass Erfolg oft eine hohe Belohnung bringt. Beispielsweise sind Facebook und Google äußerst erfolgreiche und inspirierende Milliarden Dollar Start-ups.

Man kann die Vor- und Nachteile jedes Organisationstyps folgendermaßen zusammenfassen. Etablierte große Unternehmen eignen sich möglicherweise besser für Personen mit geringerem Risikoappetit, die beispielsweise einen stabilen Einkommen benötigen, um eine Familie zu ernähren. Start-ups können persönlich erfüllender sein und für diejenigen, die mehr Risiko eingehen möchten. Ich würde persönlich lieber für einen Start-up arbeiten, weil ich gerne an Herausforderungen arbeite und stehe an der Spitze der Innovation! Ich mag auch die Freiheit, die Start-ups gewähren, um meinen Leidenschaften zu folgen.
\newpage
\section*{Task 2}
Der Deutschkurs wurde in diesem Jahr unter dem gegebenen Umstand außergewöhnlich gut durchgeführt. Der Inhalt des Kurses und die Klassen waren besonders ansprechend und mit einem relevanten Schwerpunkt. Der Kurs hat zu meinen Kenntnissen beigetragen, indem er Themen wie den deutschen Arbeitsplatz erweitert hat. Es hat auch die Grammatik und das Vokabular verfeinert und ausgebaut, die notwendig sind, um ein flüssigeres Niveau zu erreichen. Besonderes gut hat mir das Projekt gefallen, da es auf meinen Forschungs- und Analysefähigkeiten zu einem für mich interessant Thema aufgebaut hat. Alles in allem war der Kurs gut, obwohl ich keinen Fuß in ein Klassenzimmer gesetzt habe!

Der Kurs hat auch übertragbaren Fähigkeiten entwickelt. Die größte Fähigkeit, die ich aus dem Kurs gelernt habe, ist wie man sich der Sprachenlernen nähert. Als begeisterter Sprachlerner (der derzeit Türkisch und Japanisch lernt!) hatte ich immer Problemen damit, selbstlernend Sprachen zu lernen und geeigneten Ressourcen zu finden. Ich habe darüber nachgedacht, wie der Kurs Schlüsselkonzepte vermittelt hat und ich habe diese Fähigkeiten angewendet, um andere Sprachen mit Erfolg zu lernen! Ich finde, dass das Arbeiten aus einem Lehrbuche viel hilfreicher ist als das Lernen aus einer App. Lehrbücher und Unterricht sind für mich persönlich ein viel besseres Unterrichtsmedium.

Ich habe mich aus zwei Hauptgründen dafür entschieden, Deutsch an der Universität fortzusetzen. Deutsch lernen öffnet Türen zur Beschäftigung im deutschsprachigen Raum. Dies ist besonders attraktiv für mich, weil ich als Ingenieur in der Automobilbranche arbeiten will, vielleicht irgendwo bei Mercedes-Benz oder Porsche. Der Kurs hat mich zuversichtlich gemacht, mich für Praktika in Deutschland zu bewerben. Ich habe mich auch für Deutsch entschieden, um weiter Sprachen zu lernen. Deutsch ist derzeit meine dritte Sprache und ich möchte noch mehr hinzufügen! 
\end{document}