\documentclass[11pt]{article}
\usepackage{graphicx}
\usepackage{hyperref}
%\usepackage{appendix}
\usepackage{amsmath}
\usepackage{amsthm}
\usepackage{amssymb}
\usepackage{float}
\usepackage{commath}
\usepackage{booktabs}
\renewcommand{\arraystretch}{1.2}
\usepackage{siunitx}
\sisetup{detect-all}
\usepackage{listings}
\usepackage{color} %red, green, blue, yellow, cyan, magenta, black, white
\definecolor{mygreen}{RGB}{28,172,0} % color values Red, Green, Blue
\definecolor{mylilas}{RGB}{170,55,241}
\usepackage[a4paper,margin=20mm]{geometry}
\numberwithin{equation}{section}
\setlength{\parskip}{\baselineskip}
\setlength{\parindent}{0pt}
\hypersetup{
    colorlinks=true,
    linkcolor=black,
    filecolor=black,      
    urlcolor=black,
    citecolor=black
}
\urlstyle{same}
\begin{document}
\title{\textbf{UCL Mechanical Engineering 2020/2021}\\ENGF0004 48-hour Project}
\author{NCWT3}
\maketitle
\tableofcontents
\listoffigures
\section{PDEs, Matrix applications}
\subsection{Developing mathematical model}
\subsubsection{E1}
Starting with:
\begin{equation}
    \left(S_{t+\Delta t} - S_t\right) = \left(vS\right)_x\Delta t - \left(vS\right)_{x+\Delta x} \Delta t-gpS\Delta x \Delta t
\end{equation}
Dividing by $\Delta x \Delta t$:
\begin{equation}
    \frac{\left(S_{t+\Delta t} - S_t\right)\Delta x }{\Delta x \Delta t} = \frac{\left(vS\right)_x\Delta t}{\Delta x \Delta t} - \frac{\left(vS\right)_{x+\Delta x}\Delta t}{\Delta x\Delta t} - \frac{gpS\Delta x\Delta t}{\Delta x \Delta t }
\end{equation}
Simplifying:
\begin{gather}
    \frac{S_{t+\Delta t} - S_t}{\Delta t} = \frac{\left(vS\right)_x}{\Delta x} - \frac{\left(vS\right)_{x+\Delta x}}{\Delta x} - gpS\\
    \frac{S_{t+\Delta t} - S_t}{\Delta t} = \frac{\left(vS\right)_x - \left(vS\right)_{x+\Delta x}}{\Delta x} - gpS\\
    \frac{S_{t+\Delta t} - S_t}{\Delta t}+\frac{\left(vS\right)_{x+\Delta x} - \left(vS\right)_x}{\Delta x} + gpS = 0
\end{gather}
Applying our limits:
\begin{gather}
    \lim_{\Delta x \rightarrow 0, \; \Delta t \rightarrow 0} \left[\frac{S_{t+\Delta t} - S_t}{\Delta t}+\frac{\left(vS\right)_{x+\Delta x} - \left(vS\right)_x}{\Delta x} + gpS \right]= 0 \label{eq:1.1a}
\end{gather}
We can see that in the first two terms of \ref{eq:1.1a}, we have the definition of a derivative by first principles. Hence:
\begin{gather}
    \frac{\partial S}{\partial t} + \frac{\partial \left(vS\right)}{\partial x} + gpS = 0
\end{gather}
\subsubsection{E2}
Starting with:
\begin{equation}
    \frac{\rho S \Delta x \left(\Delta v\right)}{\Delta t} = \left(pS\right)_x - \left(pS\right)_{x+\Delta x} - vrS\Delta x
\end{equation}
Dividing by $\Delta x$:
\begin{gather}
    \frac{\rho S\Delta x\left(\Delta v\right)}{\Delta x \Delta t} = \frac{\left(pS\right)_x}{\Delta x} - \frac{\left(pS\right)_{x+\Delta x}}{\Delta x} - \frac{vrS\Delta x}{\Delta x}
\end{gather}
Simplifying:
\begin{gather}
    \rho S\frac{\Delta v}{\Delta t} = - \frac{\left(pS\right)_{x+\Delta x} - \left(pS\right)_x}{\Delta x} - vrS
\end{gather}
Applying our limits:
\begin{equation}
    \lim_{\Delta x \rightarrow 0, \; \Delta t \rightarrow 0} \left[ \rho S\frac{\Delta v}{\Delta t}\right] = \lim_{\Delta x \rightarrow 0, \; \Delta t \rightarrow 0} \left[ - \frac{\left(pS\right)_{x+\Delta x} - \left(pS\right)_x}{\Delta x} - vrS \right] \label{eq:1.1b}
\end{equation}
We can see that in the first two terms of \ref{eq:1.1b}, we have the definition of a derivative by first principles. Hence:
\begin{equation}
    \rho S \frac{\partial v}{\partial t} = - \frac{\partial \left(pS\right)}{\partial x} - vrS
\end{equation}
\subsubsection{E3 \& E4}
We know that:
\begin{equation}
    c = \frac{1}{S}\frac{\dif S}{\dif p} \label{eq:1.1c}
\end{equation}
Given that $S$ is only a function of the pressure $p$ and $p$ is a function of space and time, we can rewrite \ref{eq:1.1c} as:
\begin{equation}
    c = \frac{1}{S}\frac{\partial S}{\partial p}
\end{equation}
Starting with:
\begin{equation}
    \frac{\partial S}{\partial t} + \frac{\partial \left(vS\right)}{\partial x} + gpS = 0
\end{equation}
Using product rule on second term:
\begin{equation}
    \frac{\partial S}{\partial t} + v\frac{\partial S}{\partial x} + S \frac{\partial v}{\partial x} + gpS = 0
\end{equation}
Dividing by $S$:
\begin{equation}
    \frac{1}{S}\frac{\partial S}{\partial t} + \frac{v}{S} \frac{\partial S}{\partial x} + \frac{\partial v}{\partial x} + gp = 0
\end{equation}
Multiplying the first and second term by "1":
\begin{equation}
    \frac{1}{S}\frac{\partial S}{\partial t}\frac{\partial p}{\partial p} + \frac{v}{S} \frac{\partial S}{\partial x}\frac{\partial p}{\partial p} + \frac{\partial v}{\partial x} + gp = 0
\end{equation}
Rearranging:
\begin{equation}
    \frac{1}{S}\frac{\partial S}{\partial p}\frac{\partial p}{\partial t} + \frac{v}{S} \frac{\partial S}{\partial p}\frac{\partial p}{\partial x} + \frac{\partial v}{\partial x} + gp = 0
\end{equation}
Substituting $c$:
\begin{equation}
    c \frac{\partial p}{\partial t} + cv\frac{\partial p}{\partial x} + \frac{\partial v}{\partial x} + gp = 0
\end{equation}
Repeating again with:
\begin{equation}
    \rho S \frac{\partial v}{\partial t} = - \frac{\partial \left(pS\right)}{\partial x} - vrS
\end{equation}
Using product rule on second term:
\begin{equation}
    \rho S \frac{\partial v}{\partial t} = - p\frac{\partial S}{\partial x} - S \frac{\partial p}{\partial x} - vrS
\end{equation}
Dividing by $S$:
\begin{equation}
    \rho \frac{\partial v}{\partial t} = - \frac{p}{S}\frac{\partial S}{\partial x} - \frac{\partial p}{\partial x} - vr
\end{equation}
Multiplying second term by "1":
\begin{equation}
    \rho \frac{\partial v}{\partial t} = - \frac{p}{S}\frac{\partial S}{\partial x}\frac{\partial p}{\partial p} - \frac{\partial p}{\partial x} - vr
\end{equation}
Rearranging:
\begin{equation}
    \rho \frac{\partial v}{\partial t} + \frac{p}{S}\frac{\partial S}{\partial p}\frac{\partial p}{\partial x} + \frac{\partial p}{\partial x} + vr = 0
\end{equation}
Substituting $c$:
\begin{equation}
    \rho \frac{\partial v}{\partial t} + cp\frac{\partial p}{\partial x} + \frac{\partial p}{\partial x} + vr = 0
\end{equation}
\subsection{Assumption 4}
\subsubsection{Constant distensibility}
Starting with:
\begin{align}
    c\frac{\partial p}{\partial t} &= - \frac{\partial v}{\partial x} \label{eq:1.2a}\\
    \rho \frac{\partial v}{\partial t} &= -\frac{\partial p }{\partial x} \label{eq:1.2b}
\end{align}
Differentiating \ref{eq:1.2a} with respect to $x$ and \ref{eq:1.2b} with respect to $y$:
\begin{align}
    c \frac{\partial^2 p}{\partial x \partial t} = - \frac{\partial^2 v}{\partial x^2}\\
    \rho \frac{\partial^2 v }{\partial t^2} = - \frac{\partial^2 p}{\partial x \partial t}
\end{align}
Substituting:
\begin{align}
    c \left(-\rho \frac{\partial^2 v }{\partial t^2}\right) &= - \frac{\partial^2 v}{\partial x^2}\\
    c\rho \frac{\partial^2 v }{\partial t^2} &= \frac{\partial^2 v}{\partial x^2} \label{eq:1.2.2c}
\end{align}
\subsubsection{Solution to wave equation and plot}
Starting with:
\begin{equation}
    v=e^{-\left(x- \frac{1}{\sqrt{cp}}t\right)^2} \label{eq:1.2.2d}
\end{equation}
Differentiating:
\begin{align}
    \frac{\partial}{\partial x} \left(e^{-\left(x- \frac{1}{\sqrt{cp}}t\right)^2}\right) &= -2 \left(x - \frac{1}{\sqrt{cp}}t\right)e^{-\left(x- \frac{1}{\sqrt{cp}}t\right)^2} \label{eq:1.2.2a}\\
    \frac{\partial}{\partial t} \left(e^{-\left(x- \frac{1}{\sqrt{cp}}t\right)^2}\right) &= \frac{2}{\sqrt{cp}}\left(x - \frac{1}{\sqrt{cp}}t\right)e^{-\left(x- \frac{1}{\sqrt{cp}}t\right)^2}\label{eq:1.2.2b}
\end{align}
Expanding and differentiating \ref{eq:1.2.2a} with respect to $x$:
\begin{multline}
    \frac{\partial}{\partial x} \left(-2xe^{-\left(x- \frac{1}{\sqrt{cp}}t\right)^2} + \frac{2t}{\sqrt{cp}}e^{-\left(x- \frac{1}{\sqrt{cp}}t\right)^2}\right) = \\
    -2e^{-\left(x- \frac{1}{\sqrt{cp}}t\right)^2} + 4x\left(x- \frac{1}{\sqrt{cp}t}\right)e^{-\left(x- \frac{1}{\sqrt{cp}}t\right)^2} - \frac{4t}{\sqrt{cp}}\left(x - \frac{1}{\sqrt{cp}}t\right)e^{-\left(x- \frac{1}{\sqrt{cp}}t\right)^2}
\end{multline}
Factorising and simplifying:
\begin{align}
    \frac{\partial^2 v}{\partial x^2} &= 4\left(x - \frac{1}{\sqrt{cp}}t\right)^2 e^{-\left(x- \frac{1}{\sqrt{cp}}t\right)^2} -2e^{-\left(x- \frac{1}{\sqrt{cp}}t\right)^2}
\end{align}
Expanding and differentiating \ref{eq:1.2.2b} with respect to $y$:
\begin{multline}
    \frac{\partial}{\partial x} \left(\frac{2x}{\sqrt{cp}}e^{-\left(x- \frac{1}{\sqrt{cp}}t\right)^2} - \frac{2t}{cp}e^{-\left(x- \frac{1}{\sqrt{cp}}t\right)^2}\right) = \\
    - \frac{4x}{cp}\left(x - \frac{1}{\sqrt{cp}}t\right)e^{-\left(x- \frac{1}{\sqrt{cp}}t\right)^2} - \frac{2}{cp}e^{-\left(x- \frac{1}{\sqrt{cp}}t\right)^2} + \frac{4t}{cp\sqrt{cp}}\left(x - \frac{1}{\sqrt{cp}}t\right)e^{-\left(x- \frac{1}{\sqrt{cp}}t\right)^2}
\end{multline}
Factorising and simplifying:
\begin{equation}
    \frac{\partial^2v}{\partial t^2} = \frac{4}{cp}\left(x - \frac{1}{\sqrt{cp}}t\right)^2 e^{-\left(x- \frac{1}{\sqrt{cp}}t\right)^2} -\frac{2}{cp}e^{-\left(x- \frac{1}{\sqrt{cp}}t\right)^2}
\end{equation}
Multiplying by $cp$:
\begin{equation}
    cp\frac{\partial^2v}{\partial t^2} = 4\left(x - \frac{1}{\sqrt{cp}}t\right)^2 e^{-\left(x- \frac{1}{\sqrt{cp}}t\right)^2} -2e^{-\left(x- \frac{1}{\sqrt{cp}}t\right)^2}
\end{equation}
Hence, equation \ref{eq:1.2.2c} is satisfied. Plotting \ref{eq:1.2.2d} in MATLAB  for $cp = 1$ and at three discrete time points. 
\lstset{language=Matlab,%
    %basicstyle=\color{red},
    breaklines=true,%
    morekeywords={matlab2tikz},
    keywordstyle=\color{blue},%
    morekeywords=[2]{1}, keywordstyle=[2]{\color{black}},
    identifierstyle=\color{black},%
    stringstyle=\color{mylilas},
    commentstyle=\color{mygreen},%
    showstringspaces=false,%without this there will be a symbol in the places where there is a space
    numbers=left,%
    numberstyle={\tiny \color{black}},% size of the numbers
    numbersep=9pt, % this defines how far the numbers are from the text
    emph=[1]{for,end,break},emphstyle=[1]\color{red}, %some words to emphasise
    %emph=[2]{word1,word2}, emphstyle=[2]{style},    
}
\lstinputlisting{./mCode/q102b.m}
\begin{figure}[H]
    \centering
    \includegraphics[width = \textwidth]{./img/q102b.png}
    \caption{Graphs to show and compare the effect of varying $t$ for flow velocity along the vessel.}
    \label{fig:q306a}
\end{figure}
\section{Vector calculus}
\subsection{Proof that divergence of velocity equals zero}
\begin{proof}
If the fluid is incompressible, our total derivative is zero:
\begin{align}
    \frac{\textrm{D}\rho}{\textrm{D}t} &= 0\\
\end{align}
We can start to derive the divergence of the velocity by rewriting the second term in \ref{eq2.1a}:
\begin{align}
    \frac{\partial \rho}{\partial t} + \nabla \cdot \left(\rho \underline{u}\right) &= 0 \label{eq2.1a}\\ 
    \frac{\partial \rho}{\partial t} + \rho \left(\nabla \cdot \underline{u}\right) + \underline{u} \cdot \left(\nabla \rho\right) &= 0
\end{align}
Looking at the $\nabla \rho$ term:
\begin{align}
    \nabla \rho = \left(\frac{\partial \rho}{\partial x}, \, \frac{\partial \rho}{\partial y}, \, \frac{\partial \rho}{\partial z}\right)
\end{align}
We know that all derivatives of $\rho$ are zero as $\rho$ is a constant, hence:
\begin{align}
    0 + \rho \left(\nabla \cdot \underline{u}\right) + 0 &= 0\\
    \nabla \cdot \underline{u} &= 0
\end{align}
\end{proof}
\subsection{Acceleration of fluid element}
Fluid element acceleration is given by:
\begin{equation}
    \frac{\textrm{D}u}{\textrm{D}t} = \frac{\partial \underline{u}}{\partial t} + \left(\underline{u}\cdot \nabla\right)\underline{u}
\end{equation}
Flow is steady, hence
\begin{align}
    \frac{\textrm{D}u}{\textrm{D}t} &= 0 + \left(\underline{u}\cdot \nabla\right)\underline{u}\\
    &= -\omega y\frac{\partial \underline{u}}{\partial x} + -\omega x \frac{\partial \underline{u}}{\partial y} + 0 \frac{\partial \underline{u}}{\partial z}\\
    &= -\omega y\frac{\partial \underline{u}}{\partial x} + \omega x \frac{\partial \underline{u}}{\partial x} \\
    &= -\omega y \begin{pmatrix}
        0\\
        \omega\\
        0
    \end{pmatrix} + \omega x \begin{pmatrix}
        -\omega\\
        0\\
        0
    \end{pmatrix}\\
    &= \begin{pmatrix}
        -\omega^2 x\\
        -\omega^2 y\\
        0
    \end{pmatrix}
\end{align}
\subsection{Integral}
Considering the volume of an element $V$, where $V$ is the region bounded by the planes $x=0$, $y=0$, $z=0$ and $x+y+z=1$:
\begin{equation}
    \iiint\displaylimits_V \left(xyz\right)\dif z \dif y \dif x\\
\end{equation}
\subsubsection{Area of integration}
\begin{figure}[H]
    \centering
    \includegraphics[height = 7cm]{./img/q203a.jpg}
    \caption{Graph to show area of integration of function.}
\end{figure}
\subsubsection{Find the limits of integration}
We know the volume is bounded by the $x$-$y$, $x$-$z$ and $y$-$z$ planes. Hence, our lower limits are:
\begin{align}
    x = 0, \; y = 0, \; z = 0
\end{align}
Our upper bound is $x+y+z \leq 1$. Hence, the upper bound for $z$ is:
\begin{gather}
    x + y + z \leq 1\\
    z \leq 1 - x - y
\end{gather}
Upper bound for $y$ ($x$-$y$ plane $\rightarrow z = 0$):
\begin{gather}
    x + y \leq 1\\
    y \leq 1 - x
\end{gather}
Upper bound for $x$ ($y=z=0$)
\begin{gather}
    x \leq 1
\end{gather}
\subsubsection{Calculation of triple integral}
\begin{equation}
    \int_0^1 \int_0^{1-x} \int_0^{1-x-y} \left(xyz\right)\dif z \dif y \dif x \label{q2.3a}
\end{equation}
Computing the $z$ integral:
\begin{align}
    &= xy\int_0^{1-x-y} \left(z\right)\dif z\\
    &= xy \left[\frac{z^2}{2}\right]_0^{1-x-y}\\
    &= xy \left[\frac{(1-x-y)^2}{2}-\frac{0^2}{2}\right]\\
    &= \frac{xy}{2} \left(y^2 + x^2 + 2xy -2x -2y + 1\right)\\
    &= \frac{1}{2}\left(xy^3 + x^3y + 2x^2 y^2 -2x^2y - 2y^2x + xy\right) \label{q2.3b}
\end{align}
Inputting \ref{q2.3b} into \ref{q2.3a}:
\begin{align}
    \int_0^1 \int_0^{1-x} \left(\frac{1}{2}\left(xy^3 + x^3y + 2x^2 y^2 -2x^2y - 2xy^2 + xy\right) \right)\dif y \dif x \label{q2.3d}
\end{align}
Computing the $y$ integral:
\begin{align}
    &= \frac{1}{2} \int_0^{1-x}\left(xy^3 + x^3y + 2x^2 y^2 -2x^2y - 2xy^2 + xy\right) \dif y\\
    &= \frac{1}{2}\left[\frac{xy^4}{4} + \frac{x^3y^2}{2} + \frac{2x^2y^3}{3} -x^2y^2-\frac{2xy^3}{3}+ \frac{xy^2}{2}\right]_0^{1-x}\\
    &= \frac{1}{2}\left[\frac{x\left(1-x\right)^4}{4} + \frac{x^3\left(1-x\right)^2}{2} + \frac{2x^2\left(1-x\right)^3}{3} -x^2\left(1-x\right)^2-\frac{2x\left(1-x\right)^3}{3}+ \frac{x\left(1-x\right)^2}{2}\right]
\end{align}
Expanding:
\begin{multline}
    = \frac{1}{2}\left[\frac{x - 4x^2 + 6x^3 - 4x^4 + x^5}{4} \right.+ \frac{x^3 - 2x^4 + x^5}{2} + \frac{2x^2-6x^3 + 6x^4 -2x^5}{3}\\ \left. -\left(x^2-2x^3+x^4\right)-\frac{2x-6x^2 +6x^3 -2x^4}{3}+ \frac{x-2x^2 + x^3}{2}\right]
\end{multline}
Simplifying
\begin{align}
    &= \frac{x^5 -4x^4 + 6x^3 -4x^2 + x}{24}\\
    &= \frac{1}{24}\left(x^5 -4x^4 +6x^3 -4x^2 + x\right) \label{q2.3c}
\end{align}
Inputting \ref{q2.3c} into \ref{q2.3d}:
\begin{equation}
    \int_0^1 \left( \frac{1}{24}\left(x^5 -4x^4 +6x^3 -4x^2 + x\right) \right)\dif x 
\end{equation}
Computing the $x$ integral:
\begin{align}
    &= \frac{1}{24}\left[\frac{x^6}{6}-\frac{4x^5}{5}+\frac{3x^4}{2}-\frac{4x^3}{3} + \frac{x^2}{2}\right]_0^1\\
    &= \frac{1}{24}\left[\frac{1}{6} - \frac{4}{5} + \frac{3}{2} - \frac{4}{3} + \frac{1}{2}\right]\\
    &= \frac{1}{720}
\end{align}
\section{Transforms}
\subsection{Plot of data}
\lstset{language=Matlab,%
    %basicstyle=\color{red},
    breaklines=true,%
    morekeywords={matlab2tikz},
    keywordstyle=\color{blue},%
    morekeywords=[2]{1}, keywordstyle=[2]{\color{black}},
    identifierstyle=\color{black},%
    stringstyle=\color{mylilas},
    commentstyle=\color{mygreen},%
    showstringspaces=false,%without this there will be a symbol in the places where there is a space
    numbers=left,%
    numberstyle={\tiny \color{black}},% size of the numbers
    numbersep=9pt, % this defines how far the numbers are from the text
    emph=[1]{for,end,break},emphstyle=[1]\color{red}, %some words to emphasise
    %emph=[2]{word1,word2}, emphstyle=[2]{style},    
}
\lstinputlisting{./mCode/q301a.m}
\begin{figure}[H]
    \centering
    \includegraphics[width = \textwidth]{./img/q301a.png}
    \caption{Graph to show variation in signal over a period of 100 seconds.}
\end{figure}
\subsection{Plot of Fourier transform}
\lstset{language=Matlab,%
    %basicstyle=\color{red},
    breaklines=true,%
    morekeywords={matlab2tikz},
    keywordstyle=\color{blue},%
    morekeywords=[2]{1}, keywordstyle=[2]{\color{black}},
    identifierstyle=\color{black},%
    stringstyle=\color{mylilas},
    commentstyle=\color{mygreen},%
    showstringspaces=false,%without this there will be a symbol in the places where there is a space
    numbers=left,%
    numberstyle={\tiny \color{black}},% size of the numbers
    numbersep=9pt, % this defines how far the numbers are from the text
    emph=[1]{for,end,break},emphstyle=[1]\color{red}, %some words to emphasise
    %emph=[2]{word1,word2}, emphstyle=[2]{style},    
}
\lstinputlisting{./mCode/q302a.m}
\begin{figure}[H]
    \centering
    \includegraphics[height = 10cm]{./img/q302a.png}
    \caption{Graph to show absolute values of transform in the frequency domain.}
    \label{fig:q302a}
\end{figure}
\subsection{Extraction of patient's cardiac and respiratory cycle}
As seen from Figure \ref{fig:q302a}, we can extract two values from our Fourier transform. The higher peak has a frequency of \SI{0.16}{\hertz} and a period of \SI{6.25}{\second}. This represents the breathing of the subject (9.6 breaths per minute). According to a Cleveland Clinic article on vital signs, the average human breathing rate for adults should be around 12-16 breaths per minute \cite{q3.3.1}. The lower peak has a frequency of \SI{1.2}{\hertz} and a period of \SI{0.83}{\second}. This represents the heartbeat of the subject (72 beats per minute). According to the British Heart Foundation, the average resting heart rate for adults is between 60-100 beats per minute \cite{q3.3.2}.
\subsection{Frequency filter}
\subsubsection{Gaussian functions}
A Gaussian function was generated using MATLAB's "gaussmf" function. $\mu = \pm1.2$. The value for $\sigma$ was selected arbitrarily to de-noise the signal to an appropriate level
\lstset{language=Matlab,%
    %basicstyle=\color{red},
    breaklines=true,%
    morekeywords={matlab2tikz},
    keywordstyle=\color{blue},%
    morekeywords=[2]{1}, keywordstyle=[2]{\color{black}},
    identifierstyle=\color{black},%
    stringstyle=\color{mylilas},
    commentstyle=\color{mygreen},%
    showstringspaces=false,%without this there will be a symbol in the places where there is a space
    numbers=left,%
    numberstyle={\tiny \color{black}},% size of the numbers
    numbersep=9pt, % this defines how far the numbers are from the text
    emph=[1]{for,end,break},emphstyle=[1]\color{red}, %some words to emphasise
    %emph=[2]{word1,word2}, emphstyle=[2]{style},    
}
\lstinputlisting{./mCode/q304a.m}
\begin{figure}[H]
    \centering
    \includegraphics[height = 10cm]{./img/q304a.png}
    \caption{Graph to show filter, centred at positive and negative cardiac frequencies.}
    \label{fig:q304a}
\end{figure}
\subsubsection{Filtered/unfiltered Fourier data comparison}
\lstset{language=Matlab,%
    %basicstyle=\color{red},
    breaklines=true,%
    morekeywords={matlab2tikz},
    keywordstyle=\color{blue},%
    morekeywords=[2]{1}, keywordstyle=[2]{\color{black}},
    identifierstyle=\color{black},%
    stringstyle=\color{mylilas},
    commentstyle=\color{mygreen},%
    showstringspaces=false,%without this there will be a symbol in the places where there is a space
    numbers=left,%
    numberstyle={\tiny \color{black}},% size of the numbers
    numbersep=9pt, % this defines how far the numbers are from the text
    emph=[1]{for,end,break},emphstyle=[1]\color{red}, %some words to emphasise
    %emph=[2]{word1,word2}, emphstyle=[2]{style},    
}
\lstinputlisting{./mCode/q304b.m}
\begin{figure}[H]
    \centering
    \includegraphics[height = 9cm]{./img/q304b.png}
    \caption{Graph to show comparison between filtered and unfiltered FT signal.}
    \label{fig:q304b}
\end{figure}
\begin{figure}[H]
    \centering
    \includegraphics[height = 9cm]{./img/q304b2.png}
    \caption{Graph to show comparison between filtered and unfiltered FT signal (close-up).}
    \label{fig:q304b}
\end{figure}
\subsection{Filtered data}
\lstset{language=Matlab,%
    %basicstyle=\color{red},
    breaklines=true,%
    morekeywords={matlab2tikz},
    keywordstyle=\color{blue},%
    morekeywords=[2]{1}, keywordstyle=[2]{\color{black}},
    identifierstyle=\color{black},%
    stringstyle=\color{mylilas},
    commentstyle=\color{mygreen},%
    showstringspaces=false,%without this there will be a symbol in the places where there is a space
    numbers=left,%
    numberstyle={\tiny \color{black}},% size of the numbers
    numbersep=9pt, % this defines how far the numbers are from the text
    emph=[1]{for,end,break},emphstyle=[1]\color{red}, %some words to emphasise
    %emph=[2]{word1,word2}, emphstyle=[2]{style},    
}
\lstinputlisting{./mCode/q305a.m}
\begin{figure}[H]
    \centering
    \includegraphics[width = \textwidth]{./img/q305a.png}
    \caption{Graph to show filtered data from pulse oximeter.}
    \label{fig:q305a}
\end{figure}
\subsection{Effect of varying the width of Gaussian function}
The code was adjusted to created two additional cases, to make four in total:
\begin{itemize}
    \item Unfiltered data
    \item Gaussian filter with $\sigma = 0.1$
    \item Gaussian filter with $\sigma = 0.01$
    \item Gaussian filter with $\sigma = 0.001$
\end{itemize}
\lstset{language=Matlab,%
    %basicstyle=\color{red},
    breaklines=true,%
    morekeywords={matlab2tikz},
    keywordstyle=\color{blue},%
    morekeywords=[2]{1}, keywordstyle=[2]{\color{black}},
    identifierstyle=\color{black},%
    stringstyle=\color{mylilas},
    commentstyle=\color{mygreen},%
    showstringspaces=false,%without this there will be a symbol in the places where there is a space
    numbers=left,%
    numberstyle={\tiny \color{black}},% size of the numbers
    numbersep=9pt, % this defines how far the numbers are from the text
    emph=[1]{for,end,break},emphstyle=[1]\color{red}, %some words to emphasise
    %emph=[2]{word1,word2}, emphstyle=[2]{style},    
}
\lstinputlisting{./mCode/q306a.m}
\begin{figure}[H]
    \centering
    \includegraphics[width = \textwidth]{./img/q306a.png}
    \caption{Graphs to compare the effect of varying Gaussian filter width on FT signal.}
    \label{fig:q306a}
\end{figure}
Here we can see that adjusting the value of $\sigma$ effects the amount of noise that appears at the base of the peak in the Fourier transformed data. For $\sigma = 0.1$, there is still quite a bit of residual noise. $\sigma = 0.01$ and $\sigma = 0.001$ both do not exhibit any noise at the base, but we can see that for $\sigma = 0.01$, there is a slight flaring at the base.
\begin{figure}[H]
    \centering
    \includegraphics[width = \textwidth]{./img/q306b.png}
    \caption{Graphs to compare the effect of varying Gaussian filter width on signal from pulse oximeter.}
    \label{fig:q306b}
\end{figure}
Here we can see the effect of the residual noise in the $\sigma = 0.1$ case, with relatively large variations in the amplitude of the signal. We can also see the effect of the flared base in the $\sigma = 0.01$ case as a smooth decrease and then increase in the amplitude of the signal. The $\sigma = 0.001$ case represents a virtually perfect signal with a frequency of \SI{1.2}{\hertz}.
\section{Statistics}
\subsection{Confidence interval}
\lstset{language=Matlab,%
    %basicstyle=\color{red},
    breaklines=true,%
    morekeywords={matlab2tikz},
    keywordstyle=\color{blue},%
    morekeywords=[2]{1}, keywordstyle=[2]{\color{black}},
    identifierstyle=\color{black},%
    stringstyle=\color{mylilas},
    commentstyle=\color{mygreen},%
    showstringspaces=false,%without this there will be a symbol in the places where there is a space
    numbers=left,%
    numberstyle={\tiny \color{black}},% size of the numbers
    numbersep=9pt, % this defines how far the numbers are from the text
    emph=[1]{for,end,break},emphstyle=[1]\color{red}, %some words to emphasise
    %emph=[2]{word1,word2}, emphstyle=[2]{style},    
}
\lstinputlisting{./mCode/q401a.m}
\begin{table}[H]
    \centering
    \begin{tabular}{lll}
        \toprule
        & \textbf{Rest} & \textbf{Anticipation}\\
        \midrule
        $n$ & 38 & 42\\
        Mean & 86.7368 & 92.4048 \\
        Standard deviation & 11.2842 & 16.6177\\
        \bottomrule
    \end{tabular}
    \caption{Table to show values of number of elements, means and standard deviations of heart rate data.}
\end{table}
A 95\% confidence interval can be found using \ref{eq:q401a}:
\begin{equation}
    CI = \bar{x}_1 - \bar{x}_2 \pm z_{crit} \sqrt{\frac{\sigma_1^2}{n_1} + \frac{\sigma_2^2}{n_2}} \label{eq:q401a}
\end{equation}
$z_{crit} = 1.96$ for a 95\% confidence interval, hence:
\begin{gather}
    CI = 86.7368 - 92.4048 \pm 1.96 \sqrt{\frac{11.2842^2}{38} + \frac{16.6177^2}{42}}\\
    CI_L = -11.84 \hspace{1cm} CI_H = 0.51
\end{gather} 
$\bar{x}_1 - \bar{x}_2 = -5.68$, which lies in our confidence interval. Hence, we can say that there is not a statistical difference between them.
\subsection{Reasoning for test statistics}
\begin{thebibliography}{00}
    \bibitem{q3.3.1} Cleveland Clinic, "Vital Signs", \url{https://www.hopkinsmedicine.org/health/conditions-and-diseases/vital-signs-body-temperature-pulse-rate-respiration-rate-blood-pressure#:~:text=Respiration%20rates%20may%20increase%20with,to%2016%20breaths%20per%20minute.} Accessed 27/04/21 14:47
    \bibitem{q3.3.2} British Heart Foundation, "What is a normal pulse rate?", \url{https://www.bhf.org.uk/informationsupport/heart-matters-magazine/medical/ask-the-experts/pulse-rate#:~:text=A%20normal%20resting%20heart%20rate,rich%20blood%20around%20the%20body.} Accessed 27/04/21 14:45
\end{thebibliography}
\end{document}