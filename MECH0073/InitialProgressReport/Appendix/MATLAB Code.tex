\definecolor{dkgreen}{rgb}{0,0.6,0}
\definecolor{gray}{rgb}{0.5,0.5,0.5}
\definecolor{mauve}{rgb}{0.58,0,0.82}

\lstset{frame=tb,
  language=MATLAB,
  aboveskip=3mm,
  belowskip=3mm,
  showstringspaces=false,
  columns=flexible,
  basicstyle={\small\ttfamily},
  numbers=none,
  numberstyle=\tiny\color{gray},
  keywordstyle=\color{blue},
  commentstyle=\color{dkgreen},
  stringstyle=\color{mauve},
  breaklines=true,
  breakatwhitespace=true,
  tabsize=3
}
\begin{lstlisting}

%% Using MATLAB with Serial Devices (such as RS-232)
% MATLAB supports communication with serial devices including RS-232 when 
% using the Instrument Control Toolbox.  There are many devices with serial
% interfaces, including gas chronometers, mass spectrometers, 
% imaging devices, pulse oximeters, and instruments.
%
% The toolbox provides a graphical tool that allows you to configure, 
% control, and acquire data from your serial device without writing code. 
% The tool automatically generates MATLAB code that allows you to reuse
% your work.  The code example below was automatically generated by the
% tool.  Type "tmtool" at the MATLAB command line to launch this tool.
%
% This code example below shows you how you can communicate with your
% serial device using MATLAB.  The "*IDN?" command was used which is a 
% typical command for communicating with an instrument.  The commands you
% can use will depend on what your serial device supports.
%% Automatically generating a report in MATLAB
% Press the "Save and Publish to HTML" button in the MATLAB Editor to 
% execute this code example and automatically generate a report of your 
% work with the serial device.
%% Automatically generating MATLAB script for your RS-232 device
% To automatically create your own MATLAB script, launch "tmtool". Open the
% "Hardware" node, open the "Serial" node, select your serial port (such
% as COM1, press the "connect" button.  Once connected, enter your device 
% commands in the right pane, press "Session log" to see the code generated,
% and press "Save Session" to save the code to a MATLAB (.m) file.
%% MATLAB script automatically generated for the RS-232 device
% The following MATLAB script was automatically generated by interacting
% with the device configuration tool provided by the toolbox.  
% Creation time: 03-Oct-2006 20:36:43
% Create a serial port object.
obj1 = instrfind('Type', 'serial', 'Port', 'COM3', 'Tag', '');
% Create the serial port object if it does not exist
% otherwise use the object that was found.
if isempty(obj1)
    obj1 = serial('COM3');
else
    fclose(obj1);
    obj1 = obj1(1)
end
% Connect to instrument object, obj1.
fopen(obj1);
% Communicating with instrument object, obj1.
data1 = query(obj1, '*IDN?');
% Disconnect from instrument object, obj1.
fclose(obj1);
% Clean up all objects.
delete(obj1);

\end{lstlisting}
