\documentclass[11pt]{article}
%\usepackage[subtle]{savetrees}
\usepackage{geometry,parskip,lipsum}
 \geometry{
 a4paper,
 right=15mm,
 left=15mm,
 top=15mm,
 bottom=15mm
 }
\title{
    \textsc{\huge Portfolio 1 }\hfill \textsc{\large MECH0073 MEng Capstone Group Design Project}\\%[0.3cm]
    \textsc{\large Hasha Dar} \hfill \textsc{\large MEng Mechanical Engineering }\\
    \textsc{\large \today} \hfill \textsc{\large University College London }
    %\textsc{\large University College London }\\[0.3cm] % Name of your university/college
    %\textsc{\large MEng Mechanical Engineering  }\\[0.3cm] % Major heading such as course name
    %\textsc{\large MECH0073 MEng Capstone Group Design Project } % Minor heading such as course title
}
\date{}
\begin{document}
\maketitle
\section{Summary of tasks and contributions to the team work up to date}
As the team leader, I have taken on many responsibilities relating to the project. I have been arranging the team meetings between ourselves and with Ivan (Massive Analytic). I also am responsible for leading these meetings, creating agendas and managing the team's immediate action points and long term goals. Predominantly, the team would split into pairs to work on different aspects of a goal at once, or multiple goals at the same time. Currently, I am researching on how to implement the simulation environment within our workflow. This will be discussed further in Section \ref{methods}.

The Belbin Assessment was part of our first week's formative assessment, and during this I organised a meeting between ourselves to introduce one another and discuss the project. The team was required to send a document outlining the personality types of the group to Dr Bele. I collated each individuals results into one document and created the necessary charts and figures. This was sent on time to Dr Bele.

The Initial Progress Report was our first summative assignment and I was responsible for splitting up the work between the team. I was also responsible for writing the introduction, individual and team responsibilities, plan of work and deliverables sections. I was also responsible for the formatting and final editing of the assignment, having decided to typeset in \LaTeX. In an initial meeting, I decided to split the work up into two phases: a research phase and a writing phase. I organised the team to research different aspects of the project and to then write different sections of the coursework. Over the next two weeks up to the deadline, I researched and wrote my parts of the document up. I also called some further meetings for the team to discuss what parts of the document we needed to amend. Before submission, I organised a meeting with Ivan for a sense check and to ensure that we were good on the technical side of things.

After the submission of our Initial Progress Report, the team had to present to a panel from the department - the Interim Presentation. I was responsible for organising the structure of our presentation and again split tasks up within the team. The aim was to present our current ideas, scope and plan of work to the department. I was in charge of the overall structure of the slides and delegated script writing to the rest of the team. I also wrote scripts for my slides (model description and a part of the project goals and strategy slide). I proceeded to make sure that my parts of the script were memorised and practiced my sections until I could deliver them to a high level. I also called the team in early on the day to practice our presentation as a whole group.

After our Interim Presentation, we aimed to complete our research on the simulator and controller so that we can begin building our models. I split the team into three again to focus on ROS: the drone control software, Gazebo: the simulation environment and MATLAB: neural network. So far, I have liaised with Boxi (Treasurer) to purchase the SSDs required and have been researching the implementation of Gazebo with ROS.

\section{Methods used to solve design tasks and description of major results up to date}\label{methods}
In terms of the design tasks, I have been researching heavily into current solutions and general implementations of controllers within Gazebo + ROS. It seems clear from academic research that the Gazebo environment is more than adequate for the task. The previous fourth year team working on this project using a Unity simulation environment. However, due to the fact that the use case is quite different (car versus drone), the Gazebo environment is much better suited to the task. Specifically, we will be using Gazebo to build our 3D environments, and simulate scenarios that the drone will undergo during a typical flight. This will include everything from static flight, manoeuvres and external effects such as crosswinds. The implementation of the drone controller will be via ROS. ROS specifically is used as a controller for the drone within the simulation environment and the team will need to build a model with a PID controller for data collection purposes. The neural network team will use this data to create our Model Predictive Controller after cleaning the data and fine tuning the neural network. To test our Model Predictive Controller, we will use ROS again and benchmark its performance against the PID controller. Some performance parameters the team would like to measure are accuracy (overshoot, rise time, etc.) and computational intensity (how much memory is used, how much CPU is used). Physical testing will help to confirm that the controller is functional outside of a virtual environment and will further confirm the fact that the controller can be reliably trained virtually.

Thus far, we have received feedback for our Initial Progress Report and Interim Presentation. Our marks on the Initial Progress Report were quite low, and from the feedback provided, we did not provide information on some parts of the project. We scored better on our Interim Presentation, however in the feedback provided, there are areas to improve on such as specificity on the justification of the project and delivery of the presentation. I organised a team meeting to discuss these shortcomings and have made notes to improve upon these aspects in future.
\section{Plan for contributions in the rest of the project's time scale}
I plan to contribute much more on the simulation side of the project. I aim to build the simulator with Li, who is working on this aspect as well. This is to be completed by the middle of January as per our Gantt Chart. Once the simulator is completed, I would like to help the other teams with some aspects such as data cleaning (as this will be coming from the simulation directly) and ROS integration.

As the team leader, I will have further responsibilities such as organising the team, keeping track of our goals and ensuring that deadlines are met. I will also have to liaise with Massive Analytic and keep them updated with our progress.
\section{Self-evaluation}
I think that these initial months with the team have been average. Whilst we have met deadlines and produced quality work, I think one thing that I am not fulfilling as team leader is creating enough time with the group together in-person. Currently, we are meeting online at least once a week to discuss for an hour or so. I believe that I need to get the group together much more often to do work as it would help us to work more effectively. This is something I am definitely looking to improve from now on. I think that my work within the Initial Progress Report and Interim Presentation could have been better if I had focused more on the specific requirements of the two assignments. Again, had I created more time where the group worked together rather than working on pieces individually (which I also did), we may have been able to produce a more cohesive and in-depth Initial Progress Report. Similarly, with the Interim Presentation, I asked people to write their own slides after discussing the requirements once. With such a presentation, the needs and requirements are not fully understood in one meeting. If I produced more opportunities for active collaboration in the group, I think we could have produced a much more engaging presentation. We also only practiced the presentation together as a group for one morning, most likely contributing to some of the lack of engagement feedback.

With all of this considered, I believe I need to be a much more active and engaged leader. I think by putting the team together in-person and working together on the assignments will create more rapport and we will be better able to support each other to produce good work. I will be facilitating this as I move forward with the project.
\end{document}