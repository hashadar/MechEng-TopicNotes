\chapter{Properties of materials}
\section{Metals}
\subsection{General Properties}
\begin{itemize}[noitemsep]
  \item High electrical conductivity - "thermal".
  \item Heavy and plastically deformable.
        \begin{itemize}
          \item Plastic - permanent change in dimension.
        \end{itemize}
  \item Shiny when polished.
\end{itemize}
All of these properties are a function of the atomic bonding ("metallic bond"). A metallic bond is traditionally thought to be a sea of delocalised electrons "orbiting" a \emph{regular} array of ions. A regular array is known as a \emph{crystal}. In this structure, we see repulsive and attractive forces from the atoms and subatomic particles. These are all balanced and thus in \emph{equilibrium}. The interatomic distance can be defined as \( a_0 \). The interatomic distance can be modelled as a spring. Initially a tensile (or a compressive) force will generate deformation and thus \( a_0 \) but will attempt to restore to its original shape; i.e. the deformation is recoverable. This is the definition of \emph{elastic} behaviour.

Beyond a certain deflection/deformation, \emph{permanent} deformation is achieved. When the deforming force is removed, only some of the deformation is recovered (the elastic portion) and the plastic deformation remains.
\subsection{Mechanisms of plastic deformation in metals}
\subsubsection{Slip}
This is where planes of atoms (ions) slip/slide past each other. Planes orientated at \SI{45}{\degree} to the tensile of compressive load will slip first. Slip (under tension) generates lengthening of the specimen. However, the \emph{volume} of the specimen does not \emph{change} i.e. the intermolecular distance stays the same. This means that under a tensile force, the specimen becomes \emph{thinner}. Under compression, metals become \emph{shorter/fatter}.
\section{Polymers}
\subsection{Thermoplastics and Thermosets}
\subsubsection{General Properties}
\begin{itemize}[noitemsep]
  \item Thermally and electrically poor conductors.
  \item Low density.
  \item Poor reflectors.
  \item Often flexible (low stiffness).
  \item Often deformable.
        \begin{itemize}[noitemsep]
          \item "Slip" is possible.
          \item Behave plastically.
        \end{itemize}
\end{itemize}
These properties (including the mechanical properties) are due to the \emph{bonding} occurring in polymers. These bonds are \emph{covalent} within a polymer molecule but the existence of secondary bonding (e.g. Van der Waals and H bonding) are very influential on the properties.

Polyethene (or PE) is a chain structure molecule made up of carbon and hydrogen atoms. This structure is \emph{thermoplastic} or \emph{thermopolymer}. Often these chains can be 100,000 carbon atoms long. The carbon-carbon bond is tetrahedral in reality. If such a polymer was to be stretched out linearly, its length would be approximately 20 \si{\micro\meter}. Each molecule is \emph{stable} i.e. all the covalent bonds are satisfied but there must be interaction between molecules; else PE would have no physical strength. Hence, intermolecular bonding \emph{must} be present. There will be a degree of mechanical interaction with such "massive" molecules but secondary bonds (dipole interactions) dominate.

Not all polymers have this structure e.g. \emph{thermosets}. Epoxy resin such as ureaformaldehyde are thermosetting. The molecular structure of thermosets are similar to thermoplastics. However, there are \emph{real} covalent bonds between the chains instead of secondary bonding. These are called \emph{crosslinks}. Crosslinks are created through chemical reaction during the \emph{polymerisation} process. This creates a new substance and have no chemical formula. Thermoplastics are polymerised at the factory. Thermosets are sold as two products (\emph{pre-polymers}), which are to be reacted together to make the thermoset. As a consequence of the crosslinks, the properties differ to those of thermoplastics such as being stiffer, harder and they can also be stronger. They are also brittle (which can lead to weak behaviour).
\subsection{Elastomers}
\subsubsection{General Properties}
\begin{itemize}[noitemsep]
  \item Massive elastic deformation.
  \item No plastic deformation.
  \item Non recyclable (by melting).
\end{itemize}
Elastomers are effectively a subset of thermosets; they contain covalent cross links (not as many) and thus cannot be melted once formed. The crosslinks permit \emph{freedom} of deformation (movement) of the carbon chains but ultimately restrict it and thereby stop \emph{plastic} deformation. The degree of crosslinking \emph{strongly} affects the stiffness of the elastomer. The stiffness increases with the number of crosslinks.
\section{Ceramics}
\subsection{General Properties}
\begin{itemize}[noitemsep]
  \item High electrical resistivity (insulators).
  \item Highly thermally insulating.
  \item Hard.
  \item Brittle (not necessarily weak).
\end{itemize}
Ceramics are inorganic chemical compounds, usually between two or more elements: a metallic and non metallic element. The above properties are all a function of the atomic bonding; often a mix of ionic and covalent bonding. Alumina \((Al_2O_3)\), silica (Quartz) \((SiO_2)\) and silicon carbide \((SiC)\) (hard, abrasive) are all examples of ceramics. Note: the way in which a material fractures is important - brittle failure is rapid and 'clean' such that the pieces fit back together.

Consider the case of NaCl, which has non-directional ionic bonds. Thus, the ionic bond should (in theory) allow some form of slip to occur along planes - hence plastic deformation. Slip is possible but much more difficult than with the metallic bond. This is due to the repulsion of ions and the need to slip in two atom jumps (alternating + - structure between planes).

A covalent bond is very 'rigid' and resists breakage - hence no slip is possible. Very high strengths are observed but also brittle behaviour
\section{Composites}
A physical mixture of the three previous categories (not chemical). GFRP (glass fibre reinforced polymer) and CFRP (carbon fibre reinforced polymer) are examples of composites.

Individual constituents will have their own properties governed by structure and bonding. In some cases, overall properties are a mathematical average of the individual properties e.g. stiffness - glass added to polymer causes an increase in stiffness. However, many properties e.g. stiffness, strength and toughness are also a function of how the constituents are mixed e.g. arrangement, shape and orientation.

Natural composites do exist e.g. wood, which is mixture of cellulose and lignin (both polymers). Cellulose is crystalline and acts as a fibre. Lignin is amorphous and acts as the 'matrix'. A structural composite is a structure optimised to improve the stiffness to weight ratio. Properties of a composite are also about \emph{how} the two constituents \emph{interact} with each other.