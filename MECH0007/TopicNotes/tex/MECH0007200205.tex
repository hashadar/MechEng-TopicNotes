\subsection{Structure}
\subsubsection{Ionic bonding}
The ionic bond is multi-directional and is capable of being disrupted via shear, generally sliding - "slip", but quite hard, due to the need to "slip twice" as to maintain the low energy charge distribution in the lattice structure. It is formed when one electron is donated from one atom to another, creating a positive and negative ion structure. This sort of bonding is only for compounds.
\subsubsection{Covalent bonding}
These can occur between unlike atoms (like ionic bonds) \emph{and} like atoms. In covalent bonding, there are shared electrons in common orbits, which generates highly directional bonds. These are difficult to displace and break; it does not support slip. Hence, it is difficult to displace off equilibrium angles. Stiffness is usually high in covalently bonded materials. Some examples of covalently bonded materials are \( CH_4, \ SiO_2 \) and diamond. Usually they are brittle and hence, susceptible to defects (covalent solids). Polymers are covalently bonded only in part. It is not fair to say that their properties should be similar to covalently bonded materials. For example, we might expect polyethylene to be stiff and strong but this is only partially correct. If we were to test a single molecule, this would be the case. However, it is very difficult to deal with single molecules at a time as a single molecule is only about 20 \si{\micro\meter} in length. The strength of polymers such as polyethylene, is mostly a function of the way in which each molecule interacts with its neighbours. This is done via Van der Waals H bonds and secondary bonds - a dipolar interaction. These are not whole charges; only minor charge differences and thus are weak. By controlling the level of interactions, we can then control properties of such a polymer. For example, in PVC, chlorine is substituted for hydrogen atoms. Chlorine is very electronegative and loves electrons - it generates permanent dipoles. However, in PE, the molecule only contains carbon and hydrogen bonds and it is symmetric so permanent dipoles do not exist. Instead, 'temporal' (in time) dipoles exist, which come and go. These will attract and provide mechanical properties; a sense of strength. This is the definition of Van der Waals bonding.
\subsubsection{Metallic bonding}
Occurs in elements which have insufficient valence electrons to either ionically or covalently bond i.e. middle of the periodic table. Occurs in single elements e.g. gold and in different elements e.g. copper and zinc: brass. Valence are donated to the 'common' surrounding, an array of positive ions. This is non-directional, so slip is easy. Metallic bonding importantly permits alloys - atomic level mixtures, generating 'solid solutions'. These can be random mixtures and of continuously variable concentrations. An alloy has a mix of properties generated from its constituents. The mix will also have a new level of structure, called the micro-structure, that \emph{strongly} affects the mechanical properties.
\subsection{Creating solids}
Consider a liquid, cooling it down yields a solid. A liquid is a random arrangement of 'particles' (atoms or molecules). The particles get closer together as the temperature drops as the amplitude of the thermal vibrations are reduced. They begin to interact and eventually 'structure' themselves so as to minimise their free energy; called 'ordering'. We have two types of ordering: \emph{short range} order and \emph{long range} order. Short range order yields us amorphous solids and long range order yields us crystalline solids. An example of a crystalline solid is \(SiO_2\) which can come together to form \(SiO_4^{4-}\) also known as the basic building block of silica quartz. This is a 3D structure that is electrically neutral. There are four main 'forms' of ordered \(SiO_2\) and one irregular form, where only the short range order persists. This short order form is called silica glass and the regular form is referred to as quartz. These four long order forms of quartz have the same formula \(SiO_2\) but have different structures. Glassy \(SiO_2\) does not exhibit a melting point. Quartz does. Glassy \(SiO_2\) would be described as brittle. Quartz is less brittle - the energy required for fracture is higher than that of glassy \(SiO_2\). Glassy materials are referred to as \emph{amorphous} - supercooled liquids: solid. Materials 'prefer' to be crystalline as it is the lowest energy state, they like to be ordered - the most energy favourable form. To prevent crystallinity, we need to cool liquids rapidly. This is easier if the crystal structure is complex i.e. molecules but for single atoms e.g. metals it is very difficult. Amorphous metals have been produced requiring a cooling rate of six million degrees per second but these have limited use. So assume metals are crystalline.