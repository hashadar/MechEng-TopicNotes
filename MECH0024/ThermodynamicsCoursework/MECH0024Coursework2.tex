\documentclass[11pt]{article}
\usepackage{graphicx}
\usepackage{hyperref}
\usepackage{amsmath}
\usepackage{amsthm}
\usepackage{amssymb}
\usepackage[all=normal,floats,leading,paragraphs,charwidths,tracking,wordspacing]{savetrees}
\usepackage{float}
\usepackage[version = 4]{mhchem}
\usepackage{multirow}
\usepackage{commath}
\usepackage{booktabs}
\usepackage{subcaption}
\renewcommand{\arraystretch}{1.2}
\usepackage{siunitx}
\sisetup{detect-all}
\DeclareSIUnit{\atm}{atm}
\usepackage{listings}
\usepackage{color} %red, green, blue, yellow, cyan, magenta, black, white
\definecolor{mygreen}{RGB}{28,172,0} % color values Red, Green, Blue
\definecolor{mylilas}{RGB}{170,55,241}
\usepackage[a4paper,margin=15mm]{geometry}
\numberwithin{equation}{section}
\setlength{\parskip}{\baselineskip}
\setlength{\parindent}{0pt}
\hypersetup{
    colorlinks=true,
    linkcolor=black,
    filecolor=black,      
    urlcolor=black,
    citecolor=black
}
\urlstyle{same}
\lstset{language=Matlab,%
    %basicstyle=\color{red},
    breaklines=true,%
    morekeywords={matlab2tikz},
    keywordstyle=\color{blue},%
    morekeywords=[2]{1}, keywordstyle=[2]{\color{black}},
    identifierstyle=\color{black},%
    stringstyle=\color{mylilas},
    commentstyle=\color{mygreen},%
    showstringspaces=false,%without this there will be a symbol in the places where there is a space
    numbers=left,%
    numberstyle={\tiny \color{black}},% size of the numbers
    numbersep=9pt, % this defines how far the numbers are from the text
    emph=[1]{for,end,break},emphstyle=[1]\color{red}, %some words to emphasise
    %emph=[2]{word1,word2}, emphstyle=[2]{style},    
}
\begin{document}
\title{\textbf{UCL Mechanical Engineering 2021/2022}\\MECH0024 Thermodynamics Coursework}
\author{RFLH9}
\date{\today}
\maketitle
\tableofcontents
\listoffigures
\listoftables
\newpage
\section{Question 1}
\subsection{Solar power available}
Solar declination:
\begin{gather}
    \theta_d = \SI{-12}{\degree}
\end{gather}
Polar angle at latitude \SI{51.5}{\degree}:
\begin{gather}
    \theta = 90 - 51.5 = \SI{38.5}{\degree}
\end{gather}
Hour angle at 11am:
\begin{gather}
    \phi = \frac{11\cdot 360}{24} -90 = \SI{75}{\degree}
\end{gather}
Solar azimuth angle:
\begin{gather}
    \cos \psi = \cos \theta_d \sin\theta \sin \phi + \sin\theta_d \cos \theta\\
    \psi = \arccos\left[\cos \left(-12\right)\sin\left(38.5\right)\sin\left(75\right) + \sin\left(-12\right)\cos\left(38.5\right)\right]\\
    \psi = \SI{64.82}{\degree}
\end{gather}
Air mass ratio:
\begin{gather}
    M = \frac{L}{h} = \sec\psi\\
    M = 2.35
\end{gather}
\subsection{Viability of proposed provision}
\section{Question 2}
\subsection{a}
\subsubsection{Irreversibility associated with increase in steady flow exergy of steam}
\subsubsection{Maximum theoretical work available}
\subsection{Relative advantages and disadvantages of four primary energy sources utilised in thermal power generation}
\section{Question 3}
\subsection{a}
\subsubsection{Mass of methane present}
\subsubsection{Approximate level of \ce{CO} in exhaust gases}
\subsection{Effects of fuel molecular composition on ignition, temperatures and formation of exhaust pollutants during combustion}
\section{Question 4}
\subsection{a}
\subsubsection{Ideal operating voltage}
\subsubsection{Anode area}
\subsubsection{Heat loss}
\subsection{Limitations of hydrogen oxygen fuel cell vehicles}
\end{document}