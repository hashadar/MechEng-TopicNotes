\documentclass[11pt]{article}
\usepackage{graphicx}
\usepackage{hyperref}
\usepackage{appendix}
\usepackage{amsmath}
\usepackage{amsthm}
\usepackage{amssymb}
\usepackage[moderate]{savetrees}
\usepackage{natbib}
\usepackage{float}
\usepackage{multirow}
\usepackage{commath}
\usepackage{booktabs}
\usepackage{subcaption}
\renewcommand{\arraystretch}{1.2}
\usepackage{siunitx}
\sisetup{detect-all}
\usepackage{listings}
\usepackage{color} %red, green, blue, yellow, cyan, magenta, black, white
\definecolor{mygreen}{RGB}{28,172,0} % color values Red, Green, Blue
\definecolor{mylilas}{RGB}{170,55,241}
\usepackage[a4paper,margin=20mm]{geometry}
\numberwithin{equation}{section}
\setlength{\parskip}{\baselineskip}
\setlength{\parindent}{0pt}
\hypersetup{
    colorlinks=true,
    linkcolor=black,
    filecolor=black,      
    urlcolor=black,
    citecolor=black
}
\urlstyle{same}
\lstset{language=Matlab,%
    %basicstyle=\color{red},
    breaklines=true,%
    morekeywords={matlab2tikz},
    keywordstyle=\color{blue},%
    morekeywords=[2]{1}, keywordstyle=[2]{\color{black}},
    identifierstyle=\color{black},%
    stringstyle=\color{mylilas},
    commentstyle=\color{mygreen},%
    showstringspaces=false,%without this there will be a symbol in the places where there is a space
    numbers=left,%
    numberstyle={\tiny \color{black}},% size of the numbers
    numbersep=9pt, % this defines how far the numbers are from the text
    emph=[1]{for,end,break},emphstyle=[1]\color{red}, %some words to emphasise
    %emph=[2]{word1,word2}, emphstyle=[2]{style},    
}
\begin{document}
\title{\textbf{UCL Mechanical Engineering 2021/2022}\\MECH0024 Coursework}
\author{RFLH9}
\date{\today}
\maketitle
\tableofcontents
\listoffigures
\newpage
\part{SpaceX rocket engine essay}
SpaceX has been developing rocket engines for nearly two decades. The two primary engine families in use today are the Merlin series and the Raptor series. The Draco and SuperDraco series of engines have also been developed by SpaceX, however their role as Reaction Control System (RCS) thrusters and Launch Abort System engines respectively leave them to be not considered in this essay. SpaceX has developed these rocket engines to power a variety of booster rockets, including the Falcon 9, Falcon Heavy and Starship. The roles of these rockets include delivering crew and cargo to space, as well as interplanetary travel. SpaceX places a strong emphasis on reusability and a reduction of `cost to launch'. Hence, the engines powering these flights must be efficient, powerful and reliable to facilitate multiple missions.

The Merlin series of engines power the Falcon 9 and Falcon Heavy rocket boosters. These are two-stage rockets, primarily designed to place cargo and crew into earth orbit. The first stage is powered by nine Merlin `sea level' engines and the second stage is powered by one Merlin `vacuum' engine. There have been a variety of developments over the past years and the current generation of Merlin engines are the Merlin 1D. The Merlin 1D uses Rocket Propellant 1 (RP-1) and liquid oxygen for combustion in a gas-generator staged combustion cycle.
\part{Sea Level and Vacuum Merlin 1D rocket engines analysis}
\section{}
\subsection{Estimation of ideal OF ratio \& comparison with typical value}
\subsection{Estimation of isentropic index $\gamma$}
\subsection{Estimation of molecular weight of combustion products $M_g$}
\subsection{Estimation of heat capacity $c_p$}
\subsection{Estimation of the adiabatic flame temperature of the combustion product in the combustion chamber}
\section{Determination of mass flux through both rocket engines}
\section{Estimation of exit velocity of jet exhaust in both cases}
\section{Estimation of exit pressure of the exhaust in both cases}
\section{}
\subsection{Estimation of  rocket thrust for the Sea Level Merlin 1D during the initial take-off and the Vacuum Merlin 1D operating in space}
\subsection{Estimation of the specific impulse in both cases}
\subsection{Discussion of differences between estimations and reported values}
\section{Sketch of exit flow/shock expansion fan characteristics}
\subsection{At sea level}
\subsection{At an altitude of \SI{5.5}{\kilo\meter}}
\section{Description of other technologies (and their working principles), which accommodate the change in back-pressure on thrust performance}
\end{document}